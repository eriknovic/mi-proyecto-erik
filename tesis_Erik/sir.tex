\chapter{Integraci�n sem�ntica de recursos de informaci�n en una memoria corporativa}
\label{cap:sir}

%revizar este documento C:\Users\Gatito\Dropbox\Gesti�n Sem�ntica\Tarea12O\Integracion Sem�ntica de los recursos.doc
En particular, permiten hacer el proceso de integraci�n (b�squeda y recuperaci�n) de informaci�n significativa de los recursos en una memoria corporativa. Para lograr esta integraci�n, se deben efectuar las siguientes actividades: \textit{1) modelar el conocimiento de los recursos en un formato est�ndar, 2) explotar el conocimiento impl�cito de los recursos y describir el vocabulario (conceptos y relaciones) de la memoria}, y \textit{3) buscar y recuperar la informaci�n sobre los recursos, para responder una pregunta dada.}

\section{Representaci�n del conocimiento de los recursos (modelo de datos)}
Alg�n texto...

\section{Explotaci�n del conocimiento en el modelo (axiomas)}
M�s texto...

\section{B�squeda y recuperaci�n (consulta) del conocimiento de los recursos}
M�s texto...
