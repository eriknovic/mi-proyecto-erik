\chapter{Evaluaci�n experimental}
\label{cap:exp}

Precisi�n
?Es la proporci�n del material recuperado que realmente es relevante, del total de documentos recuperados? [2, 3]. En donde la precisi�n est� dada por la relaci�n entre Documentos Relevantes Recuperados y Documentos recuperados. Y cuyo intervalo est� entre el cero y uno.
Presisi�n =  (Documentos Relevantes Recuperados)/(Total de Documentos recuperados)
Dibujo de ejemplo
Exhaustividad
?Proporci�n de material relevante recuperado del total de documentos que son relevantes, Donde la exhaustividad es inversamente proporcional a la precisi�n? [3]. Igual que en la precisi�n el intervalo est� entre el cero y uno.
Exhasutividad =  (Documentos Relevantes Recuperados)/(Total de Documentos Relevantes en la MC)


\section{Escenarios de experimentaci�n}
Alg�n texto...

\section{Experimentaci�n}
M�s texto...

\section{Resultados}
M�s texto...
