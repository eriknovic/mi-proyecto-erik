\chapter{Algoritmos para le generaci�n de datos simulados}
\label{aped:AlgDS}
%C�digos interfaz de Usuario
\textbf{\textit{Algoritmo para generar datos artificialmente de los recursos persona}}
\begin{enumerate}
\item $N\leftarrow$ n�mero de personas a describir;
\item T�picos de Redes y Telecomunicaciones $TopRyT \leftarrow \{ \theta_1 \dots \theta_k \}$;
\item \textbf{for} $i\leftarrow 1$ \textbf{to} $N$ \textbf{do}
\begin{enumerate}
\item Elegir el nombre, apellido paterno y apellido materno de unas listas predefinidas;
\item Concatenar el nombre y apellidos electos, donde cada palabra sea separada por un espacio en blanco;
\item Guardar este nombre en una lista tipo cola ($Names$);
\end{enumerate}
\item Lista de Nombres $Names \leftarrow \{ nombre_1 \dots nombre_N \}$;
\item $modelo_{rdf} \leftarrow$ Crear un modelo rdf para los recursos persona;
\item \textbf{for} $i\leftarrow 1$ \textbf{to} $N$ \textbf{do}
\begin{enumerate}
\item $\sigma_i \leftarrow$ crear el recurso $i$ y establecer un identificador URI para �ste;
\begin{enumerate}
\item Seleccionar el i-�simo nombre de la lista de nombres ($nombre_i$);
\item Quitar los espacios en blanco de esta cadena;
\item Establecer esta cadena como identificador del recurso;
\end{enumerate}
\item Para cada caracter�stica significativa de una persona elaborar las literales y elegir los objetos de �sta;
\begin{enumerate}
\item Literal \textit{nombre} $\leftarrow$ seleccionar el i-�simo nombre de la lista de nombres ($nombre_i$);
\item Objeto \textit{G�nero} $\leftarrow$ elegir el sexo de la persona a partir del nombre;
\item Literal \textit{A�o} $\leftarrow$ establecer el a�o aleatoriamente en un intervalo del 2000 al 2013;
\item Objeto \textit{Ocupaci�n} $\leftarrow$ elegir una ocupaci�n para la persona de las tres posibles (Estudiante, Profesor o Empleado);

/** \emph{Establecer la edad a partir de la ocupaci�n de la persona} **/
\item \textbf{if} (Ocupaci�n $\equiv$ `Empleado' $\parallel$ Ocupaci�n $\equiv$ `Profesor') \textbf{then}
\begin{itemize}
\item [$ $] Literal \textit{Edad} $\leftarrow$ un n�mero aleatorio en el intervalo de 27 a 57;
\end{itemize}
\item \textbf{else if} (Ocupaci�n $\equiv$ `Estudiante') \textbf{then}
\begin{itemize}
\item [$ $] \textit{Edad} $\leftarrow$ un n�mero aleatorio en el intervalo de 18 a 28;
\end{itemize}

/** \emph{Elegir una organizaci�n de un listado a partir de la ocupaci�n de la persona} **/
\item \textbf{if} (Ocupaci�n $\equiv$ `Estudiante' $\parallel$ Ocupaci�n $\equiv$ `Profesor') \textbf{then}
\begin{itemize}
\item [$ $] Literal \textit{Organizaci�n} $\leftarrow$  se elige una universidad de un listado preestablecido de universidades;
\end{itemize}
\item \textbf{else if} (Ocupaci�n $\equiv$ `Empleado') \textbf{then}
\begin{itemize}
\item [$ $] \textit{Organizaci�n} $\leftarrow$ se elige una organizaci�n de una lista preestablecida de organizaciones;
\end{itemize}
\item Literal \textit{Email} $\leftarrow$ construir el email a partir del nombre de la persona (cambiando los espacios en blanco por guiones bajos), la organizaci�n donde labora y otras palabras especiales;
\item Literal \textit{Sitio Web} $\leftarrow$ construir el URL concatenando el nombre (cambiando espacios en blanco por guiones), la organizaci�n donde labora, la ocupaci�n, una extensi�n de p�ginas web y otras palabras especiales;

/** Elegir otros individuos que se relacionan con esta persona a partir de la ocupaci�n de la persona **/
\item \textbf{if} (Ocupaci�n $\equiv$ `Estudiante') \textbf{then}
\begin{itemize}
\item [$ $] \textit{Conocidos} $\leftarrow$ se eligen dos nombres aleatorios de la lista nombres, cuya ocupaci�n de �stos sea Empleado o Profesor;
\item [$ $] Quitar los espacios en blanco de los nombres;
\item [$ $] Guardar estas cadenas en una lista de conocidos;
\end{itemize}
\item \textbf{else if} (Ocupaci�n $\equiv$ `Estudiante' $\parallel$ Ocupaci�n $\equiv$ `Profesor') \textbf{then}
\begin{itemize}
\item [$ $] \textit{Conocidos} $\leftarrow$ se eligen dos nombres a cinco nombre aleatorios de la lista nombres, cuya ocupaci�n de �stos sea Empleado o Profesor;
\item [$ $] Quitar los espacios en blanco de los nombres;
\item [$ $] Guardar estas cadenas en una lista de conocidos;
\end{itemize}
\item Objeto \textit{Habilidades ling��sticas} $\leftarrow$ se eligen de forma aleatoria de 1 a 3 idiomas de una lista preestablecida de idiomas;
\item Objeto \textit{Conocimientos de RyT} $\leftarrow$ se eligen de manera aleatoria de 5 a 7 T�picos de Redes y Telecomunicaciones ($TopRyT$);
\item Objeto \textit{Competencias profesionales} $\leftarrow$ se eligen de forma aleatoria de 3 a 4 competencias de un listado preestablecido de competencias;
\end{enumerate}
\item Para cada caracter�stica significativa de una persona construir sus aserciones respectivas.
\end{enumerate}
\end{enumerate}

