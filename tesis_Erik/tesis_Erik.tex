\documentclass[12pt,titlepage,letterpaper]{book}
\usepackage{titlesec}
\usepackage{anysize}
\usepackage{graphicx}
\usepackage[nottoc]{tocbibind}
\usepackage{float}
\usepackage{fancyhdr}
\usepackage{color,soul}
%\usepackage[algo2e,boxed, linesnumbered,dotocloa]{algorithm2e}
\usepackage[lined, boxed, linesnumbered, dotocloa]{algorithm2e}
\usepackage{amsfonts}
\usepackage{url}
\usepackage[spanish]{babel}
\usepackage[latin1]{inputenc}
\usepackage[T1]{fontenc}
\usepackage{cite}
\usepackage{alltt}
\usepackage{array}
\usepackage{multirow}
\usepackage{mdwmath}
\usepackage{verbatim}
\usepackage{ragged2e}
\usepackage[titletoc, toc, page]{appendix}
\usepackage{acronym}
\usepackage{threeparttable}
\usepackage{subfigure}

\usepackage{longtable}
\usepackage{booktabs}
%\newcommand{\pref}{\ref}
\usepackage{algorithmic}
\usepackage{tabbing}

\definecolor{black}{rgb}{0,0,0}
\setulcolor{black}

\newcommand{\bigrule}{\titlerule[0.5mm]}
\titleformat{\chapter}[display] % cambiamos el formato de los cap�tulos
{\bfseries\Huge} % por defecto se usar�n caracteres de tama�o \Huge en negrita
{% contenido de la etiqueta
 \titlerule % l�nea horizontal
 \filleft % texto alineado a la derecha
 \Large\chaptertitlename\ % "Cap�tulo" o "Ap�ndice" en tama�o \Large en lugar de \Huge
 \Large\thechapter} % n�mero de cap�tulo en tama�o \Large
{0mm} % espacio m�nimo entre etiqueta y cuerpo
{\filleft} % texto del cuerpo alineado a la derecha
[\vspace{0.5mm} \bigrule] % despu�s del cuerpo, dejar espacio vertical y trazar l�nea horizontal gruesa
\pagestyle{fancy}
\fancyhf{}
\fancyhead[LO]{\leftmark} % En las p�ginas impares, parte izquierda del encabezado, aparecer� el nombre de cap�tulo
\fancyhead[RE]{\rightmark} % En las p�ginas pares, parte derecha del encabezado, aparecer� el nombre de secci�n
\fancyhead[RO,LE]{\thepage} % N�meros de p�gina en las esquinas de los encabezados

\renewcommand{\chaptermark}[1]{\markboth{\textbf{\thechapter. #1}}{}} % Formato para el cap�tulo: N. Nombre
\renewcommand{\sectionmark}[1]{\markright{\textbf{\thesection. #1}}} % Formato para la secci�n: N.M. Nombre

\renewcommand{\headrulewidth}{0.6pt} % Ancho de la l�nea horizontal bajo el encabezado
\renewcommand{\footrulewidth}{0.6pt} % Ancho de la l�nea horizontal sobre el pie (que en este ejemplo est� vac�o)
\setlength{\headheight}{1.5\headheight} % Aumenta la altura del encabezado en una vez y media
%\marginsize{3cm}{2cm}{2.5cm}{2.5cm}% Margenes de la pagina

%\def\listfigurename{Lista de Figuras}
%\def\listtablename{Lista de Tablas}
%\def\contentsname{Contenido}
%\def\bibname{Referencias}
\setcounter{totalnumber}{30}

\begin{document}
%\maketitle
\def\listtablename{Lista de Tablas}
\def\listfigurename{Lista de Figuras}
\def\contentsname{Contenido}

\pagenumbering{roman}
\begin{titlepage}
\thispagestyle{empty}
\begin{figure} [htbp]
\begin{center}
\includegraphics[width=0.6\textwidth]{uam}
%\caption{Modelo de simulaci�n para el SAD.}
%\label{fig:modsadtf}
\end{center}
\end{figure}
\vspace{2cm}

\begin{center}
\Large
\textbf{Integraci�n Sem�ntica de Recursos en una Memoria Corporativa}\\
\vspace{1cm}
\large
Id�nea Comunicaci�n de Resultados para obtener el grado de\\
\vspace{0.7cm}
\textsc{Maestro en Ciencias}\\
\textsc{(Ciencias y Tecnolog�as de la Informaci�n)}\\
por\\
Erik Alarc�n Zamora\\
\vspace{1cm}

Asesores:\\
Dra. Reyna Carolina Medina Ram�rez\\
\vspace{0.7cm}
Dr. H�ctor P�rez Urbina\\
\vspace{3cm}
M�xico, D.F. Enero 2014
\end{center}
\vfill
\end{titlepage}
\setcounter{page}{2}
\begin{center}
\end{center}

\thispagestyle{empty}
\newpage

\chapter*{Resumen}
El �rea de Redes y Telecomunicaciones (RyT) del departamento de Ingenier�a El�ctrica (IE) de la Universidad Aut�noma Metropolitana (UAM) tiene una amplia y rica variedad (heterogeneidad en formato, contenido y estructura) de recursos de informaci�n. Algunos ejemplos de estos recursos de informaci�n son: los profesores y alumnos del departamento IE, art�culos cient�ficos, notas de curso, bases de datos de los trabajadores del dpto. IE, libros, presentaciones, manuales, inventarios,  especificaciones de circuitos el�ctricos.

Cada recurso representa el conocimiento sobre investigaciones, colaboraciones, proyectos, cursos y temas de inter�s de los profesores y alumnos en el dominio RyT.  Por ejemplo, los art�culos cient�ficos, presentaciones, notas de curso e inclusive el propio profesor autor de estos documentos y multimedia son fuentes de informaci�n. Todo el conocimiento de una organizaci�n representado a trav�s de los recursos, se conoce como memoria corporativa \cite{Ontoinra2002}.

Una adecuada gesti�n del conocimiento en una memoria corporativa (MC) se traduce en varias ventajas a nivel operacional, como: una organizaci�n bien informada y con mejores tomas de decisi�n, una herramienta de aprendizaje para las personas adscritas a la organizaci�n, una base de conocimiento persistente y accesible para estas personas, un instrumento para b�squeda, recuperaci�n e intercambio de conocimiento entre personas, por mencionar algunas.

Para llevar a cabo esta gesti�n de los recursos en una MC, se necesitan dos operaciones: 1) la representaci�n del conocimiento sobre los recursos y 2) la b�squeda sobre esta representaci�n. En las tecnolog�as de la Informaci�n, hay varios enfoques tradicionales de representar/buscar el conocimiento de los recursos, como: motores de b�squeda  sint�cticos y bases de datos relacionales. Pero, el enfoque que nos llam� la atenci�n, es el de las Tecnolog�as Sem�nticas.

Las Tecnolog�as Sem�nticas se basan en el uso de tecnolog�as, herramientas y est�ndares para: la representaci�n de los recursos en un formato est�ndar, establecer un vocabulario conceptual, la explotaci�n del conocimiento mediante reglas, la busqueda y recuperaci�n de la informaci�n a partir de la representaci�n est�ndar, el uso de aplicaciones gen�ricas para la creaci�n, manipulaci�n y visualizaci�n de la informaci�n sobre los recursos, y para que los expertos en el dominio sean los encargados de suministrar y evaluar la informaci�n sobre los recursos.

En esta tesis de maestr�a, se propone una metodolog�a para la representaci�n, b�squeda, explotaci�n e integraci�n del conocimiento de los recursos de informaci�n en una memoria corporativa, mediante el uso de tecnolog�as sem�nticas. Esta metodolog�a est� guiada por dos casos de uso base y  la memoria corporativa es del �rea de RyT de la UAM.

\begin{itemize}
    \item El primer caso de uso (Cartograf�a de competencias) consiste en la b�squeda de las personas (adscritas o relacionadas al depto. IE) a partir de sus caracter�sticas profesionales. En particular, se buscan a las personas por las competencias de profesionales, ling��sticas y sobre los temas que conocen de Redes y Telecomunicaciones. Por ejemplo, "todos los profesores de la UAM con conocimientos en radios cognitivos y que lean en ingl�s". Este primer caso tambi�n  contempla la b�squeda de profesores que pueden impartir un curso, a partir de un conjunto de temas b�sicos que debe saber para dicho curso.
	 \item El segundo caso de uso (B�squeda de recursos digitales) consiste en la b�squeda de documentos y archivos multimedia, con base a uno o varios criterios de b�squeda (autor, t�tulo, a�o, temas de RyT, entre otros). Por ejemplo, "todos los art�culos de Tim Berners Lee sobre Web Sem�ntica y mayores al 2009".
\end{itemize}

La metodolog�a para el desarrollo del modelo, la explotaci�n y la integraci�n del conocimiento sobre los recursos en una MC, se ha dividido en varias etapas que concuerdan con cada uno de los objetivos de la tesis. Los objetivos de la tesis son los siguientes:

\begin{itemize}
    \item Un modelo (representaci�n del conocimiento) de los recursos a partir de los dos casos de uso en un formato est�ndar.
    \item Un modelo coherente y del cual se explote el conocimiento sobre los recursos (ontolog�a), a partir del uso de axiomas y un programa razonador.
    \item La busqueda y recuperaci�n (integraci�n) de los recursos que satisfagan las necesidades informativas de los usuarios, a partir de un motor de consulta.
    \item Un prototipo (navegaci�n y consultas espec�ficas) para la interacci�n f�cil y visual de  los usuarios con el modelo .
    \item Evaluar los resultados devueltos y el tiempo de ejecuci�n de las consultas a la ontolog�a.
\end{itemize}

En las tecnolog�as de la web sem�ntica, el marco de descripci�n de recursos (RDF) es la soluci�n para la representaci�n del conocimiento de manera formal sobre los recursos en la MC. La representaci�n se basa en la descripci�n de las caracter�sticas significativas o relaciones sem�nticas de/entre los recursos. Por ejemplo, Jorge Aparicio Reyes tiene 29 a�os, vive en el Estado de M�xico, lee en Ingl�s, conoce a Erik Alarc�n, estudia en la UAM y tiene conocimientos en sistemas operativos, java y flash.

Si bien cada recurso de la MC tiene un nombre propio, en el marco RDF cada persona, documento, multimedia o concepto tiene un identificador �nico de recurso \cite{BLURI98} (URI). Con la finalidad de no tener ambig�edades a la hora de referirse a un recurso. Por ejemplo, el URI de Jorge Aparicio es http://www.mi-ejemplo.com/JorgeAparicio. Para cada recurso (identificado con URI) se describen las caracter�sticas/relaciones en forma de triples (sujeto-predicado-objeto) y cada elemento de un triple es un URI o en algunos casos el objeto es una Literal.

Esta representaci�n de las caracter�sticas se encuentra en un formato est�ndar y para almacenar estos triples, se emplea un triplestore. En este trabajo de tesis se emple� el triplestore Apache Jena que proporciona almacenamiento, un motor de consulta y un razonador.

Las descripciones representan la informaci�n expl�cita de los recursos, pero, esta informaci�n expl�cita tiene conocimiento impl�cito. Por ejemplo, un alumno, ni�o, profesor, empleado, madre, hijo son personas, pero �stas como tal no tienen un triple que establezca que son personas. Entonces, para explotar este conocimiento impl�cito de los recursos, se proponen un conjunto de reglas o axiomas que permiten establecer estas relaciones. Aunque, para materializar estos triples a partir de los axiomas, es necesario un programa razonador que infiera estos triples. Este razonador tambien permite encontrar inconsistencias en el modelo. Algunos triplestores integran o permiten importar un razonador, en el caso de Jena permite las dos opciones.

El modelo que captura el conocimiento expl�cito (descripciones) de los recursos y los axiomas que completan el conocimiento sobre �stos, se denomina ontolog�a. En est� tesis se hicieron dos ontolog�as; una para cada caso de uso, y tambien se modific� una ontolog�a legada que tiene conceptos del �rea de RyT. Esta �ltima ontolog�a se emplea para vincular a personas, documentos y multimedia con los t�picos de RyT. 

La consulta de los triples en el modelo, ya sea �nicamente descripciones (triples expl�citos) o una ontolog�a con razonador (triples expl�citos e inferidos), se hace con un motor de b�squeda (integrado en el triplestore) que compara los triples con un conjunto patrones; aquellos triples que concuerden, se recuperar� la informaci�n que se solicit� en la consulta.

Un motor de consulta y un razonador que materializa triples en una ontolog�a, son una buena combinaci�n, ya que permiten consultar el conocimiento inferido (triples inferidos) y reducir la complejidad de las consultas. Por ejemplo, se tienen seis individuos que afirman que son alumno, ni�o, profesor, empleado, madre, hijo respectivamente, tambien se tienen los axiomas que establecen que alumno, ni�o, profesor, empleado, madre, hijo son personas y se tiene la siguiente pregunta "Qui�nes son personas". Si se emplea solamente un motor de b�squeda, entonces no habr� ning�n resultado, pero si se emplea la combinaci�n motor y razonador, los seis individuos ser�n respuesta, porque estos seis individuos tienen el triple que afirma que son personas.

Los usuarios del �rea de RyT no est�n familiarizados con las tecnolog�as sem�nticas y en particular, al uso de la sintaxis de consulta. Entonces para facilitar a �stos la interacci�n y consulta del conocimiento de la ontolog�a, se propone un prototipo que medie (interfaz) entre los usuarios y la ontolog�a, espec�ficamente este prototipo tiene los siguientes objetivos:

\begin{itemize}
    \item Navegaci�n a trav�s de la informaci�n de los recursos; guiada por los casos de uso.
    \item Estructurar la pregunta de un usuario.
    \item Mapear las preguntas a consultas para el motor de consulta.
    \item Ejecutar la consulta con el motor de consulta, el razonador y la ontolog�a.
    \item Publicar la informaci�n de los recursos respuesta en un formato visual agradable al usuario.
\end{itemize}

En esta tesis dos de los aspectos importantes a evaluar son: el desempe�o de Apache Jena a la hora de consultar la ontolog�a, as� como el n�mero y cu�les resultados responden estas consultas. Para llevar a cabo estas dos evaluaciones se obtuvieron un conjunto b�sico de preguntas para interrogar el modelo, para cada pregunta se sabe de ante manos el n�mero y los recursos que la responden. En la primer evaluaci�n, para cada consulta b�sica se calcula 20 veces el tiempo aproximado en milisegundos y se saca un tiempo promedio. Mientras, en la segunda evaluaci�n, para cada consulta se compara el n�mero/recursos que responde el motor con los recursos que previamente se sabe que la responden.

Las contribuciones de esta tesis son:
\begin{enumerate}
    \item Una metodolog�a para la Integraci�n Sem�ntica de Recursos en la MC de Redes y Telecomunicaciones.
    \item Identificaci�n y descripci�n de los principales escenarios de b�squeda/recuperaci�n de los recursos en la MC de RyT.
    \item Ontolog�as (Triples RDF + axiomas) que capturan el conocimiento de los recursos (apegados a los dos casos de uso) en la memoria corporativa RyT.
    \item Prototipo para la consulta interactiva de los usuarios con las ontolog�as de RyT.
    \item Evaluaci�n del desempe�o y calidad de resultados del triplestore Jena para la consulta de informaci�n.
\end{enumerate}

\chapter*{Agradecimientos}
Durante la realizaci�n de este trabajo, varias personas me han acompa�ado, soportado y han dado su apoyo incondicional.

Muchas gracias a:

Mis asesores, la Dra. Carolina Medina Ram�rez y el Dr. H�ctor P�rez Urbina por aceptarme y guiarme en este grandioso proyecto, tambi�n por sus consejos, comentarios, correcciones y ense�anzas.

\tableofcontents
\listoftables
\listoffigures
\renewcommand{\tablename}{Tabla}

\chapter*{Acr�nimos}

\begin{center}
\small
\begin{longtable}{lp{3.0in}c}
\toprule
\multicolumn{1}{c}{Acr�nimo}
                & \multicolumn{1}{c}{Descripci�n}
                                & \multicolumn{1}{c}{Definici�n}\\ \midrule\addlinespace[2pt] \endhead
\bottomrule\endfoot
RyT				& Redes y Telecomunicaciones
                                & \pageref{sym:RyT}\\
IE			&  Ingenier�a El�ctrica
                                & \pageref{sym:IE}\\
UAMI			&  Universidad Aut�noma Metropolitana Unidad Iztapalapa
                                & \pageref{sym:UAMI}\\
MC			&  Memoria Corporativa
                                & \pageref{sym:MC}\\
MO			&  Memoria Organizacional
                                & \pageref{sym:MO}\\
TI			&  Tecnolog�as de la Informaci�n
                                & \pageref{sym:TI}\\
                                
MBSK			&  Motores de B�squeda Sint�cticos basados en Keywords
                                & \pageref{sym:mbsk}\\
GBDR			&  Gestor de Bases de Datos Relacional
                                & \pageref{sym:gbdr}\\
BD			&  Base de Datos
                                & \pageref{sym:BD}\\
TS			&  Tecnolog�as Sem�nticas
                                & \pageref{sym:TS}\\
\end{longtable}
\end{center}
\addcontentsline{toc}{chapter}{Acr�nimos}
\thispagestyle{empty}
\cleardoublepage

\justifying
\pagenumbering{arabic}
%Cap�tulos de la tesis
\section{Introducci�n}
\label{cap:intro}
Las personas todos los d�as est�n en contacto con diferentes organizaciones. Por ejemplo, el ni�o que asiste a la escuela primaria, el estudiante que asiste a la universidad, la ama de casa que compra productos en una tienda departamental, la persona que hace un dep�sito o cobra en una instituci�n bancaria, la persona que solicita un servicio en alguna dependencia gubernamental, el empleado que trabaja en una empresa, inclusive una familia es una organizaci�n.

El concepto de organizaci�n tiene diferentes definiciones, nosotros elegimos la siguiente definici�n: ``\textit{una organizaci�n es una entidad a trav�s de la cual las personas realizan actividades y de las cuales por lo menos algunas se dirigen a la consecuci�n de fines comunes (metas) de las personas del grupo}'' \cite{TeoOrg-James}. De esta definici�n, se tiene que una organizaci�n alcanza mayores logros, porque varias personas se coordinan y dirigen sus esfuerzos conjuntamente. Las organizaciones deben poner atenci�n en las siguientes actividades para alcanzar sus metas y objetivos \cite{TeoOrg-Richard}:

\begin{enumerate}
    \item Reunir recursos para alcanzar las metas y los resultados deseados.
    \item Producir bienes y servicios de manera eficiente.
    \item Buscar formas innovadoras de producir y distribuir con mayor eficiencia bienes y servicios.
    \item Utilizar tecnolog�as de informaci�n y manufactura.
    \item Adaptar, evolucionar e influir en un entorno que cambia con rapidez.
    \item Crear valor para due�os, empleados y clientes.
    \item Hacer frente y adaptarse a los cambios que plantea la diversidad del mundo laboral, problemas �ticos, responsabilidad social y coordinaci�n de los empleados.
\end{enumerate}

La administraci�n es un concepto importante para una organizaci�n y �ste se define como: ``un conjunto de actividades dirigido a aprovechar los recursos de manera eficiente y eficaz con el prop�sito de determinar y alcanzar los objetivos de la organizaci�n'' \cite{TeoAdmon-Reinaldo}. A partir de esta definici�n, se tienen dos elementos importantes: actividades y recursos. Las actividades en una organizaci�n pueden ser \textit{b�squeda de informaci�n, almacenamiento de los recursos, intercambio de informaci�n, control de bienes y materiales, control de inventario, colaboraci�n con otras personas}, por mencionar algunas. Mientras, los \textit{recursos} son ``el medio que posee una organizaci�n para realizar las actividades que le permitan lograr los objetivos'' \cite{DisOrg-Gilli}. Una organizaci�n puede tener los siguientes recursos: materiales o f�sicos, humanos (personas), financieros (dinero) e inform�ticos. La finalidad de la administraci�n en una organizaci�n es que �sta sea estable, crezca y prospere.

La administraci�n en una organizaci�n tiene diferentes enfoques que dependen de los principales elementos de la misma, por ejemplo: las metas, el proceso interno y los recursos. En particular, nuestro foco de atenci�n son los recursos de informaci�n. Porque �stos son los instrumentos que representan y encapsulan el conocimiento de una organizaci�n. Algunos ejemplos de estos recursos son: una persona, una base de datos, un libro, un archivo multimedia, informes anuales, por mencionar algunos.

La administraci�n de los recursos puede realizarse con alguna herramienta de las Tecnolog�as de la Informaci�n. La finalidad de estas herramientas es facilitar, eficientar y agilizar las actividades relacionadas con la administraci�n de los recursos. Por un lado, el enfoque manual consiste en almacenar y organizar los recursos digitales (documentos, archivos de audio, presentaciones, documentos escaneados, etc) en carpetas que tienen cierta estructura. Por otro lado, el enfoque autom�tico permite delegar ciertas tareas de gesti�n a programas computacionales; las dos herramientas comunes de este enfoque son: los sistemas gestores de bases de datos relacionales y los motores de b�squeda basados en keywords.

Un \textit{motor de b�squeda} \cite{MBSPap} es un sistema de recuperaci�n de la informaci�n que a partir de las palabras clave, realiza una b�squeda documental. Este motor responde al usuario con aquellos documentos que en su contenido tienen las palabras clave. Mientras, un \textit{gestor de bases de datos relacional} es un mecanismo para el almacenamiento y recuperaci�n de la informaci�n sobre una Base de Datos. Estos gestores se basan en esta idea: \textit{la base de datos es percibida como un conjunto de tablas (relaciones) bajo un mismo contexto, donde, una tabla es una matriz que guarda datos} \cite{SGBDDIA}. Un gestor emplea un esquema conceptual para las tareas de almacenamiento de informaci�n. El esquema permite describir un conjunto de objetos, aspectos relevantes y las interrelaciones de/entre estos, as� como restricciones de integridad. Para fines de recuperaci�n de la informaci�n, se emplean lenguajes de consulta para las bases de datos. %En general, cualquiera de estas dos herramientas tiene  un menor tiempo en el proceso de b�squeda, pero la calidad de los resultados va depender de los algoritmos y de la representaci�n de los recursos.

El enfoque manual y las dos herramientas del enfoque autom�tico tienen algunos detalles que dificultan la gesti�n en los recursos de una organizaci�n. En el caso de una soluci�n manual, si hay un crecimiento explosivo de los archivos (recursos digitales), entonces la b�squeda de recursos se vuelve un proceso tardado, pesado y cansado para las personas. Mientras que las dificultades del enfoque autom�tico son:

\begin{enumerate}
\item Un motor de b�squeda en ocasiones recupera documentos innecesarios para los usuarios.
\item Un motor proporciona resultados inadecuados, cuando existe ambig�edad en las palabras.
\item Una representaci�n deficiente en una BD relacional, puede causar anomal�as en los datos encontrados cuando el modelo crece.
\item Un modelo relacional inadecuado propicia a tener datos inconsistentes, lo que provoca, problemas en la generaci�n y validaci�n de la informaci�n \cite{SGBDDIA}.
\item Perdida de informaci�n en un modelo, cuando se representan las atributos sobre los recursos \cite{SGBDDIA}.
\end{enumerate}

Las \textit{tecnolog�as sem�nticas} \cite{SemTecRetr} son un conjunto de metodolog�as, lenguajes, aplicaciones, herramientas y est�ndares, para obtener y suministrar el significado de la informaci�n\footnote{L. Feigenbaum, \textquotedblleft Semantic Web vs. Semantic Technologies\textquotedblright, Disponible en: \url{http://www.cambridgesemantics.com/semantic-university/semantic-web-vs-semantic-technologies}}. Estas tecnolog�as permiten representar y administrar el conocimiento, por ello, son una soluci�n interesante para la administraci�n de los recursos en una organizaci�n. A continuaci�n, se presentan los beneficios del uso de �stas:

\begin{itemize}
    \item \textbf{Formato est�ndar}: una persona, documento, objeto f�sico o digital, concepto, idea, en general, cualquier recurso posee informaci�n significativa y �til para las personas. Esta informaci�n puede estar incrustada en el recurso o puede ser referente a �ste, por ejemplo, en un libro nos interesa saber sobre qu� trata, el t�tulo, los autores, la fecha de edici�n, entre otros. Por otro lado, los datos de los recursos pueden ser de distintas formas: estructurados (bases de datos), semiestructurados (lenguajes de etiquetas, como XML y HTML) o sin estructura (orientados al texto). Tambi�n, la informaci�n de un recurso puede expresarse en distintos \textit{tipos de archivo}, por ejemplo en un documentos digital (doc, pdf, odp), una presentaci�n (ppt, pdf) o un v�deo (mpeg, avi, mp4). Esta diversidad en los recursos hace dif�cil la administraci�n de los mismos. Por ello, las tecnolog�as proponen representar los recursos a trav�s de sus caracter�sticas significativas en un formato est�ndar, para que, los procesos autom�ticos puedan acceder, procesar, razonar, combinar, reutilizar y compartir esta informaci�n.
    \item \textbf{Enriquecer el conocimiento}: las tecnolog�as sem�nticas permiten la introducci�n de reglas de inferencia para enriquecer el modelo de conocimiento impl�cito. La finalidad de estas reglas es que un programa especial realice inferencia sobre �stas para hacer expl�cito el conocimiento impl�cito. De esta manera, los procesos autom�ticos pueden aprovechar este conocimiento, para fines de b�squeda de informaci�n. Por ejemplo, una persona, un perro y un gato pertenecen al campo sem�ntico mam�feros, si se introduce la regla que establece que todo gato, perro o persona es un mam�fero, entonces, un proceso autom�tico podr� identificar quienes son mam�feros.
    \item \textbf{Flexibilidad e interoperabilidad}: una caracter�stica importante en las tecnolog�as sem�nticas es la flexibilidad. Esta caracter�stica se refiere a la facilidad para representar y mantener el conocimiento de un dominio. Esta representaci�n se basa en la descripci�n de los recursos a partir de sus caracter�sticas significativas y relaciones en un formato est�ndar. Otra caracter�stica relacionada a la flexibilidad, es la interoperabilidad. Este concepto se refiere a que gracias a los est�ndares pueden emplearse una variedad de herramientas y aplicaciones.
\end{itemize}

Existen distintos tipos de organizaciones que dependen del enfoque con el que se mira. Si es con respecto al alcance, se tienen corporaciones multinacionales, peque�os y medianos negocios, as� como negocios familiares. Cuando el enfoque es el objeto final, se tienen organizaciones que fabrican productos o proveen servicios. Si es a partir de la naturaleza de la organizaci�n, se tienen instituciones econ�micas (empresas), fundaciones, organizaciones sin fines de lucro e instituciones p�blicas.

Esta tesis de maestr�a se enfoca en las organizaciones de investigaci�n (institutos o universidades), porque tienen �reas o equipos de investigaci�n. En concreto, la organizaci�n seleccionada como caso de estudio es el  grupo de investigaci�n del \textit{�rea de Redes y Telecomunicaciones} de la \textit{Universidad Aut�noma Metropolitana Unidad Iztapalapa}. Los recursos significativos en esta organizaci�n son: \textit{personas (profesores y alumnos), documentos (art�culos cient�ficos, libros, tesis), bases de datos, archivos multimedia (presentaciones, v�deos, im�genes)}, solo por mencionar algunos. Porque representan el conocimiento de los profesores (miembros de esta organizaci�n) sobre sus investigaciones, colaboraciones, proyectos, actividades, cursos y temas de inter�s. Una adecuada administraci�n de los recursos, se traduce en un grupo de investigaci�n bien informado con mejores tomas de decisiones, as� como una base de conocimiento persistente y accesible para los profesores y alumnos

Esta Id�nea Comunicaci�n de Resultados est� organizada de la siguiente manera:

En el cap�tulo \ref{cap:marcointro} se describe la problem�tica principal de esta investigaci�n, as� como algunos conceptos b�sicos como son memoria corporativa, integraci�n, recurso de informaci�n. Los principales conceptos, definiciones, est�ndares de los elementos pertenecientes a las tecnolog�as sem�nticas, se presentan en el cap�tulo \ref{cap:ets}. En el cap�tulo \ref{cap:soa} se presenta la revisi�n en la literatura para dos actividades. Primero, la revisi�n de trabajos sobre la integraci�n sem�ntica de los recursos en una memoria corporativa. Segundo, revisi�n de las herramientas para desarrollar e implementar la integraci�n sem�ntica de recursos. El cap�tulo \ref{cap:sir} describe nuestra metodolog�a para la integraci�n sem�ntica de recursos en una memoria corporativa. El cap�tulo \ref{cap:piu} describe los objetivos y caracter�sticas del prototipo para la integraci�n sem�ntica de recursos. Las pruebas y resultados (desempe�o y calidad de las respuestas) hechos/obtenidos al gestor del modelo sem�ntico, as� como al modelo para el �rea de redes y telecomunicaciones, se presentan en el cap�tulo \ref{cap:exp}. Finalmente, las conclusiones sobre la integraci�n sem�ntica de los recursos, el uso de las tecnolog�as sem�nticas y los resultados de nuestra experimentaci�n, se presentan en el cap�tulo \ref{cap:concl}. En esta secci�n tambi�n se presentan algunos trabajos futuros que identificamos.
\chapter{Descripci�n del problema}
\label{cap:dop}
El �rea de Redes y Telecomunicaciones del Departamento de Ingenier�a El�ctrica de la Universidad Aut�noma Metropolitana es una organizaci�n que se constituye por un conjunto diverso de personas. Estas personas se pueden clasificar en dos tipos: las que pertenecen al n�cleo del �rea y las temporales. Por un lado, el n�cleo del �rea est� compuesto por trece profesores-investigadores, quienes imparten cursos en la UAM y hacen investigaci�n sobre alguno de los temas del �rea. Por otro lado, las personas temporales son: 1) Los alumnos que realizan alg�n proyecto o servicios social y cuyo responsable de ellos es un profesor del n�cleo, 2) Otros profesores que son contratados temporalmente por la universidad y que tienen conocimiento en los temas del �rea, 3) Empleados de la universidad que proporcionan servicios administrativos a los profesores del n�cleo. 4) Empleados de otras organizaciones que colaboran con los profesores del n�cleo.


El elemento clave en una \textbf{\textit{organizaci�n}} no es un edificio o una serie de pol�ticas, sino el conjunto de personas, porque son �stas quienes realizan las actividades esenciales para alcanzar las metas y objetivos de la organizaci�n.

La importancia de las personas, radica en los siguientes hechos:
Una personas 

Por que estos son quienes hacen el trabajo, aportando sus conocimientos, habilidades y actitudes para bien o para alcanzar las metas de la organizaci�n.

Las personas en una organizaci�n tienen sus propias actitudes, valores, habilidades y objetivos, este hecho es importante porque n

por estas caracter�sticas, es importante que estas personas desempe�en las actividades adecuadas para lograra las metas y objetivos de la organizaci�n.

Estas personas tienen actitudes, habilidades, valores y objetivos personales; para sacar un mejor provecho de �stas en pro de la obtenci�n de las metas en la organizaci�n. Es necesaria una adecuada gesti�n de las personas.

Me gustar�a hablar un poco m�s sobre las personas como instrumentos generadores informaci�n, constructoras de recursos documentales, multimedia y otras actividades de las cuales estas son el elemento o fuente de informaci�n.

Si e una organizaci�n el principal elemento son las personas, el conocimiento que generan y consumen debe ser un su producto muy importante para la organizaci�n. M�s que un conocimiento particular o de unos cuantas personas, sino de todo el conocimiento que deben generar todas las personas de la organizaci�n. 

Ya que este conocimiento es la llave para tener nuevas ideas, abarcar otros mercados, una base de conocimiento, una manera de compartir ideas, as� como un lenguaje t�cnico entendi� por todas las personas.

Las actividades cotidianas que llevan acabo las personas de una organizaci�n o miembros de la misma son actividades cotidianas y estructuradas, entonces, estas personas a partir de la experiencia y proceso continuo de realizaci�n de las actividades, crean experiencia que es una forma de conocimiento. Para rescatar este conocimiento personal y as� construir una base de conocimiento las personas, almacenan la informaci�n sobre su conocimiento en documentos, v�deos, im�genes, en cualquier objeto que le permita plasmar sus ideas. Entonces tanto las personas y los archivos son fuentes de informaci�n o conocimiento para otras personas o inclusive para ellos mismos. Entonces para tener una eficiente y eficaz manera de trabajar con los recursos de la organizaci�n, es necesario administrar adecuadamente estos recursos "personas, documentos, archivos digitales".

Este problema general de administraci�n de los recursos personas, documentos, multimedia que son conocidos como recursos informativos. Es el tema principal de nuestras tesis. Para identificar y solucionar este problema, nosotros partimos de ciertos hechos b�sicos en una organizaci�n. Par empezar la organizaci�n con la que trabajamos son los profesores del departamento de ingenier�a el�ctrica que pertenecen al �rea de redes y telecomunicaciones y que a su vez esta contenido en la Universidad Aut�noma Metropolitana unidad iztapalapa.



\section{Memoria Corporativa}
Alg�n texto...

\section{Casos de uso}
M�s texto...

\chapter{Tecnolog�as Sem�nticas}
\label{sec:ets}

\section{Definiciones y descripciones}
Alg�n texto...

\section{Marco de Descripci�n de Recursos (RDF)}
M�s texto...

\section{Lenguaje de consulta sobre grafos RDF (SPARQL)}
M�s texto...

\section{Reglas de inferencia (RDF(S)/OWL) y razonadores}
M�s texto...

\section{Ventajas de las tecnolog�as Sem�nticas}
M�s texto...

\chapter{Estado del arte}
\label{cap:soa}
La \textbf{\textit{integraci�n de los recursos de informaci�n en una memoria corporativa}} ha sido poco explotada por las organizaciones o �reas de investigaci�n. Existen algunos trabajos sobre la \textit{integraci�n de informaci�n} que incorporan a las tecnolog�as sem�nticas para representar y enriquecer el conocimiento de un dominio dado, as� como, la b�squeda de informaci�n a partir de este conocimiento. En este estado del arte, se consideran los trabajos que cumplen con alguno de nuestros criterios de investigaci�n. Estos criterios son descritos en la Tabla \ref{tab:crit}

%%%Los trabajos  considerados para este \textit{estado del arte} son aquellos que utilizan alguno de los criterios de la Tabla \ref{tab:crit}.
\begin{table}[!htb]
\renewcommand{\arraystretch}{1.3}
\centering
\begin{tabular}{>{\centering\arraybackslash}m{1.5in} >{\centering\arraybackslash}m{3.7in} >{\centering\arraybackslash}m{0.9in}}
\hline
\textbf{Criterio} & \textbf{Descripci�n} & \textbf{Definici�n formal}\\
\hline
\hline
Integraci�n de informaci�n en los recursos & �sta consiste en el proceso de b�squeda y recuperaci�n de la informaci�n sobre los recursos de informaci�n. & Secci�n \ref{sec:intdk} \\
\hline
Memoria corporativa & �sta es la representaci�n consistente y formal del conocimiento en una organizaci�n. & Secci�n \ref{sec:mecor} \\
\hline
Modelo sem�ntico & �ste es la representaci�n del conocimiento a partir de las tecnolog�as sem�nticas. & Secciones \ref{sec:rdf} y \ref{sec:reginf} \\
\hline
Inferencia en el modelo & �sta consiste en deducir informaci�n a partir de los axiomas en el modelo. & Secci�n \ref{sec:reginf} \\
\hline
Interfaz visual para la integraci�n & �sta es un aplicaci�n con una enfoque visual para que las personas pregunten o naveguen a trav�s de la informaci�n en el modelo sem�ntico. & Secci�n \ref{cap:piu} \\
\hline 
\end{tabular}
\caption{Criterios considerados para el \textit{estado del arte} de la integraci�n sem�ntica de recursos.}
\label{tab:crit}
\end{table}

La Secci�n \ref{sec:eoaisr} describe los trabajos que se estudiaron para la integraci�n de los recursos y al final de �sta secci�n se presenta una tabla comparativa de estos trabajos, as� como los valores asociados a los criterios de investigaci�n.

En este \textit{estado del arte}, se contempla un estudio de las aplicaciones para realizar la \textit{integraci�n sem�ntica de los recursos en una memoria corporativa}. Las aplicaciones estudiadas, se agrupan de acuerdo a las siguientes funcionalidades.

\begin{enumerate}
\item Escribir los declaraciones en forma de triple y guardarlos en alguna sintaxis est�ndar.
\item Escribir los axiomas mediante los vocabularios est�ndar (OWL y RDF(S)).
\item Gesti�n del grafo RDF, es decir, carga del modelo, consulta de informaci�n e inferencia.
\end{enumerate}

La Secci�n \ref{sec:eoats} muestra y describe las aplicaciones estudiadas con base en su funcionalidad; al final de cada agrupaci�n, se da a conocer: \textit{cu�l herramienta se eligi� para facilitar y efectuar la funcionalidad dada}.

\section{Integraci�n sem�ntica de recursos de informaci�n}
\label{sec:eoaisr}
El principal objetivo de la \textit{integraci�n de los recursos} es buscar y recuperar informaci�n que est� en los recursos, para responder las necesidades informativas de las personas. Una \textit{integraci�n sem�ntica} de recursos emplea las tecnolog�as sem�nticas con la finalidad de recuperar informaci�n significativa en los recursos a partir de las caracter�sticas y relaciones de estos. El uso de una \textit{memoria corporativa} para la integraci�n sem�ntica, se traduce en informaci�n y conocimiento de los recursos bajo un dominio particular.

Algunos trabajos exploran o emplean el enfoque de las \textit{tecnolog�as sem�nticas} para fines de integraci�n del conocimiento, representaci�n de una memoria corporativa o b�squeda de informaci�n. A continuaci�n, se describe el estado actual del conocimiento referente a estos trabajos.

\begin{itemize}
\item La \textbf{\textit{arquitectura del modelo dual}} \cite{Archetype} es una propuesta para la representaci�n consistente y comprensible de la informaci�n cl�nica de cualquier persona. La finalidad de esta arquitectura es facilitar el acceso de historial cl�nico de los pacientes a los profesionales de la salud. La informaci�n de estos historiales esta distribuida en varios sistemas independientes y heterog�neos. Esta arquitectura se basa en un modelo que por un lado representa la informaci�n y por el otro el conocimiento. En la representaci�n de la informaci�n se describen las estructuras de datos comunes. Mientras, en la representaci�n del conocimiento se emplean arquetipos para representar el conocimiento formal de conceptos cl�nicos. Este trabajo presenta una herramienta para desarrollar los arquetipos de datos cl�nicos. Esta herramienta es llamada LinkEHR-Ed. La finalidad de est� es que los profesionales de la salud y expertos en tecnolog�as de la informaci�n sean los principales constructores del conocimiento.
\end{itemize}

Nosotros \textbf{\textit{elegimos este trabajo}} por las siguientes razones:
1) \textit{la arquitectura representa el conocimiento de los historiales cl�nicos (dominio de la salud)}, 2) \textit{la arquitectura solventa la informaci�n distribuida y formatos propios de un sistema}, 3) \textit{la arquitectura modela el conocimiento a manera de un componente terminol�gico y un componente asertivo}, 4) \textit{los arquetipos agregan una capa sem�ntica y proporcionar el conocimiento formal para el sector salud} y 5) \textit{una herramienta para construir arquetipos asociados al sector salud}.

\begin{itemize}
\item El \textbf{\textit{marco de integraci�n sem�ntica}} \cite{Accommodation} es una propuesta para solucionar de manera eficaz y flexible la integraci�n de la informaci�n en el dominio del alojamiento en-l�nea. La finalidad de esta integraci�n es facilitar la reuni�n y compartici�n de informaci�n referente al alojamiento en-l�nea, donde, esta informaci�n est� en constante cambio. Este marco de integraci�n tiene un conjunto de caracter�sticas b�sicas: 1) emplear una ontolog�a para facilitar el acceso a la informaci�n integrada y solucionar la heterogeneidad en la estructura de la informaci�n, 2) emplear un proceso que resolver la naturaleza din�mica de las fuentes de informaci�n, 3) permitir a los propietarios de la informaci�n participar en el proceso de integraci�n, 4) emplear una serie de esquemas para el intercambio de informaci�n.
\end{itemize}

Este trabajo \textbf{\textit{se eligi�}} por estas razones: 1) \textit{proporciona un marco de trabajo para la integraci�n de la informaci�n}, 2) \textit{emplea una ontolog�a para modelar la informaci�n del dominio de alojamiento en-l�nea} y 3) \textit{emplear la ontolog�a como instrumento para resolver la heterogeneidad en la estructura de la informaci�n y el acceso a la informaci�n integrada}.

\begin{itemize}
\item Jun Zhai et al. \cite{JunZhai} proponen una \textit{\textbf{integraci�n sem�ntica} con base en ontolog�as para \textbf{sistemas de informaci�n de energ�a el�ctrica}}, donde, estos sistemas son heterog�neos con funciones y organizaciones descentralizadas. Esta integraci�n, por un lado, \textit{emplea al \textbf{lenguaje de marcado extensible} (XML\label{sym:xml} para el intercambio de informaci�n entre estos sistemas}. Por otro lado, esta integraci�n \textit{utiliza una \textbf{ontolog�a} para describir formalmente la informaci�n a nivel conceptual en el dominio de la electricidad}. Este trabajo propone una arquitectura de tres capas para esta integraci�n sem�ntica: 1) capa fuentes de datos heterog�neos distribuidos, 2) capa de integraci�n de la informaci�n y 3) capa de sistemas de aplicaci�n.
\end{itemize}

Este trabajo de Jun Zhai \textbf{\textit{es interesante}}, en primera, \textit{porque emplea una ontolog�a como especificaci�n para integrar la informaci�n a nivel sem�ntico}, y en segunda, \textit{el uso de XML para el intercambio de informaci�n y solucionar la heterogeneidad de datos en los sistemas de energ�a el�ctrica}.

\begin{itemize}
\item Xin y Guangleng \cite{WangXin} emplean un \textit{enfoque basado en las \textbf{ontolog�as} para capturar la \textbf{informaci�n del design rationale}}. Este design rationale es un conocimiento para explicar qu� y c�mo se dise�a un producto, as� como para apoyar la reutilizaci�n, comunicaci�n y verificaci�n de dise�os en empresas manufactureras. En este trabajo, se emplea una memoria corporativa para las actividades de gesti�n del conocimiento, en particular, las actividades de captura y disponibilidad en el design rationale. Las ontolog�as permiten el acceso uniforma a las fuentes de informaci�n, y en este trabajo, �stas modelan el design rational para el background del dise�o de autos de carga.
\end{itemize}

Este trabajo \textbf{\textit{es importante}} por estas razones: 1) \textit{utilizar una memoria corporativa para el conocimiento del design rational}, 2) \textit{emplear ontolog�as para capturar la informaci�n del design rational} y 3) \textit{acceder de manera uniforma a los recursos de informaci�n mediante las ontolog�as}.

\begin{itemize}
\item \textbf{\textit{PCOGEME}} \cite{Behja} es un \textbf{\textit{entorno de colaboraci�n}} para la \textit{creaci�n, gesti�n, difusi�n, mantenimiento de memorias corporativas}. En este trabajo, las memorias corporativas son mecanismos para la gesti�n del conocimiento y documentos. PCOGEME propone un modelo de interacci�n basado en las ontolog�as para la representaci�n y gesti�n de estas memorias. El funcionamiento de PCOGEME se basa en la lluvia de ideas y un mecanismo de toma de decisiones consensuadas, para la construcci�n de memorias corporativas mediante el uso de ontolog�as.
\end{itemize}

Nosotros \textbf{\textit{elegimos}} este trabajo, porque emplea los siguientes elementos: 1) \textit{memoria corporativa como instrumentos de gesti�n del conocimiento}, 2) \textit{representaci�n del conocimiento mediante el uso de ontolog�as} y 3) \textit{el uso de un entorno de gesti�n del conocimiento}.

La Tabla \ref{tab:soasi} es un sumario de los valores asociados a nuestros criterios investigaci�n para cada trabajo estudiado. Las cabeceras en esta Tabla est�n en forma abreviada y estos son sus significados: \textbf{IIR} = integraci�n de informaci�n en los recursos, \textbf{MC} = memoria corporativa, \textbf{MS} = modelo sem�ntico, \textbf{IeM} = inferencia en el modelo y \textbf{IVpI} = interfaz visual para la integraci�n.

%Estos criterios de investigaci�n est�n en la Tabla \ref{tab:crit}.

\begin{table}[!htb]
\renewcommand{\arraystretch}{1.3}
  \centering
  \smallskip 
  \begin{threeparttable}
      \begin{tabular}{>{\centering\arraybackslash}m{1.5in} >{\centering\arraybackslash}m{0.6in} >{\centering\arraybackslash}m{0.8in} >{\centering\arraybackslash}m{0.95in} >{\centering\arraybackslash}m{0.8in} >{\centering\arraybackslash}m{0.8in}}
		\hline
		\textbf{Trabajo} & \textbf{IIR} & \textbf{MC} & \textbf{MS} & \textbf{IeM} & \textbf{IVpI}\\
		\hline
		\hline
		Arquitectura del modelo dual & Si & No & Modelo de referencia y arquetipos & No & No \\
		\hline
		Marco de integraci�n sem�ntica & Si & No & Informaci�n XML y ontolog�a global & No & Prototipo \\
		\hline
		Arquitectura para la integraci�n en SIEE\tnote{a} & Si & No & tripletas RDF y axiomas RDF(S) & Consistencia & No \\
		\hline
		Metodolog�a para ontolog�as en el design rationale & Si & Design rationale & Ontolog�a & No & No \\
		\hline
		Construcci�n de MC en forma colaborativa & No & SSII\tnote{b} & Ontolog�a & No & No \\
		\hline 
     \end{tabular}
     \begin{tablenotes}
       \item[a] \scriptsize Sistemas de Informaci�n de Energ�a El�ctrica. 
       \item[b] \scriptsize Sociedad de Servicios en Ingenier�a Inform�tica.
     \end{tablenotes}
  \end{threeparttable}
  \caption{Comparativa entre los trabajos estudiados y nuestros criterios para la integraci�n sem�ntica de recursos.}
  \label{tab:soasi}
\end{table}

En esta Tabla \ref{tab:soasi}, el valor `\textbf{Si}' indica que el trabajo cumple con ese criterio, mientras, el valor `\textbf{No}' indica lo contrario.

%%%[04:14:23 p.m.] Carolina Medina: 1.- LA descripci�n de cada trabajo/enfoque (resumen). Por qu� lo seleccion� o incluye [ref]
%%%[04:14:54 p.m.] Carolina Medina: 2.- culminar la secci�n con una tabla comparativa en la cual se pongan
%%%[04:15:35 p.m.] Carolina Medina: Sistemas comparados/ caracter�sticas que tienen o deben de cumplir

\section{Herramientas para la integraci�n sem�ntica de recursos}
\label{sec:eoats}
Un \textbf{\textit{descriptor sem�ntico de recursos}} \cite{Uren:2006} es una herramienta para crear y almacenar tripletas RDF a partir de la \textit{informaci�n expl�cita en los recursos}. Las tripletas que son generadas por esta herramienta, est�n escritas en una de las siguientes sintaxis: \textit{RDF/XML, Turtle, N-triple y N3}. El principal objetivo de un \textit{descriptor} es construir instancias y relacionar �stas con determinados valores u otras instancias (\textit{concepto de triple}). Algunas de estas herramientas requieren un TBox para saber cu�les clases y propiedades, pueden emplearse en los triples. Un descriptor proporciona una \textit{interfaz gr�fica de usuario} (GUI\label{sym:gui}) para simplificar a los usuarios la creaci�n y modificaci�n de las declaraciones. Algunos descriptores sugieren informaci�n para las declaraciones a partir de un proceso de aprendizaje en un corpus documental o de im�genes.

En la siguiente lista, se presentan los descriptores sem�nticos que nosotros estudiamos.
% Estas interfaces tienen un editor de documentos para que un usuario: \textit{1) visualice el contenido de estos, 2) seleccione los datos (literales: cadenas, enteros, flotantes) y 3) asocie los datos a una propiedad}.

\begin{itemize}
\item \textbf{\textit{OntoMat Annotizer}} \cite{OntoBasDoc} es una herramienta para hacer anotaciones sem�nticas de p�ginas web, documentos basados en texto plano y lenguajes de marcado\footnote{M. Siroker, ``OntoMat Annotizer,''  Disponible en: \url{http://projects.semwebcentral.org/projects/ontomat/}}. El objetivo de esta herramienta es que el usuario cree de manera amigable instancias y declaraciones de �stas, mediante la funcionalidad de arrastrar y soltar (drag-and-drop).
\item \textbf{\textit{MnM}} \cite{Uren:2006} es una herramienta que proporciona apoyo automatizado y semiautomatizado para describir p�ginas Web con contenido sem�ntico\footnote{The Open University, ``MnM,''  Disponible en: \url{http://projects.kmi.open.ac.uk/akt/MnM/}}. MnM tiene GUI que integra un editor de ontolog�a, navegador Web, un editor de instancias y de propiedades. El objetivo de esta herramienta es la descripci�n de documentos a partir de declaraciones derivadas de ontolog�as preexistentes.
\item \textbf{\textit{GATE}} \cite{Cunningham2011a} es un entorno de desarrollo integrado (IDE\label{sym:ide}) para el desarrollo de componentes en el procesamiento del lenguaje humano y el procesamiento de texto\footnote{The University of Sheffield, ``GATE,''  Disponible en: \url{http://gate.ac.uk/}}. Las tareas en el procesamiento de texto son: \textit{miner�a web, extracci�n de informaci�n  y descripciones sem�nticas}.
\item \textbf{\textit{Aktive Media}} \cite{Uren:2006} es una GUI para la descripci�n autom�tica de una colecci�n de im�genes o documentos (batch annotation) para un contexto espec�fico. ``\textit{El objetivo de Aktive es automatizar el proceso de descripci�n, mediante la sugerencia interactiva de la informaci�n al usuario, mientras �ste est� describiendo}\footnote{A. Chakravarthy, V. Lanfranchi , F. Ciravegna, ``AKTive Media,''  Disponible en: \url{http://www.aktors.org/technologies/aktivemedia/index.html}}''. Estas sugerencias se hacen con base en axiomas y descripciones previas.
\end{itemize}

La finalidad de un descriptor es facilitar la generaci�n de descripciones en forma de triple. Sin embargo, hay varias razones, por las cu�les, no se elige una de estas herramientas para alcanzar este fin. Las razones son: 1) \textit{todas estas aplicaciones permiten hacer declaraciones de documentos e im�genes, por tal raz�n, no proporcionan una soluci�n a la heterogeneidad en formato}, 2) \textit{OntoMat Annotizer y MnM no interpretan los axiomas que est�n escritos con los vocabularios OWL y RDF(S)}, 3) \textit{Aktive Media y GATE cambian las URIs en las tripletas por sus propios URIs}, 4) \textit{OntoMat Annotizer y MnM no tienen versi�n estable} y 5) \textit{Aktive Media, GATE y MnM no tienen documentaci�n disponible para solucionar problemas de configuraci�n}.

Un \textbf{\textit{script}} es un c�digo que se escribe en un lenguaje de programaci�n y se utiliza para la escritura y almacenamiento de descripciones en forma de tripletas. El prop�sito es facilitar y agilizar el proceso de generaci�n de tripletas en alguna sintaxis est�ndar. Aunque un script no posee una interfaz gr�fica para seleccionar la informaci�n de los recursos. Esto se puede solucionar mediante el uso de formularios web que capturen la informaci�n sobre los recursos. Posteriormente, la informaci�n es guardada en alg�n documento de texto plano, para que un script transforme esta informaci�n en triples RDF. Por tal raz�n, \textbf{\textit{un script es la opci�n electa}} para representar el conocimiento expl�cito en forma de tripletas.\\

Un \textbf{\textit{editor de ontolog�a}} \cite{Tode} es una herramienta que proporciona una serie de interfaces amigables para la construcci�n y mantenimiento de ontolog�as. Estos editores proporcionan las siguientes funcionalidades b�sicas a los usuarios: 1)\textit{definir las clases, propiedades, instancias y axiomas}, 2) \textit{cargar, almacenar, importar y exportar ontolog�as que son escritas con lenguajes est�ndar (RDF(S)y OWL)} y 3) \textit{visualizar las clases, propiedades e individuos}.

\begin{itemize}
\item \textbf{\textit{Prot�g�}} \cite{protege} es una plataforma con herramientas para la creaci�n, visualizaci�n y manipulaci�n de ontolog�as en diversos formatos de representaci�n\footnote{Stanford Center for Biomedical Informatics Research, ``Prot\'{e}g\'{e},''  Disponible en: \url{http://protege.stanford.edu/}}. Esta plataforma proporciona al usuario una interfaz amigable para la definici�n de clase, propiedades y axiomas, as� como la introducci�n de datos. La arquitectura de esta herramienta se puede extender a trav�s de plug-ins y APIs. Esta herramienta tiene licencia open-source Mozilla Public License\footnote{Mozilla, ``Mozilla Public License,''  Disponible en: \url{http://www.mozilla.org/MPL/}}.
\item \textbf{\textit{pOWL}} \cite{pow} es una herramienta para la visualizaci�n y edici�n de ontolog�as v�a web\footnote{S\"{o}ren Auer, ``pOWL,''  Disponible en: \url{http://aksw.org/Projects/Powl.html}}. Esta herramienta soporta la carga y edici�n de ontolog�as con vocabularios RDF(S) y OWL, generaci�n de consultas y almacenamiento del modelo en una base de datos relacional.
\item \textbf{\textit{TopBraid Composer}} \cite{OntologyManagement} es un IDE para "\textit{desarrollar, gestionar y probar configuraciones de los modelos de conocimiento e instancias de las bases de conocimiento}"\footnote{TopQuadrant, Inc., ``TopBraid Composer,''  Disponible en: \url{http://www.topquadrant.com/products/TB_Composer.html}}. Esta herramienta proporciona un conjunto de editores para visualizar grafos RDF y diagramas de clase. Existen tres versiones de esta herramienta: maestro, est�ndar y gratuita. La versi�n gratuita permite crear y editar archivos OWL/XML, as� como consultar con el lenguaje SPARQL.
\item \textbf{\textit{SWOOP}} \cite{swoop} es un editor para crear y editar ontolog�as, comprobar inconsistencias, navegar por las ontolog�as, compartir y reutilizar los datos existentes\footnote{University of Maryland, ``SWOOP,''  Disponible en: \url{https://code.google.com/p/swoop/downloads/list}}. Este editor ofrece un entorno con aspecto de navegador web para facilitar la navegaci�n y edici�n de ontolog�as OWL. Este editor provee una interfaz amigable y eficaz para los usuarios web promedios.
\end{itemize}

%Prot�g� se puede extender sus funcionalidades a trav�s de plug-ins y APIs. Estas funcionalidades son: 1) visualizar los axiomas, clases y propiedades en forma de grafo, 2) exportar ontolog�as a una variedad de formatos, 3) importar y hacer uso de alg�n motor de inferencia o razonador, 4) cambiar el URI del vocabulario, 5) exportar las ontolog�a a otras sintaxis (RDF/XML, Turtle, Manchester), por mencionar algunas.

\textbf{\textit{Prot�g�}} es el editor electo para representar los axiomas en una ontolog�a. Porque este editor proporciona estos beneficios: 1) \textit{una interfaz amigable e intuitiva para el usuario}, 2) \textit{amplia documentaci�n y tutoriales, as� como una comunidad de desarrolladores}, 3)\textit{facilidad de extender la funcionalidad de esta herramienta, gracias a su arquitectura de plug-ins}, 4) \textit{variedad de sintaxis para las ontolog�as, como: Turtle, Manchester, OWL/XML o XML/RDF}, 5) \textit{visualizaci�n del grafo (axiomas, clases y propiedades)}, 6) \textit{incorporar razonadores, como: Pellet\footnote{Clark \& Parsia, LLC, ``Pellet,''  Disponible en: \url{http://clarkparsia.com/pellet/}}, Fact++\footnote{Clark \& Parsia, LLC, ``Pellet,''  Disponible en: \url{http://clarkparsia.com/pellet/}} y HermiT\footnote{Oxford University, ``HermiT,''  Disponible en: \url{http://hermit-reasoner.com/}}} y 
7) \textit{incorporar un motor de consulta SPARQL}.\\

Un \textbf{\textit{triplestore}} \cite{dbpedia_2012} es un programa para \textit{el almacenamiento e indexaci�n de tripletas RDF}, con el fin de permitir la consulta eficiente de informaci�n sobre estas tripletas. Estos triplestores emplean el est�ndar SPARQL como lenguaje de consulta para consultar el grafo RDF. Algunos triplestores soportan la capacidad de inferir en el grafo RDF a partir de axiomas, mediante la incorporaci�n o importaci�n de un razonador para ello. Los triplestores se idealizan como \textit{sistema gestor de bases de datos para modelos basados en triplestas RDF}.

En el siguiente listado, se presentan y describen los triplestores que estudiamos.

\begin{itemize}
\item \textbf{\textit{Apache Jena}} \cite{McBride} es un \textit{marco de trabajo} que  proporciona un conjunto de interfaces de programaci�n de aplicaciones (API\label{sym:api}) para Java. Estas APIs ofrecen las siguientes funcionalidades: \textit{lectura, procesamiento y escritura de triples RDF, as� como axiomas RDF(S) y OWL, un motor de inferencia y un motor de consulta SPARQL}. La finalidad de Jena es desarrollar aplicaciones que usan las tecnolog�as sem�nticas para la representaci�n del conocimiento\footnote{The Apache Software Foundation, ``Apache Jena,''  Disponible en: \url{http://jena.apache.org/}}.%%%Estas APIs tienen las siguientes funcionalidades: 1) \textit{lectura, procesamiento y escritura de triples RDF en alguna sintaxis est�ndar (RDF/XML, N-triples y turtle)}. 2) soporte de axiomas en los lenguajes OWL y RDF(S), 3) \textit{un motor de inferencia que soporta axiomas en OWL y RDF(S)}, 4) \textit{almacenamiento eficiente de los triples en el disco duro} y 5) \textit{un motor de consultas con soporte para el lenguaje SPARQL}.
\item \textbf{\textit{Stardog}} \cite{stardog} es una base de datos para modelos sem�nticos. El prop�sito de esta herramienta es la ejecuci�n de consultas sobre los datos RDF que est�n bajo su gesti�n directa\footnote{Clark \& Parsia, LLC, ``Stardog,''  Disponible en: \url{http://stardog.com/}}. Esta herramienta emplea los protocolos \textit{HTTP  y SNARL} para \textit{acceder y controlar de manera remota el modelo de datos RDF, inferencia a partir de axiomas en lenguaje OWL} y \textit{consultas SPARQL}. %%%Stardog soporta el lenguaje est�ndar SPARQL y emplea los protocolos \textit{HTTP  y SNARL} para: acceder y controlar de manera remota el modelo de datos RDF, la inferencia y el an�lisis de datos en el lenguaje OWL. Stardog apoyado en el motor de b�squeda de texto Lucene, proporciona la capacidad de b�squedas sem�nticas que consiste en la indexaci�n de literales RDF.
\item \textbf{\textit{4store}} \cite{Fstore} \textit{es un sistema para el almacenamiento RDF que incorpora un motor de consultas SPARQL}\footnote{Garlik, ``4store,''  Disponible en: \url{http://4store.org/}}. Las principales fortalezas de esta herramienta son el rendimiento, seguridad, escalabilidad y estabilidad.
\item \textbf{\textit{Sesame}} \cite{Sesame} \textit{es un \textit{marco de trabajo} est�ndar de facto para el an�lisis, almacenamiento, inferencia y consulta de datos RDF}\footnote{Aduna, ``Sesame,''  Disponible en: \url{http://www.openrdf.org/index.jsp}}. Este marco proporciona una API que puede emplearse sobre los distintos \textit{sistemas de almacenamiento RDF} para consultar y acceder a esta informaci�n de manera remota. %Finalmente, Sesame soporta: los principales sintaxis para los triples RDF y el lenguaje de consulta SPARQL.
\end{itemize}

Cualquiera de estos triplestore es una opci�n viable para efectuar tareas de almacenamiento y b�squeda de informaci�n en grafos RDF. Aunque, el m�s interesante desde nuestra perspectiva es Apache Jena. Las razones del \textit{porqu� emplear esta herramienta}, son: 1) \textit{amplia documentaci�n y tutoriales para el desarrollo de modelos sem�nticos}, 2) \textit{integraci�n de Jena en IDEs para el lenguaje Java, como Eclipse}\footnote{The Eclipse Foundation, ``Eclipse IDE,''  Disponible en: \url{http://www.eclipse.org/}}, 3) \textit{proyecto open-source bajo la licencia Apache\footnote{The Eclipse Foundation, ``Licencia Apache v. 2.0 ,''  Disponible en: \url{http://www.apache.org/licenses/LICENSE-2.0.html}} versi�n 2}, 4) \textit{un conjunto de librer�as para crear, cargar, almacenar y consultar declaraciones, as� como axiomas en OWL y RDF(S)}, 5) \textit{un motor de inferencia para realizar razonamiento en ontolog�as que emplean axiomas OWL y RDF(S)} y 6) \textit{una amplia comunidad de desarrolladores}.

%%%Fuentes: C:\Users\Gatito\Dropbox\Gesti�n Sem�ntica\Actividades 12P\Semana 8 y 9\Anotaciones Sem�nticas.docx

%%%C:\Users\ARTE\Dropbox\Gesti�n Sem�ntica\Actividades 13I\feedback\CMED_8_FEB_Integracion Sem�ntica de los recursos.docx

% TABLA ONTOLOGIAS en: C:\Users\Gatito\Dropbox\Gesti�n Sem�ntica\Actividades 12P\Semana 10\doc auxiliar.docx

\chapter{Integraci�n sem�ntica de recursos de informaci�n en una memoria corporativa}
\label{cap:sir}
%revizar este documento C:\Users\Gatito\Dropbox\Gesti�n Sem�ntica\Tarea12O\Integracion Sem�ntica de los recursos.doc
La \textit{integraci�n de los recursos} es el proceso de b�squeda y recuperaci�n significativa de informaci�n existente en los recursos, para responder una consulta dada por un usuario. Si esta integraci�n se hace mediante el uso de herramientas, est�ndares, metodolog�as y aplicaciones pertenecientes a las \textit{tecnolog�as sem�nticas}, entonces, se dice que �sta es una \textbf{\textit{integraci�n sem�ntica de los recursos (ISR\label{sym:isr})}}.

La \textit{integraci�n sem�ntica de recursos} puede implementarse en una \textit{memoria corporativa (MC)}. Porque una memoria tiene un conjunto diverso de recursos de informaci�n, los cuales representan el conocimiento en una organizaci�n (dominio particular). Estas son las principales razones de esta \textit{integraci�n en una memoria corporativa}: 1) \textit{solucionar la heterogeneidad de los recursos y la ambig�edad de la informaci�n en una memoria corporativa}, 2) \textit{adaptar el conocimiento cambiante o explosivo en los recursos}, 3) \textit{extender y mantener un modelo (representaci�n) del conocimiento}, 4) \textit{permitir consultas espec�ficas a partir de las caracter�sticas y relaciones de los recursos}, 5) \textit{recuperar informaci�n significativa de los recursos para que respondan las preguntas de las personas adscritas en la organizaci�n} y 6) \textit{emplear herramientas, aplicaciones, vocabularios y formatos est�ndar}.

El desarrollo de la \textit{integraci�n sem�ntica de recurso} se hace con base en una \textbf{\textit{secuencia ordenada de m�todos (metodolog�a)}}. Esta tesis describe una \textbf{\textit{propuesta de metodolog�a}} para la \textit{integraci�n sem�ntica de recursos en una memoria corporativa}, la cual est� guiada por dos \textit{casos de uso}.

La \textbf{\textit{finalidad}} de esta propuesta es \textit{facilitar y guiar a los desarrolladores en estas dos tareas: 1) construir un modelo sem�ntico (otolog�as) y 2) consultar informaci�n en este modelo}. Mientras, los principales objetivos de �sta son:

\begin{itemize}
\item realizar la ISR en cualquier memoria corporativa, por ejemplo \textit{Biom�dica, Qu�mica, Biolog�a, Computaci�n, Econom�a, Zoolog�a, por mencionar algunas}
\item emplear distintos \textit{casos de uso} para la ISR y no limitar el n�mero de �stos.
\item representar una MC en un formato est�ndar con un vocabulario consensuado y asociado al contexto de la MC.
\item utilizar vocabularios est�ndares para los axiomas, as� como el uso del lenguaje SPARQL para las consultas.
\end{itemize}

Esta metodolog�a est� organizada en tres etapas principales:

\begin{enumerate}
\item \textbf{\textit{Representaci�n del conocimiento en los recursos}} consiste en identificar los recursos de la memoria corporativa y representar los metadatos (conocimiento expl�cito) de estos recursos mediante el marco RDF.
\item \textbf{\textit{Enriquecimiento del conocimiento en el modelo}} consiste en introducir axiomas en OWL y RDF(S) para extender, completar y adaptar el conocimiento expl�cito de los recursos.
\item \textbf{\textit{B�squeda y recuperaci�n de la informaci�n en el modelo}} consisten en identificar las principales consultas de los usuarios en el dominio y ejecutar �stas mediante el uso de un \textit{motor de b�squeda SPARQL} junto con un \textit{razonador}, para recuperar informaci�n de los recursos.
\end{enumerate}

La primera y segunda etapa consisten en la construcci�n del modelo sem�ntico, mientras la tercera etapa consiste en la consulta de informaci�n en este modelo.

En esta metodolog�a, uno de los elementos clave es el \textbf{\textit{caso de uso}}. Porque este elemento encamina en el desarrollo de la integraci�n sem�ntica. En concreto, los \textit{casos de uso} permiten encontrar: 1) \textit{qu� caracter�sticas y relaciones son significativas}, 2) \textit{qu� reglas de inferencia son necesarias} y 3) \textit{cu�les consultas son importantes}.

La Figura \ref{fig:arq} muestra la arquitectura para la \textit{integraci�n sem�ntica de recursos}. Esta arquitectura es gen�rica para ser desarrollada e implementada en cualquier memoria corporativa. Los componentes de esta arquitectura se construyen, utilizan e implementan mediante el uso de nuestra propuesta de metodolog�a.

\begin{figure}[!htb]
\centering
%\includegraphics[width=3.2in]{Arquitectura}
\includegraphics[width=0.6\textwidth]{Arquitectura}
\caption{Arquitectura general para la Integraci�n Sem�ntica de Recurso en una Memoria Corporativa.}
\label{fig:arq}
\end{figure}

Esta arquitectura se dise�o con base en el modelo de tres capas: \textbf{\textit{usuario, negocio}} y \textbf{\textit{datos}}.

\begin{itemize}
\item En la capa de usuario: se tiene un conjunto de p�ginas Web din�micas y est�ticas que proporcionan la interfaz visual. Esta interfaz proporciona una manera f�cil y sencilla de estructurar las preguntas de los usuarios, as� como la visualizaci�n de los resultados vinculados a estas preguntas. Las p�ginas est�ticas proporcionan los formularios para que los usuarios \textit{estructuren las preguntas} y \textit{capturen la informaci�n} a buscar en la MC. Mientras, las p�ginas din�micas proporcionan la informaci�n que responde las preguntas en un formato visual agradable al usuario.
\item En la capa de negocios: una aplicaci�n transforma la informaci�n recopilada de las p�ginas est�ticas en patrones tripletas y construir una consulta SPARQL. Posteriormente, esta aplicaci�n invoca al triplestore para efectuar estas actividades: 1) solicitar y cargar la ontolog�a, 2) hacer inferencia en una ontolog�a mediante el uso de un razonador y 3) buscar y recuperar la informaci�n en el modelo inferido mediante el uso de motor de b�squeda SPARQL y la consulta SPARQL.
\item En la capa de datos (conocimiento): la ontolog�a modela el conocimiento de los recursos de una memoria corporativa en un formato est�ndar y con un vocabulario consensual. El componente asertivo contiene las descripciones de las caracter�sticas y relaciones expl�citas de los recursos. Mientras, el componente terminol�gico contiene los axiomas que definen y restringen la manera en que se relacionan los recursos.
\end{itemize}

Esta propuesta de metodolog�a se pone en pr�ctica para la \textit{memoria corporativa} del grupo de investigaci�n perteneciente al �rea de Redes y Telecomunicaciones del departamento de Ingenier�a El�ctrica de la Universidad Aut�noma Metropolitana Unidad Iztapalapa. Los principales usuarios en la integraci�n son los \textbf{\textit{profesores-investigadores}} del n�cleo del �rea de Redes y Telecomunicaciones, as� como los \textbf{\textit{estudiantes}} que realizan alg�n proyecto o servicios social y est�n a cargo de profesor del n�cleo.

Los \textit{casos de uso} b�sicos en esta metodolog�a son \textit{la cartograf�a de competencias} y \textit{la b�squeda de recursos digitales}.

\begin{enumerate}
\item La cartograf�a de competencias es la b�squeda y recuperaci�n de las personas a partir de las caracter�sticas personales y profesionales (competencias, capacidades, conocimientos en los temas del dominio).
\item La b�squeda de recursos digitales es la b�squeda y recuperaci�n de los documentos y archivos multimedia a partir del contenido de �stos (temas del dominio, autor, a�o).
\end{enumerate}

Estos dos \textit{casos de uso} son independientes entre ellos, por tal raz�n, cada uno tiene una respectiva ontolog�a. La ontolog�a de la cartograf�a de competencias modela el conocimiento expl�cito e impl�cito de los recursos persona, con base en las caracter�sticas personales y profesionales de �stos. Mientras, la ontolog�a de los recursos digitales modela el conocimiento expl�cito e impl�cito del contenido y acerca de �stos.

En ambos casos de uso, un aspecto importante es que tanto personas como recursos digitales se vinculan con los temas del �rea de Redes y Telecomunicaciones(RyT). Espec�ficamente, los conocimientos de las personas son relaciones entre personas y temas de RyT. Mientras, los t�picos pertenecientes en los recursos digitales son las relaciones entre recursos digitales y temas de RyT. Por tal raz�n, se construye una tercer ontolog�a para modelar el vocabulario consensual del �rea de Redes y Telecomunicaciones.

La Figura \ref{fig:ontocu} muestra el modelo sem�ntico en forma de un diagrama de venn. En este diagrama, las circunferencias representan las tres ontolog�as: cartograf�a de competencias, recursos digitales y vocabulario de RyT. Estas circunferencias no tienen intersecci�n, porque cada ontolog�a representa un determinado recurso de informaci�n.

%% Diagrama de Venn ontolog�as.
%\begin{figure}[!htb]
%\centering
%\includegraphics[width=0.6\textwidth]{Arquitectura}
%\caption{Diagrama de Venn para visualizar las tres ontolog�as que conforman el modelo sem�ntico}
%\label{fig:ontocu}
%\end{figure}

Las Secci�n \ref{sec:repKrec} describe la representaci�n del conocimiento expl�cito en los recursos de informaci�n. La Secci�n \ref{sec:enrKrec} describe los axiomas para enriquecer el conocimiento en el conocimiento expl�cito. Finalmente, la Secci�n \ref{sec:byrKrec} describe la b�squeda y recuperaci�n de informaci�n en el modelo sem�ntico.

%%Mientras la recuperaci�n consiste en consultar el modelo resultante mediante un motor SPARQL y un razonador los triples del grafo RDF.
%%%La integraci�n integraci�n sem�ntica se efect�a en una memoria corporativa (MC), porque esta integraci�n considera algunas caracter�sticas importantes de una memoria corporativa como: 1) el crecimiento explosivo de recursos, 2) heterogeneidad en formato, contenido y estructura de los recursos, 3) ambig�edades en la informaci�n, 4) evoluci�n del conocimiento en los recursos (agregar, eliminar, modificar o renovar), entre otras.
%Pero, para casos pr�cticos, nuestra propuesta se pone en pr�ctica para el grupo de investigaci�n en el �rea de Redes y Telecomunicaciones del depto. de Ingenier�a El�ctrica de la Universidad Aut�noma Metropolitana Unidad Iztapalapa. Mientras, los casos de uso son: la cartograf�a de competencias y la b�squeda de recursos digitales.
\section{Representaci�n del conocimiento en los recursos}
\label{sec:repKrec} 
La primera actividad es \textbf{\textit{identificar los principales recursos de informaci�n}} para construir la memoria corporativa. Esta identificaci�n se hace a partir del an�lisis de los \textit{casos de uso}. Los recursos asociados al primer caso de uso son: profesores adscritos al �rea de RyT, estudiantes asociados a uno de �stos profesores, empleados de otras organizaciones que colaboran con los profesores. Mientras, los recursos digitales son: art�culos cient�ficos relacionados a los temas de investigaci�n, libros y p�ginas Web de referencia, tesis de maestr�a y doctorado de los alumnos, reportes t�cnicos de los profesores y sus estudiantes, presentaciones de cursos o congresos, audios de reuniones o clases, v�deo tutoriales e im�genes de referencia.

La Figura \ref{fig:mccu} muestra el esquema de la memoria corporativa de �rea de Redes y Telecomunicaciones, donde los \textit{recursos de informaci�n} est�n clasificados por caso de uso.

\begin{figure}[!htb]
\centering
%\includegraphics[width=3.2in]{Arquitectura}
\includegraphics[width=0.7\textwidth]{CasosUsoMC}
\caption{Recursos de informaci�n agrupados por casos de uso para nuestra memoria corporativa.}
\label{fig:mccu}
\end{figure}

La siguiente actividad es \textbf{\textit{adquirir el conocimiento o informaci�n}} en los recursos de informaci�n, mediante la utilizaci�n de los dos \textit{casos de uso}. En esta adquisici�n debe considerarse un hecho importante del \textbf{\textit{marco de trabajo RDF}}, el cual es ``\textit{cualquier persona, lugar, documento, objeto abstracto o f�sico se representa a partir de una serie de caracter�sticas y relaciones significativas de �ste}''. Por esta raz�n, la \textit{adquisici�n del conocimiento} se hace con base en las caracter�sticas y relaciones de los \textit{recursos de informaci�n}.

Un \textbf{\textit{diagrama de clases}} es una manera visual para describir y formalizar el conocimiento (caracter�sticas, relaciones y clases) en los recursos de informaci�n. Los \textit{diagramas de clases} de la memoria del �rea de RyT se obtienen a partir de los dos \textit{casos de uso}.

La Figura \ref{fig:dccc} presenta las clases, caracter�sticas y relaciones para los recursos persona que est�n asociados a la cartograf�a de competencias.

%%% Diagrama de clases Personas
%\begin{figure}[!htb]
%\centering
%\includegraphics[width=0.7\textwidth]{CasosUsoMC}
%\caption{Diagrama de clases para la cartograf�a competencias.}
%\label{fig:mccu}
%\end{figure}

La Figura \ref{fig:dcbrd} presenta las clases, caracter�sticas y relaciones para los documentos y archivos multimedia que est�n asociados a la b�squeda de recursos digitales.

%%% Diagrama de clases recursos digitales
%\begin{figure}[!htb]
%\centering
%\includegraphics[width=0.7\textwidth]{CasosUsoMC}
%\caption{Diagrama de clases para la b�squeda de recursos digitales.}
%\label{fig:mccu}
%\end{figure}

La siguiente actividad es la \textbf{\textit{representaci�n del conocimiento}} e informaci�n mediante el \textbf{\textit{marco de trabajo RDF}}. Esta representaci�n tiene cuatro pasos: 1) \textit{asignar identificadores a los recursos}, 2) \textit{asignar identificadores a las propiedades}, 3) \textit{reconocer si los valores de las propiedades son otros recursos o literales} y 4) \textit{construir tripletas}.

El primer paso, es asignar un URI para cada recurso de informaci�n. Esta asignaci�n de identificadores URI se hace con base en los casos de uso:

\begin{itemize}
\item Los recursos pertenecientes a la cartograf�a de competencia, emplean el siguiente URI:``\url{http://arte.izt.uam.mx/ontologies/personRyT.owl}\#'' y el prefijo ``\textit{sirp}'' para abreviar esta cadena.
\item Los recursos pertenecientes a la b�squeda de recursos digitales, utilizan el siguiente URI: ``\url{http://arte.izt.uam.mx/ontologies/digiResourceRyT.owl}\#'' y el prefijo ``\textit{sird}'' para abreviar esta cadena.
\end{itemize}

La Tabla \ref{tab:uriper} enuncia los identificadores URI de algunos profesores del �rea de RyT. En esta Tabla, la primera columna tiene los nombres de los profesores y la segunda columna enuncia los identificadores URI de �stos.

%%%
%%% Tabla ejemplos de URI's personas
%%%

La Tabla \ref{tab:urird} enuncia los identificadores URI de algunos recursos digitales de los profesores de RyT. En la primer columna de esta tabla se enuncia el nombre completo del recursos digital y la segunda columna presenta los identificadores (URI) de estos recursos.

%%%
%%% Tabla ejemplos de URI's
%%%

El siguiente paso es asignar un identificador URI para cada propiedad. Estos identificadores so construidos con base en las caracter�sticas y relaciones en los diagramas de clases (Figuras \ref{fig:dccc} y \ref{fig:dcbrd}). Los identificadores URI de estas propiedades dependen del caso de uso. Por un lado, si �stas pertenecen a la cartograf�a de competencias, entonces emplean el prefijo ```\textit{sirp}''. Por otro lado, si pertenecen a la b�squeda de recursos digitales, entonces emplean el prefijo ```\textit{sird}''.

La Tabla \ref{tab:ccprop} presenta algunos identificadores de las propiedades que pertenecen a la cartograf�a de competencias. La primera columna enuncia la caracter�stica o relaci�n, mientras la segunda columna presenta el identificador URI, el cual est� abreviado con el prefijo ``sirp''.

%%
%% Tabla de muestra de URI para propiedades
%%


La Tabla \ref{tab:brdprop} presenta algunos identificadores de las propiedades de la b�squeda de recursos digitales. La primera columna enuncia la caracter�stica o relaci�n, mientras la segunda columna presenta  el URI que est� abreviado con el prefijo ``\textit{sird}''.

%%
%% Tabla de muestra de URI para propiedades
%%

El siguiente paso es \textbf{\textit{identificar el tipo de valor}} de las caracter�sticas en el \textit{diagrama de clases}. Por un lado, si el \textit{objeto} es una cadena o un n�mero, entonces es una \textit{literal}. Por otro lado, si el \textit{objeto} es otro recurso, entonces es un \textit{identificador URI}. En el caso de las \textbf{\textit{relaciones}}, los valores son otros recursos, por ello, �stos deben ser \textit{identificadores URI}.

El �ltimo paso en la representaci�n es mapear los recursos, propiedades, as� como literales y otros objetos en forma de tripletas. En esta tesis, el mapeo se hace mediante la combinaci�n de formularios y scripts. Los objetivos de los formularios son la recuperaci�n y almacenamiento de la informaci�n acerca de los recursos en \textit{hojas de c�lculo}. Mientras, la finalidad de los scripts es transformar la informaci�n de las \textit{hojas de c�lculo} en forma de tripletas RDF para su almacenamiento en archivos ``\textit{.rdf}''.

Este es el procedimiento para realizar el mapeo de informaci�n a tripletas RDF:

\begin{enumerate}
\item Identificar la informaci�n que debe ser adquirida en los recursos con base en los diagramas de clases.
\item Construir los formularios para los recursos de informaci�n (persona, documento y multimedia) mediante Google Form\footnote{Google, ``Formularios,''  Disponible en: \url{https://support.google.com/drive/topic/1360904?hl=es&ref_topic=2811744}}, con el prop�sito de agilizar el proceso de recopilaci�n de la informaci�n en los recursos.
\item Enviar los formularios v�a email a los profesores o alumnos, para que ellos escriban la informaci�n sobre las caracter�sticas y relaciones de los recursos.
\item Recuperar y almacenar las respuestas de cada formulario en una de tres \textit{hojas de c�lculo} (persona, documento y multimedia).
\item Descargar la informaci�n de cada \textit{hoja de c�lculo} en un archivo CSV\label{sym:csv} (persona, documento y multimedia).
\item Transformar cada fila de un archivo csv a un conjunto de tripletas RDF, mediante los scripts que est�n escritos en Java y con la librer�a Jena.
\item Almacenar las tripletas RDF asociadas a una fila (descripci�n sem�ntica de un recurso) en un archivo ``\textit{rdf}'' con la sintaxis de serializaci�n Turtle.
\end{enumerate}

%%Dibujo proceso de eneraci�n de tripletas.

La Figura \ref{fig:drper} presenta el mapeo de las caracter�sticas y relaciones de una persona del �rea de RyT en forma de tripletas RDF.

%%
%% Figura descripciones de recursos del tipo persona
%%

La Figura \ref{fig:drper} presenta el mapeo de las caracter�sticas y relaciones para un recurso documento perteneciente al �rea de RyT en forma de tripletas RDF.

%%
%% Figura descripciones de recursos del tipo persona
%%

Las tripletas RDF del conocimiento expl�cito para los recursos \textit{persona} conforman el componente asertivo (ABox) de la \textit{ontolog�a cartograf�a de competencias}. Mientras, las tripletas RDF del conocimiento expl�cito de los recursos \textit{documento y multimedia} constituyen el componente asertivo de la \textit{ontolog�a Recursos Digitales}.

\section{Enriquecimiento del conocimiento en el modelo}
\label{sec:enrKrec} 
La etapa de representaci�n del conocimiento, nos permite describir el conocimiento expl�cito en los \textit{recursos de informaci�n}. Ahora bien, este conocimiento puede ser enriquecido mediante la introducci�n de axiomas o reglas de inferencia.

Los axiomas permiten representar el conocimiento impl�cito en los recursos y las relaciones de �stos. Por ejemplo, los profesores, empleados y estudiantes son personas, por ello, deben tenerse tres axiomas que establezcan que un profesor es una persona, un empleado es una persona y un estudiante es una persona.

Para cada \textit{caso de uso} debe encontrarse el respectivo conjunto de axiomas (TBox). Este proceso de b�squeda de axiomas se hace con base en los siguientes elementos:1) \textit{diagramas de clase}, 2) \textit{cualidades en las relaciones} y 3) \textit{operaciones de la teor�a de grupos}. A continuaci�n, se describen y argumentan los \textbf{\textit{axiomas}} que se identificaron para las ontolog�as de \textit{cartograf�a de competencias y recursos digitales}.

\textbf{Herencia de clases}\\
En el �rea de Redes y Telecomunicaciones, las personas pueden agruparse en cuatro clases b�sicas: \textbf{\textit{Estudiante}}, \textbf{\textit{Profesor}}, \textbf{\textit{Investigador}} y \textbf{\textit{Empleado}}. Las personas pueden pertenecer a m�s de una clase, por ejemplo, un profesor puede ser un investigador o un empleado, as� como un estudiante puede ser un profesor o un empleado. Los profesores y empleados son profesionistas, por ello, la clase \textbf{\textit{Profesionista}} es super-clase de \textbf{\textit{Profesor}} y \textbf{\textit{Empleado}}. Finalmente, cualquier individuo perteneciente a una de estas cinco clases, pertenece es una persona. De esta manera, las cinco clases tienen como super-clase la clase \textbf{\textit{Persona}}. La Figura \ref{fig:jdc} muestra la jerarqu�a de clases para los recursos persona.

%%
%% Figura herencia de clases personas
%%

En el �rea de Redes y Telecomunicaciones, los principales recursos digitales se pueden agrupar en: \textbf{\textit{Art�culo}}, \textbf{\textit{Libro}}, \textbf{\textit{Tesis}}, \textbf{\textit{P�gina Web}}, \textbf{\textit{Reporte T�cnico}}, \textbf{\textit{Audio}}, \textbf{\textit{V�deo}}, \textbf{\textit{Presentaci�n}} e \textbf{\textit{Imagen}}. Estas nueve clases son disjuntos, por ello, no tienen individuos en com�n. Los recursos digitales se agrupan en dos clases generales: \textbf{\textit{Documentos}} y \textbf{\textit{Mutimedia}}. Las primeras cinco clases Art�culo, Libro, Tesis, P�gina Web, Reporte T�cnico tienen como super-clase a la clase \textbf{\textit{Documento}}. Mientras las otras cuatro clases b�sicas tienen como super-clase a la clase \textbf{\textit{Multimedia}}. Finalmente, cualquier individuo de estas once once clases es un recurso digital, por ello, las once clases son subclases de la clase \textbf{\textit{Recurso Digital}}. La Figura \ref{fig:jdrd} muestra la jerarqu�a de clases para los recursos digitales.

%%
%% Figura herencia de clases recursos digitales
%%

\textbf{Herencia de propiedades}\\
Las propiedades \textbf{\textit{lee}}, \textbf{\textit{habla}} y \textbf{\textit{escribe}} son habilidades ling��sticas. Estas propiedades pueden generalizarse mediante la propiedad \textbf{\textit{tiene lenguaje}}, con el fin de indicar que una persona tiene alg�n conocimiento ling��sticos en un idioma. Por ello, las propiedades \textit{lee, habla y escribe} son subpropiedades de la propiedad \textit{tiene lenguaje}.

Las propiedades \textbf{\textit{trabaja en}} y \textbf{\textit{estudia en}} son relaciones entre una persona y un lugar de trabajo, por ello estas dos propiedades son generalizadas a partir de la propiedad \textit{tiene lugar de trabajo}.

Las propiedades \textbf{\textit{tiene asesor}} y \textbf{\textit{tiene colega}} son relaciones que vinculan a dos personas. La propiedad \textit{tiene asesor} vincula a un estudiante con un profesor. Mientras, la propiedad \textit{tiene colega} vincula a dos profesionistas. Estas dos propiedades pueden generalizarse mediante la propiedad \textbf{\textit{conoce a}}, es decir, \textit{tiene colega} y \textit{tiene asesor} son subpropiedades de la propiedad \textit{conoce a}.

La Figura \ref{fig:dpXrp} muestra la jerarqu�a de propiedades para el \textit{caso de uso} de la cartograf�a de competencias.

%%
%% Figura herencia de propiedades recursos digitales
%%

Axiomas de herencia de propiedades para la ontolog�a de recursos persona

Axiomas de herencia de propiedades para la ontolog�a de recursos digitales

En esta tesis, la finalidad de los axiomas es efectuar tareas de inferencia.

Hay que mostrar las jerarqu�as de clases y de propriedades, as� como los axiomas de nuestras ontolog�as.

Dado que los casos de uso son independientes, decidimos utilizar una ontolog�a para cada uno y otra que es de uso com�n (ODARyT).


\section{B�squeda y recuperaci�n de informaci�n en el modelo}
\label{sec:byrKrec} 
Aqu� hay que retomar nuestros casos de uso y expandirlos con consultas espec�ficas que se hacen sobre las ontolog�as definidas en la secci�n anterior.

Es importante hacer notar que ciertas consultas explotan a los axiomas, por lo que se requiere de un razonador.
\chapter{Prototipo}
\label{cap:piu}
La \textit{integraci�n sem�ntica de recursos} es el proceso de \textit{b�squeda y recuperaci�n de informaci�n} en los \textit{recursos de informaci�n} a partir del uso de tecnolog�as sem�nticas. Esta \textit{b�squeda y recuperaci�n} debe efectuarse por un usuario que maneje o tenga estas nociones:

\begin{itemize}
\item \textbf{Tecnolog�as sem�nticas}: tripletas rdf, consultas SPARQL, ABox, TBox, patrones, variables resultado o auxiliares, inferencia.
\item \textbf{Herramientas y operaciones de gesti�n de tripletas}: triplestore, motor de inferencia, motor de consulta, carga de modelo, forma de recuperaci�n de la informaci�n, inferencia en un modelo.
\item \textbf{Representaci�n del dominio}: vocabularios (URIs), propiedades, axiomas, prefijos.
\end{itemize}

Sin embargo, no todos los \textit{profesores-investigadores} y \textit{estudiantes} tienen estos conocimientos, es decir, no todos los usuarios del �rea de redes y telecomunicaciones aprovechan la \textit{integraci�n sem�ntica de recursos}. Este motivo 

Por esta raz�n, se propone construir un prototipo de \textit{interfaz gr�fica de usuario}. La \textbf{\textit{finalidad}} de �ste es \textit{recuperar informaci�n} en un \textit{modelo sem�ntico}, para responder una consulta de cualquier usuario.

Estos son los objetivos funcionales de este prototipo:
\begin{itemize}
\item Estructurar las consultas de un usuario.
\item Permitir b�squedas complejas de los usuarios.
\item Recuperar la informaci�n de los \textit{recursos de informaci�n} en un modelo sem�ntico.
\item Utilizar modelos sem�nticos con inferencia.
\item Proporcionar un servicio transparente con respecto al uso de tecnolog�as sem�nticas.
\item Permitir el acceso v�a web.
\end{itemize}

nosotros proponemos un prototipo para la interacci�n amigable de los usuarios con el modelo RDF, de esta forma, un usuario podr� consultar al modelo RDF y visualizar los resultados asociados a la misma, sin que �ste tenga conocimientos en las Tecnolog�as Sem�nticas.


Los objetivos particulares del prototipo son: 1) permitir a los usuarios estructurar su pregunta, para que se mapee a una consulta SPARQL. 2) cargar el modelo RDF y los axiomas, e invocar el razonador para inferir nuevas relaciones al grafo RDF. 3) invocar el motor de consulta SPARQL, para que ejecute la consulta SPARQL al grafo RDF inferido. 4) mostrar para cada resultado un conjunto de datos significativos (nombre, ruta, lenguaje, etc.). A partir de estos objetivos se plantea la siguiente arquitectura.


%%%%%Figura casos de uso%%%%%
\begin{figure}[!hp]
\centering
\includegraphics[height=3.5in, width=1.05\textwidth]{casosUsoIWGen}
\caption{Casos de uso para el prototipo integrador sem�ntico.}
\label{fig:iwnp}
\end{figure}


%%%%%Descripci�n caso vlidaci�n%%%%%
\begin{table}[!hp]
\renewcommand{\arraystretch}{1.0}
\centering
\begin{tabular}{| >{\raggedright\arraybackslash}m{1.5in} | >{\raggedright\arraybackslash}m{4.5in} | }
\hline 
\textbf{Nombre}& Validaci�n Usuario.\\ 
\hline
\textbf{Objetivo:}& Validar a un usuario para el uso de la interfaz Web de la integraci�n sem�ntica de recursos.\\
\hline
\textbf{Actor}& Usuarios.\\
\hline
\textbf{Pre-condici�n}& El usuario debe estar registrado en el prototipo.\\
\hline
\textbf{Secuencia de interacciones} & Prototipo de Interfaz Web para la Integraci�n Sem�ntica de Recursos.\\
\hline
\multicolumn{2}{| >{\raggedright\arraybackslash}m{6.0in} |}{1.- El usuario introduce el URL del Login en su navegador, por ejemplo: \url{http://localhost:8080/appsir/login.html}.}\\
\multicolumn{2}{| >{\raggedright\arraybackslash}m{6.0in} |}{2.- El usuario escribe su \textit{nombre de usuario} y \textit{contrase�a}, despu�s da clic en Enviar (Send).}\\
\multicolumn{2}{| >{\raggedright\arraybackslash}m{6.0in} |}{3.- La interfaz Web cambia la p�gina Web Login a la p�gina principal que se denomina Home.} \\
\hline
\textbf{Flujos secundarios} &\\
\hline
\multicolumn{2}{| >{\raggedright\arraybackslash}m{6.0in} |}{3b.- El usuario escribe err�neamente su \textit{nombre de usuario} o \textit{contrase�a}, despu�s da clic en Enviar (Send).}\\
\multicolumn{2}{| >{\raggedright\arraybackslash}m{6.0in} |}{4b.- La interfaz Web recarga la p�gina Web Login y borra los datos de los campos \textit{nombre de usuario} y \textit{contrase�a}.} \\
\hline
\textbf{Post-condici�n}& El usuario es valido y la interfaz Web presenta la p�gina Web principal.\\
\hline
\end{tabular}
\caption{Especificaci�n para validar a un usuario.}
\label{tab:iwlog}
\end{table}

%%%%%Figura formulario para validar%%%%%
\begin{figure}[hp]
\centering
\subfigure[Formulario Web: Validar al usuario]{
\includegraphics[totalheight=0.40\textheight]{FormLogin} 
}
\subfigure[Interfaz Web: Home]{
\includegraphics[totalheight=0.40\textheight]{IntWebHome} 
}
\caption{Validaci�n del acceso de un usuario al prototipo de integraci�n sem�ntica de recursos.}
\label{fig:iwbxt}
\end{figure}

%%%%%Navegaci�n entre recursos de informaici�n%%%%%
\begin{table}[hp]
\renewcommand{\arraystretch}{1.0}
\centering
\begin{tabular}{| >{\raggedright\arraybackslash}m{1.5in} | >{\raggedright\arraybackslash}m{4.5in} | }
\hline 
\textbf{Nombre}& Navegaci�n entre personas/documentos/multimedia.\\ 
\hline
\textbf{Objetivo:}& Desplazarse y visualizar las \textit{caracter�sticas significativas} de las \textit{personas}, \textit{documentos} o \textit{recursos multimedia} del �rea de Redes y Telecomunicaciones.\\
\hline
\textbf{Actor}& Usuarios.\\
\hline
\textbf{Pre-condici�n}& El usuario est� en la p�gina \textit{Home}.\\
\hline
\textbf{Secuencia de interacciones} & Prototipo de Interfaz Web para la Integraci�n Sem�ntica de Recursos.\\
\hline
\multicolumn{2}{| >{\raggedright\arraybackslash}m{6.0in} |}{1.- El usuario selecciona una de estas tres opciones: \textit{Personas (Persons)}, \textit{Documentos (Documents)}, \textit{Multimedia (Multimedia)}.}\\
\multicolumn{2}{| >{\raggedright\arraybackslash}m{6.0in} |}{2.- La interfaz Web muestra algunas caracter�sticas significativas que dependen del tipo de \textit{recursos de informaci�n}.}\\
\multicolumn{2}{| >{\raggedright\arraybackslash}m{6.0in} |}{2.1.- Si la opci�n es Persona, entonces son mostrados el \textit{nombre (Name)}, \textit{ocupaci�n (Occupation)}, \textit{g�nero (Gender)}, \textit{email (Email)} y \textit{p�gina personal (WebSite)} para todas las \textit{personas} en la \textit{memoria corporativa} del �rea RyT.} \\
\multicolumn{2}{| >{\raggedright\arraybackslash}m{6.0in} |}{2.2.- Si la opci�n es Documento, entonces son mostrados el \textit{t�tulo (Title)}, \textit{idioma (Language)}, \textit{a�o (Year)} y \textit{ruta (Path)} para todos los \textit{documentos} en la \textit{memoria corporativa} del �rea RyT.} \\
\multicolumn{2}{| >{\raggedright\arraybackslash}m{6.0in} |}{2.3.- Si la opci�n es Multimedia, entonces son mostrados el \textit{t�tulo (Title)}, \textit{autores (Authors)}, \textit{A�o (Year)}, \textit{tipo recurso (Resource Type)}, \textit{idioma (Language)} y \textit{ruta (Path)} para todos los \textit{archivos multimedia} en la \textit{memoria corporativa} del �rea RyT.} \\
\hline
\textbf{Post-condici�n}& La interfaz Web muestra una secci�n para filtrar los recursos Persona, Documento o Multimedia a partir de las subclases de �stos.\\
&Tambi�n, dependiendo la clase Persona, Documento o Multimedia, la \textit{interfaz Web} muestra un bot�n de \textit{b�squeda avanzada por clase (Advanced Search Person, Document, Multimedia)}, y para cada \textit{recurso de informaci�n} se muestra un bot�n de \textit{m�s detalles} \textit{(More Details)}.\\
\hline
\end{tabular}
\caption{Especificaci�n para la navegaci�n entre personas/documentos/multimedia.}
\label{tab:iwnav}
\end{table}

%%%%% Figuras pantallas resultados de la navegaci�n entre recursos.
\begin{figure}[hp]
\centering
\subfigure[Interfaz Web: navegaci�n entre personas del �rea RyT]{
\includegraphics[totalheight=0.28\textheight]{IntWebNavPer} 
}
\subfigure[Interfaz Web: navegaci�n entre documentos de la memoria corporativa del �rea RyT]{
\includegraphics[totalheight=0.28\textheight]{IntWebNavDoc} 
}
\subfigure[Interfaz Web: navegaci�n entre recursos multimedia de la memoria corporativa del �rea RyT]{
\includegraphics[totalheight=0.28\textheight]{IntWebNavMult} 
}
\caption{Interfaces Web: navegaci�n entre personas/documentos/multimedia.}
\label{fig:iwnav}
\end{figure}

%%%%% Caso de uso b�squeda por topicos%%%%%
\begin{table}[hp]
\renewcommand{\arraystretch}{1.0}
\centering
\begin{tabular}{| >{\raggedright\arraybackslash}m{1.5in} | >{\raggedright\arraybackslash}m{4.5in} | }
\hline 
\textbf{Nombre}& B�squeda por t�pico.\\ 
\hline
\textbf{Objetivo:}& Buscar \textit{informaci�n significativa} en los \textit{recursos de informaci�n} a partir de los t�picos de redes y telecomunicaciones.\\
\hline
\textbf{Actor}& Usuarios.\\
\hline
\textbf{Pre-condici�n}& El usuario est� en la p�gina \textit{Home}.\\
\hline
\textbf{Secuencia de interacciones} & Prototipo de Interfaz Web para la Integraci�n Sem�ntica de Recursos.\\
\hline
\multicolumn{2}{| >{\raggedright\arraybackslash}m{6.0in} |}{1.- El usuario selecciona la opci�n \textit{T�picos de redes (Topics RyT)}.}\\
\multicolumn{2}{| >{\raggedright\arraybackslash}m{6.0in} |}{2.- La interfaz Web muestra un formulario para estructurar la pregunta.} \\
\multicolumn{2}{| >{\raggedright\arraybackslash}m{6.0in} |}{3.- El usuario escribe los t�picos de RyT, separando cada uno de �stos con una coma.} \\
\multicolumn{2}{| >{\raggedright\arraybackslash}m{6.0in} |}{4.- El usuario selecciona el operador \textit{y (and)} u \textit{o (or)} para comparar los t�picos de RyT.} \\
\multicolumn{2}{| >{\raggedright\arraybackslash}m{6.0in} |}{5.- El usuario selecciona una opci�n de dos; \textit{exactamente los t�picos escritos} o \textit{t�picos asociados con aquellos ya escritos}.} \\
\multicolumn{2}{| >{\raggedright\arraybackslash}m{6.0in} |}{6.- El usuario selecciona el tipo de \textit{recurso de informaci�n}: \textit{Todos}, \textit{Persona}, \textit{Documento}, \textit{Multimedia}, \textit{Art�culo}, \textit{Libro} y \textit{Presentaci�n}.} \\
\multicolumn{2}{| >{\raggedright\arraybackslash}m{6.0in} |}{7.- Indicar como criterio de b�squeda el idioma o lenguaje.} \\
\multicolumn{2}{| >{\raggedright\arraybackslash}m{6.0in} |}{7.1.- El usuario selecciona el \textit{check box} \textit{lenguaje (Language)}.} \\
\multicolumn{2}{| >{\raggedright\arraybackslash}m{6.0in} |}{7.2.- El usuario selecciona un lenguaje del \textit{combo box}.} \\
\multicolumn{2}{| >{\raggedright\arraybackslash}m{6.0in} |}{8.- El usuario elige un opci�n para ordenar los resultados de la b�squeda.} \\
\multicolumn{2}{| >{\raggedright\arraybackslash}m{6.0in} |}{9.- Establecer la cantidad de resultados a mostrar.} \\
\multicolumn{2}{| >{\raggedright\arraybackslash}m{6.0in} |}{9.1.- El usuario selecciona el \textit{check box} \textit{n�mero de resultados (Number of results)}.} \\
\multicolumn{2}{| >{\raggedright\arraybackslash}m{6.0in} |}{9.2.- El usuario escribe el n�mero de recursos a recuperar.} \\
\multicolumn{2}{| >{\raggedright\arraybackslash}m{6.0in} |}{10.- El usuario da clic da en Enviar (Send).} \\
\multicolumn{2}{| >{\raggedright\arraybackslash}m{6.0in} |}{11.- La interfaz Web muestra el \textit{nombre (Name)}, \textit{tipo de recurso (Resource Type)}, \textit{ruta o p�gina web (URL)} y \textit{t�picos de redes (Topics RyT)}, para cada recurso resultante.} \\
\hline
\textbf{Post-condici�n}& La interfaz Web muestra el conjunto de resultados de la b�squeda.\\
&Tambi�n, la interfaz Web muestra para cada recurso el bot�n de \textit{m�s detalles} \textit{(More Details)}.\\
\hline
\end{tabular}
\caption{Especificaci�n para la b�squeda por t�pico(s) de redes y telecomunicaciones.}
\label{tab:iwnm}
\end{table}

%%%%%Figura pantalla b�squeda por topicos
\begin{figure}[hp]
\centering
\subfigure[Formulario Web: b�squeda por t�picos]{
\includegraphics[totalheight=0.40\textheight]{FormAllRes} 
}
\subfigure[P�gina Din�mica: Resultados de b�squeda]{
\includegraphics[totalheight=0.40\textheight]{IntWebNavAll} 
}
\caption{Formulario Web para la b�squeda por temas en todos los recursos de informaci�n y la p�gina Web mostrando los resultados de esta b�squeda.}
\label{fig:iwbxt}
\end{figure}

%%%%%Filtrar por subclases de las clases Persona, Documento o Multimedia%%%%%
\begin{table}[hp]
\renewcommand{\arraystretch}{1.0}
\centering
\begin{tabular}{| >{\raggedright\arraybackslash}m{1.5in} | >{\raggedright\arraybackslash}m{4.5in} | }
\hline 
\textbf{Nombre}& Filtrar por tipo personas/documentos/multimedia.\\ 
\hline
\textbf{Objetivo:}& Buscar los \textit{recursos de informaci�n} a partir de las subclases de Persona/Documento/Multimedia y mostrar un conjunto de \textit{caracter�sticas significativas} de �stos.\\
\hline
\textbf{Actor}& Usuarios.\\
\hline
\textbf{Pre-condici�n}& El usuario est� en la p�gina de \textit{Navegaci�n entre personas/documentos/multimedia}.\\
\hline
\textbf{Secuencia de interacciones} & Prototipo de Interfaz Web para la Integraci�n Sem�ntica de Recursos.\\
\hline
\multicolumn{2}{| >{\raggedright\arraybackslash}m{6.0in} |}{1.- El usuario selecciona la opci�n para filtrar los recursos a partir de las subclases de las clases Persona, Documento o Multimedia.}\\
\multicolumn{2}{| >{\raggedright\arraybackslash}m{6.0in} |}{1.1.- Si la navegaci�n es entre personas, entonces el usuario puede elegir las opciones: \textit{estudiante (Student)}, \textit{profesionista (Professional)}, \textit{profesor (Teacher)}, \textit{empleado (Employee)} e \textit{investigador (Researcher)}.}\\
\multicolumn{2}{| >{\raggedright\arraybackslash}m{6.0in} |}{1.2.- Si la navegaci�n es entre documentos, entonces el usuario puede elegir las opciones: \textit{libro (Book)}, \textit{art�culo (Paper)}, \textit{tesis (Thesis)}, \textit{reporte t�cnico (TechnicalReport)} e \textit{p�gina web (Webpage)}.}\\
\multicolumn{2}{| >{\raggedright\arraybackslash}m{6.0in} |}{1.3.- Si la navegaci�n es entre multimedia, entonces el usuario puede elegir las opciones: \textit{audio (Audio)}, \textit{imagen (Image)}, \textit{presentaci�n (Presentation)} y \textit{v�deo (Video)}.}\\
\multicolumn{2}{| >{\raggedright\arraybackslash}m{6.0in} |}{2.- La interfaz Web muestra algunas \textit{caracter�sticas significativas} de los \textit{recursos resultantes}.}\\
\multicolumn{2}{| >{\raggedright\arraybackslash}m{6.0in} |}{2.1.- Si es Personas, entonces son mostrados el \textit{nombre (Name)}, \textit{ocupaci�n (Occupation)}, \textit{g�nero (Gender)}, \textit{email (Email)} y \textit{p�gina personal (WebSite)} para todas las \textit{personas} en la \textit{memoria corporativa} del �rea RyT.} \\
\multicolumn{2}{| >{\raggedright\arraybackslash}m{6.0in} |}{2.2.- Si la opci�n es Documentos, entonces son mostrados el \textit{t�tulo (Title)}, \textit{idioma (Language)}, \textit{a�o (Year)} y \textit{ruta (Path)} para todos los \textit{documentos} en la \textit{memoria corporativa} del �rea RyT.} \\
\multicolumn{2}{| >{\raggedright\arraybackslash}m{6.0in} |}{2.3.- Si la opci�n es Multimedia, entonces son mostrados el \textit{t�tulo (Title)}, \textit{autores (Authors)}, \textit{A�o (Year)}, \textit{tipo recurso (Resource Type)}, \textit{idioma (Language)} y \textit{ruta (Path)} para todos los \textit{archivos multimedia} en la \textit{memoria corporativa} del �rea RyT.} \\
\hline
\textbf{Post-condici�n}& La interfaz Web muestra nuevamente la secci�n para filtrar los recursos Persona, Documento o Multimedia a partir de las subclases de �stos.\\
&Tambi�n, la \textit{interfaz Web} dependiendo el tipo de recursos (Persona, Documento o Multimedia), muestra un bot�n de \textit{b�squeda avanzada (Advanced Search Person, Document o Multimedia)}, y para cada \textit{recurso respuesta} se muestra un bot�n de \textit{m�s detalles} \textit{(More Details)}.\\
\hline
\end{tabular}
\caption{Especificaci�n para filtrar los recursos personas/documentos/multimedia por la subclases de �stos.}
\label{tab:iwfilt}
\end{table}

%%%%%%%%%%%%%%%%%%%%
%%    FIGURA DE   %%
%%%%%%%%%%%%%%%%%%%%

%%%%% Caso de uso b�squeda por topicos%%%%%
\begin{table}[hp]
\renewcommand{\arraystretch}{1.0}
\centering
\begin{tabular}{| >{\raggedright\arraybackslash}m{1.5in} | >{\raggedright\arraybackslash}m{4.5in} | }
\hline 
\textbf{Nombre}& B�squeda Avanzada de Personas.\\ 
\hline
\textbf{Objetivo:}& Estructurar la consulta de un usuario y buscar la informaci�n significativa de las personas del �rea de RyT, para responder esta consulta. \\ 
\hline
\textbf{Actor}& Usuarios.\\
\hline
\textbf{Pre-condici�n}& El usuario est� en la p�gina de \textit{navegaci�n entre personas}.\\
\hline
\textbf{Secuencia de interacciones} & Prototipo de Interfaz Web para la Integraci�n Sem�ntica de Recursos.\\
\hline
\multicolumn{2}{| >{\raggedright\arraybackslash}m{6.0in} |}{1.- El usuario selecciona la opci�n \textit{T�picos de redes (Topics RyT)}.}\\
\multicolumn{2}{| >{\raggedright\arraybackslash}m{6.0in} |}{2.- La interfaz Web muestra un formulario para estructurar la pregunta.} \\
\multicolumn{2}{| >{\raggedright\arraybackslash}m{6.0in} |}{3.- El usuario escribe los t�picos de RyT, separando cada uno de �stos con una coma.} \\
\multicolumn{2}{| >{\raggedright\arraybackslash}m{6.0in} |}{4.- El usuario selecciona el operador \textit{y (and)} u \textit{o (or)} para comparar los t�picos de RyT.} \\
\multicolumn{2}{| >{\raggedright\arraybackslash}m{6.0in} |}{5.- El usuario selecciona una opci�n de dos; \textit{exactamente los t�picos escritos} o \textit{t�picos asociados con aquellos ya escritos}.} \\
\multicolumn{2}{| >{\raggedright\arraybackslash}m{6.0in} |}{6.- El usuario selecciona el tipo de \textit{recurso de informaci�n}: \textit{Todos}, \textit{Persona}, \textit{Documento}, \textit{Multimedia}, \textit{Art�culo}, \textit{Libro} y \textit{Presentaci�n}.} \\
\multicolumn{2}{| >{\raggedright\arraybackslash}m{6.0in} |}{7.- Indicar como criterio de b�squeda el idioma o lenguaje.} \\
\multicolumn{2}{| >{\raggedright\arraybackslash}m{6.0in} |}{7.1.- El usuario selecciona el \textit{check box} \textit{lenguaje (Language)}.} \\
\multicolumn{2}{| >{\raggedright\arraybackslash}m{6.0in} |}{7.2.- El usuario selecciona un lenguaje del \textit{combo box}.} \\
\multicolumn{2}{| >{\raggedright\arraybackslash}m{6.0in} |}{8.- El usuario elige un opci�n para ordenar los resultados de la b�squeda.} \\
\multicolumn{2}{| >{\raggedright\arraybackslash}m{6.0in} |}{9.- Establecer la cantidad de resultados a mostrar.} \\
\multicolumn{2}{| >{\raggedright\arraybackslash}m{6.0in} |}{9.1.- El usuario selecciona el \textit{check box} \textit{n�mero de resultados (Number of results)}.} \\
\multicolumn{2}{| >{\raggedright\arraybackslash}m{6.0in} |}{9.2.- El usuario escribe el n�mero de recursos a recuperar.} \\
\multicolumn{2}{| >{\raggedright\arraybackslash}m{6.0in} |}{10.- El usuario da clic da en Enviar (Send).} \\
\multicolumn{2}{| >{\raggedright\arraybackslash}m{6.0in} |}{11.- La interfaz Web muestra el \textit{nombre (Name)}, \textit{tipo de recurso (Resource Type)}, \textit{ruta o p�gina web (URL)} y \textit{t�picos de redes (Topics RyT)}, para cada recurso resultante.} \\
\hline
\textbf{Post-condici�n}& La interfaz Web muestra el conjunto de resultados de la b�squeda.\\
&Tambi�n, la interfaz Web muestra para cada recurso el bot�n de \textit{m�s detalles} \textit{(More Details)}.\\
\hline
\end{tabular}
\caption{Especificaci�n para la b�squeda por t�pico(s) de redes y telecomunicaciones.}
\label{tab:iwnm}
\end{table}


%%%%%%%

%%%%%%%La integraci�n sem�ntica de recursos (ISR) a partir del uso de un
%%%%%%%triplestore, no es una tarea que cualquier usuario (profesor o estudiante) puede hacer,
%%%%%%%ya que �ste debe estar familiarizado con el triplestore, el
%%%%%%%lenguaje de consulta SPARQL y las ternas RDF.
%%%%%%%
%%%%%%%
%%%%%%%La manera de probar el enfoque sem�ntico y la integraci�n sem�ntica es desarrollar un prototipo de sistema.
%%%%%%%
%%%%%%%Este prototipo basa su funcionamiento en los elementos, tecnolog�as y est�ndares actuales de la Web Sem�ntica.
%%%%%%%
%%%%%%%Nosotros para construir el sistema seguimos una serie de actividades. En donde estas actividades se apegan a la metodolog�a que propusimos.

%%%%%%%%%%%%%%%%%%%%%%%%%%%
%%%%%%%%
%%%%%%%%%%%%%%%%%%%%%%%%%%%


%%No solo basta con un razonador también se requiere un modulo integrador de la información que transforme las consultas de los usuarios expresadas en lenguaje natural, a lenguaje que sea interpretado por el razonador. Además este integrador es el encargado de regresar los enlaces y los datos del recurso. De esta manera el motor de búsqueda queda de la siguiente manera:

%%Para llevar acabo está integración de la información es necesario un prototipo que satisfaga más eficientemente las consultas que escribe el usuario. De esta manera es necesario un análisis documental y técnico de los módulos de Anotaciones, Ontología de Dominio y Razonadores. Con la finalidad de tener una propuesta de sistema con las últimas novedades hechas en búsqueda y recuperación de la información basada en la semántica de los recursos.

%%se requiere proporcionar interfaces fáciles de usar, para simplificar a los miembros el proceso de generación de anotaciones y colocar en contexto su trabajo.  Un buen enfoque para un sistema de anotaciones es aquel donde se maneja una única interfaz, y es en esta donde los usuarios crean, modifican y comparten sus anotaciones.

%%La interfaz debe tener ..., para que los usuarios estructuren sus consulta, capturando los valores que desean buscar. En la interfaz debe proporcionar un navegador entre personas, documentos, multimedia, para que los usuarios que no tienen algún conocimiento previo de las personas y recursos digitales, puedan tener una vision general de la información de los recursos.

%%La interfaz debe permitir hacer las siguientes actividades a los usuarios:
%%•	Login.
%%•	Navegar entre personas.
%%o	Filtrar por ocupación.
%%o	Mostrar la información más detallada de una persona.
%%o	Búsqueda Avanzada de las personas.
%%•	Navegar entre documentos.
%%o	Filtrar por clase de documento.
%%o	Mostrar la información más detallada de un documento.
%%o	Búsqueda Avanzada de documentos.
%%•	Navegar entre multimedia.
%%o	Filtrar por clase multimedia.
%%o	Mostrar la información más detallada de un recurso multimedia.
%%o	Búsqueda Avanzada de recursos multimedia.
%%•	Búsqueda en todos los recursos de información por información semejante.

%%Las distintas aplicaciones que hay pueden ser elaboradas para Windows, Linux, Macs, u otros sistemas operativos, y también se pueden tener distintas versiones del mismo sistema para poder trabajar entre los distintos sistemas operativos. Lo ideal es que sin importar cual sea el sistema operativo el usuario pueda realizar sus anotaciones semánticas, para logara esto se puede emplear una aplicación web o aplicaciones elaboradas en java.
%%Si se emplea una aplicación web no es necesario instalar algún software extra, solo basta que el usuario acceda utilizando su navegador web de preferencia y comience el proceso de creación de anotaciones semánticas.

%%Por otro lado al usar una aplicación basada en Java, es necesario tener Java Development Kit (JDK) que es independiente de la plataforma, para tener un entorno amigable al usuario.

%%Finalmente, los usuarios necesitan una interfaz de usuario, para la consulta de información de las tripletas. Nosotros proponemos una interfaz amigable que sea accesible vía Web. De esta manera, los usuarios no instalan ningún componente y simplemente acceden a la página Web del sistema.
%%La interfaz al ser accesible vía Web, requiere ser instalada en un servidor Web. Para tomar la decisión sobre qué servidor es el apropiado para la interfaz. Nosotros debemos tomar en cuenta, el lenguaje de implementación del triplestore. Si el lenguaje es PHP9, entonces, podemos emplear un servidor HTTP Apache10. En otro caso, si el lenguaje es Java11 y permite implementar Servlet, entonces el servidor es Apache Tomcat12. 

\chapter{Desarrollo experimental}
\label{cap:exp}
%%%% Observaci�n consiste en la medida y registro de hechos observables.
La \textit{integraci�n sem�ntica de los recursos} es el proceso de b�squeda y recuperaci�n de informaci�n en un \textit{grafo de conocimiento (ontolog�a)}. Un \textit{motor de b�squeda de tripletas} es el mecanismo encargado de realizar la consulta de informaci�n en los grafos de conocimiento, para responder una consulta dada. Este \textit{motor de tripletas} generalmente pertenece a un triplestore. En esta tesis, se emplea el \textit{triplestore Jena}. �ste proporciona dos componentes importantes: 1) \textit{un motor de b�squeda} para tripletas RDF, denominado \textbf{\textit{ARQ}} y 2) \textit{un motor de inferencia} que soporta los axiomas de nuestras ontolog�as, los cuales son descritos en la Secci�n \ref{sec:enrKrec}.

\section{Observaciones}
La \textit{calidad de los resultados} depende del uso o no de inferencia en nuestros modelo sem�ntico. Si Jena emplea un \textit{modelo sin inferencia} como el ejemplo de la Figura \ref{fig:grafoSR}, entonces el motor ARQ puede proporcionar todos, varios o ning�n resultado. Esta variedad en la entrega de resultados, depende de las consultas SPARQL: 1) \textit{consultas sobre las declaraciones de los recursos}, como la consulta de la Figura \ref{fig:q112rql}, 2) \textit{consultas que agrupan varios patrones para un criterio de b�squeda}, como las consultas de las Figuras \ref{fig:q2rqlSR} y \ref{fig:q2torqlSR}, y 3) \textit{consultas simplificadas}, como en las Figuras \ref{fig:q2rqlCR} y \ref{fig:q2torqlCR}. En contraste, Jena puede entregar mejores resultados cuando emplea un modelo que se obtiene de la inferencia en una ontolog�a. Un ejemplo de este modelo se presenta en la Figura \ref{fig:grafoCR}.

Una caracter�stica asociada al uso de inferencia, es el impacto en el \textit{tiempo de procesamiento} de \textit{Jena} para responder las consultas. Por un lado, se ha observado que el \textit{tiempo de consulta} de Jena es peque�o, menor a medio segundo, cuando usa un \textit{modelo sin inferencia}. Mientras, el \textit{tiempo de consulta} de Jena para un \textit{modelo con inferencia} es mayor en comparaci�n con el tiempo del \textit{modelo sin inferencia}.

%%%% Hip�tesis es una soluci�n para un problema dado.
\section{Hip�tesis}
Con base en estas observaciones, nuestras dos hip�tesis de experimentaci�n  son �stas:
\begin{enumerate}
\item \textit{El \textbf{triplestore Jena} obtiene \textbf{mejores resultados} cuando utiliza nuestros modelos con inferencia}.
\item \textit{El \textbf{tiempo de consulta} de Jena es \textbf{mayor} para nuestros \textit{modelos con inferencia} en comparaci�n con nuestros \textbf{modelos sin inferencia}}.
\end{enumerate}

%%%% M�todo consiste en: la elecci�n de los sujetos para confirmar la muestra, el procedimiento a seguir para �stos, las variables consideradas: dependiente, independiente y auxiliares.

%%Elaborar un experimento que ponga a prueba una hip�tesis
\section{Experimentaci�n}
Esta experimentaci�n consiste en la realizaci�n de dos actividades para probar nuestras hip�tesis de experimentaci�n. \textit{La \textbf{primera actividad} es \textbf{evaluar la calidad de los resultados} de Jena para los modelos con y sin inferencia}. Esta evaluaci�n consiste en estas etapas: 1) establecer una serie de consultas para interrogar nuestros modelos, 2) encontrar manualmente cu�ntos y cu�les recursos responden las consultas, 3) ejecutar las consultas con el motor ARQ de Jena y 4) comparar los recursos dados por Jena con las respuestas manuales.

\textit{La \textbf{segunda actividad} consiste en medir los \textbf{tiempos promedio de procesamiento} de Jena para las consultas de nuestra primera actividad}. La finalidad de esta segunda actividad es comparar los \textit{tiempos de consulta} para un modelo con inferencia y otro que no emplea �sta. En ambos modelos se eval�a el tiempo desde \textit{la ejecuci�n de Jena para una consulta a un modelo (con o sin inferencia)} hasta \textit{la presentaci�n de los resultados en pantalla}.

%%La determinaci�n del tiempo para modelos sin inferencia, consiste en medir los tiempos de: 1) \textit{ejecuci�n de la consulta en en el modelo} y 2) \textit{recuperaci�n de la informaci�n}. De la misma manera, la medici�n de tiempos para un modelo con inferencia es parecida a la medici�n en un modelos sin inferencia. La excepci�n es que en un modelo con inferencia, se toman en cuenta los tiempos para: \textit{el proceso de inferencia en el modelo} y \textit{la ejecuci�n de la consulta al modelo inferido}.

En esta tesis, el proceso de \textit{integraci�n sem�ntica} est� asociado a dos \textit{casos de uso} (cartograf�a de competencias y b�squeda de recursos digitales). Ahora bien, nuestra experimentaci�n consiste en probar la \textit{calidad de los resultados} y el \textit{tiempo de procesamiento} para la \textit{integraci�n sem�ntica de recursos} en la \textit{memoria corporativa} del �rea de Redes y Telecomunicaciones. Por esta raz�n, las dos actividades de nuestra experimentaci�n deben ser aplicadas a nuestros dos \textit{casos de uso}.

%%Elecci�n de los sujetos para confirmar la muestra
\section{Sujetos de experimentaci�n}
Los sujetos de nuestra experimentaci�n son un conjunto de personas, documentos y archivos multimedia que son generados artificialmente. Esta \textit{generaci�n artificial} consiste en el uso de scripts para: 1) \textit{asignar un identificador URI para un conjunto de \textbf{recursos de informaci�n ficticios}} y 2) \textit{generar tripletas RDF para estos recursos con base en las \textbf{propiedades} y \textbf{clases} de nuestras ontolog�as, as� como \textbf{datos aleatorios}}. 

Un script genera un conjunto de declaraciones para los recursos persona. Mientras, otro script genera las declaraciones para los documentos y archivo multimedia. El algoritmo \ref{alg:fsg} presenta el funcionamiento general de ambos scripts para la generaci�n y almacenamiento de tripletas RDF.

\begin{algorithm}
\SetKwData{Sigma}{$\sigma_i$}
\SetKwData{Model}{$modelo_{rdf}$}
$N\leftarrow$  n�mero de \textit{recursos ficticios de informaci�n} a describir\;
\Model$\leftarrow$Crear un modelo rdf\;
\For{$i\leftarrow 1$ \KwTo $N$}{
	\Sigma$\leftarrow$Crear el recurso $i$ y establecer un identificador URI para �ste\;
	Elaborar los valores para cada caracter�stica significativa de este recurso (\Sigma)\; \label{lin:values}
	Escribir las aserciones, concatenando el URI del recurso (\Sigma), las propiedades de la ontolog�a y los valores del paso \ref{lin:values}\;
}
Guardar el \Model en un archivo con extensi�n ``\textit{rdf}'' y sintaxis de serializaci�n \textit{Turtle}\;
\caption{Funcionamiento b�sico de scripts para la generaci�n de tripletas artificialmente.}
\label{alg:fsg} 
\end{algorithm}

El ap�ndice \ref{aped:AlgDS} presenta los dos algoritmos con el funcionamiento detallado de los scripts. Un algoritmo para los datos simulados de los recursos persona y el otro para los recursos digitales.

La finalidad del uso de \textit{informaci�n simulada} es tener r�pidamente un volumen grande de datos en nuestros ABox. La \textit{cantidad de informaci�n} en estos ABox debe ser realista con respecto al �rea de Redes y Telecomunicaciones (RyT). Ya que al tener informaci�n realista, nuestra experimentaci�n se ajusta a la cantidad de datos que esperamos manejar en la integraci�n sem�ntica. Otra raz�n del uso de informaci�n simulada es ver si Jena soporta esta escala realista de datos (seg�n los profesores del �rea RyT).

Las cantidades de \textit{recursos persona} son 60 recursos artificiales y 13 recursos reales, dando un total de 73 personas. Mientras, las cantidades para los \textit{recursos digitales} son 16 recursos reales y 1314 recurso simulados, un total de 1330 recursos digitales.

En nuestra experimentaci�n, algunos \textit{recursos persona} tienen la declaraci�n que los asignan expl�citamente a una de estas clases: \textit{Estudiante (sirp:Student)}, \textit{Empleado (sird:Emplyee)} y \textit{Profesor (sirp:Teacher)}. Otros recursos persona carecen de esta asignaci�n, pero con inferencia �stos pueden clasificarse en una o varias clases de la \textit{ontolog�a cartograf�a de competencias}.

En concreto, se tienen estas \textit{cantidades de recursos} por clase: 51 recursos son profesionistas y 23 son \textit{estudiantes}. Los 51 recursos persona mediante inferencia son asignados a la clase \textit{Profesionista (sirp:Professional)}. De estos 51 profesionistas se tiene que 19 son profesores y 9 son empleados. Por otro lado, de los 23 estudiantes se tiene que 9 recursos est�n asignados a la clase \textit{Estudiante} y los 14 restantes por inferencia son asignados a la clase \textit{Estudiante}. Existen 13 recursos persona que mediante inferencia se clasifican en la clase \textit{Investigador (sirp:Researcher)}.

La Tabla \ref{tab:noRP} muestra las cantidades de \textit{recursos persona} por clases de la \textit{ontolog�a cartograf�a de competencias}. La \textit{primera columna} presenta el nombre de las clases, la \textit{segunda} el n�mero de recursos que tienen la declaraci�n que los asigna expl�citamente a una clase y la \textit{tercera columna} el n�mero de recursos que por inferencia tienen la declaraci�n para asignarlos a una clase.

\begin{table}[!htb]
\renewcommand{\arraystretch}{1.2}
\centering
\begin{tabular}{| >{\centering\arraybackslash}m{2in} | >{\centering\arraybackslash}m{1.5in} | >{\centering\arraybackslash}m{1.5in} | }
\hline 
\multirow{2}{*}{\textbf{Clase}} & \multicolumn{2}{c|}{\textbf{N�mero de Recursos}} \\
\cline{2-3} 
 & \textbf{Asignaci�n expl�cita} & \textbf{Asignaci�n expl�cita y con inferencia}\\
\hline 
\hline
Persona & 0 & 73\\
\hline
Investigador & 0 & 13\\
\hline
Profesionista & 0 & 51\\
\hline
Estudiante & 11 & 23\\
\hline
Profesor & 19 & 19\\
\hline
Empleado & 9 & 9\\
\hline
\end{tabular}
\caption{N�mero de recursos persona por clase.}
\label{tab:noRP}
\end{table}

De la misma manera que los \textit{recursos persona}, algunos \textit{recursos digitales} tienen la declaraci�n que los asignan expl�citamente a una clase. Otros recursos carecen de esta asignaci�n, pero con inferencia �stos pueden clasificarse en una o varias clases de la \textit{ontolog�a b�squeda de recursos digitales}.

Los 1330 recursos digitales de nuestra experimentaci�n se clasifican en: 156 art�culos, 366 libros, 34 reportes t�cnicos, 146 p�ginas web, 73 tesis, 42 videos, 42 audios, 77 im�genes y 112 presentaciones. De los 156 art�culos, 89 recursos tienen la declaraci�n expl�cita a la clase \textit{Art�culo (sird:Paper)} y los restantes 67 mediante inferencia tienen la declaraci�n a esta clase. En los libros, 185 recursos tienen la declaraci�n explicita y 181 recursos mediante inferencia tienen la declaraci�n a la clase \textit{Libro (sird:Book)}. De la misma manera, 79 recursos tienen la declaraci�n explicita a la clase \textit{P�gina Web (sird:PageWeb)} y 31 recursos a la clase \textit{Tesis (sird:Thesis)}. Mientras, 67 recursos mediante inferencia pertenecen a la clase  \textit{P�gina Web} y 42 a la clase \textit{Tesis}. Por ultimo, los 1330 recursos se clasifican en 815 \textit{Documnetos (sird:Document)} y 515 \textit{Multimedia (sird:Multimedia)}. Las declaraciones a estas dos clases se obtienen mediante inferencia.

La Tabla \ref{tab:noRD} presenta las cantidades de recursos digitales por clases de la \textit{ontolog�a recursos digitales}. Esta Tabla \ref{tab:noRD} presenta la misma estructura de la Tabla \ref{tab:noRP} 

\begin{table}[!htb]
\renewcommand{\arraystretch}{1.1}
\centering
\begin{tabular}{| >{\centering\arraybackslash}m{2in} | >{\centering\arraybackslash}m{1.5in} | >{\centering\arraybackslash}m{1.5in} | }
\hline 
\multirow{2}{*}{\textbf{Clase}} & \multicolumn{2}{c|}{\textbf{N�mero de Recursos}} \\
\cline{2-3} 
 & \textbf{Asignaci�n expl�cita} & \textbf{Asignaci�n expl�cita y con inferencia}\\
\hline 
\hline
Recurso Digital & 0 & 1330\\
\hline
Documento & 0 & 815\\
\hline
Art�culo & 89 & 156\\
\hline
Reporte T�cnico & 34 & 34\\
\hline
P�gina Web & 79 & 146\\
\hline
Tesis & 31 & 73\\
\hline
Libro & 185 & 366\\
\hline
Multimedia & 0 & 515\\
\hline
Presentaci�n & 112 & 112\\
\hline
Audio & 42 & 42\\
\hline
V�deo & 42 & 42\\
\hline
Imagen & 77 & 77\\
\hline
\end{tabular}
\caption{N�mero de recursos digitales por clase.}
\label{tab:noRD}
\end{table}

\section{Metodolog�a}
Esta experimentaci�n se ha dividido en dos actividades. La primera actividad consiste en la \textit{evaluaci�n de la calidad de los resultados} que proporciona el triplestore Jena. Mientras, la segunda actividad consiste en medir los \textit{tiempos promedios} que toma Jena para: consultar los modelos con/sin inferencia y mostrar los resultados al usuario.

El \textit{primer paso} en la \textit{evaluaci�n de la calidad de resultados} es identificar un conjunto b�sico de consultas para interrogar las ontolog�as \textit{cartograf�a de competencias} y \textit{b�squeda de recursos digitales}. En la Secci�n \ref{sec:byrKrec} se hizo un an�lisis e identificaci�n de preguntas base que posteriormente se transformaron a consultas SPARQL. Estas consultas de la Secci�n \ref{sec:byrKrec} son reutilizadas para la primer actividad de esta experimentaci�n.

El \textit{segundo paso} es encontrar cu�ntos y cu�les recursos son relevantes para responder las consultas de nuestra experimentaci�n. Para ello, se hace una \textit{b�squeda manual} exhaustiva de los recursos relevantes que responden las consultas de experimentaci�n. Esta b�squeda se hace sobre los recursos de nuestra memoria corporativa que est� guiada por los casos de uso. Para cada consulta, se anotan los identificadores URI de los \textit{recursos relevantes} y la cantidad de �stos.

La Tabla \ref{tab:qrynoRP} muestra las preguntas base para la \textit{cartograf�a de competencias}. En esta Tabla, la primera columna presenta el identificador para cada una de las diecinueve pregunta, la segunda columna enuncia la pregunta y la tercer columna presenta el n�mero de recursos que responden a �sta. De la misma manera, la Tabla \ref{tab:qrynoRD} presenta un identificador, las preguntas y cantidad de recursos que responden a �stas, para la \textit{b�squeda de recursos digitales}. En estas dos Tablas, no se presentan los identificadores URI de los recursos que responden a las preguntas. Porque algunas consultas tienen muchos recursos relevantes.

\begin{table}[!hp]
\renewcommand{\arraystretch}{1.4}
\centering
\begin{tabular}{>{\centering\arraybackslash}m{1in} >{\arraybackslash}m{3.5in} >{\centering\arraybackslash}m{1in}}
\hline 
Id. Consulta & Pregunta & No. de Recursos\\
\hline
\hline 
Q1.1 &  �Cu�les el nombre, correo, sitio web, g�nero y edad de las personas del �rea de RyT? & 73 \\
\hline
Q1.2 & �Cu�l es el nombre, sitio web y el lugar donde laboran las personas del RyT? & 73\\
\hline 
Q1.3 & �Qui�nes son mayores de 20 a�os y menores de 45 a�os? & 50\\
\hline 
Q1.4 & �Qui�nes son profesionistas del �rea de RyT? & 51\\
\hline 
Q1.5 & �Qui�nes trabajan en la Clark \& Parsia y son del sexo Masculino? & 3\\
\hline 
Q1.6 & �Qui�nes son estudiantes y leen en ingl�s? & 8\\
\hline 
Q1.7 & �Quienes hablan, leen y escriben en ingl�s? & 16\\
\hline 
Q1.8 & �Qu� estudiantes saben algo de ingl�s? & 6\\
\hline 
Q1.9 & �Qu� profesores tienen la capacidad de s�ntesis? & 2\\
\hline 
Q1.10 & �Qu� profesionistas tienen conocimiento en los temas de Web Sem�ntica? & 58\\
\hline
Q1.11 & �Qu� profesores tienen conocimientos en Sistemas Distribuidos? & 3\\
\hline 
Q1.12 & �Qui�nes tienen conocimiento en Java, OWL, RDF, Threads, C, OpenMP? & 1\\
\hline 
Q1.13 & �Qu� estudiantes tienen alg�n conocimiento en los subtemas de Sistemas Operativos? & 33\\
\hline 
Q1.14 & �Qui�nes trabajan en una Universidad? & 24\\
\hline 
Q1.15 & �Quienes laboran en la UAM y tienen alg�n conocimiento en  Web Sem�ntica? & 19\\
\hline 
Q1.16 & �Qu� personas tienen como asesor a Carolina Medina? & 2\\
\hline 
Q1.17 & �Qui�nes son los colegas de Ricardo Marcelin? & 8\\
\hline 
Q1.18 &  �Qui�nes conocen a Carolina Medina Ramirez? & 11\\
\hline 
Q1.19 & �Qu� personas son profesores-investigadores? & 9\\
\hline 
\end{tabular}
\caption{Preguntas y cantidad de personas que responden a �stas.}
\label{tab:qrynoRP}
\end{table}

\begin{table}[!hp]
\renewcommand{\arraystretch}{1.4}
\centering
\begin{tabular}{>{\centering\arraybackslash}m{1in} >{\arraybackslash}m{3.5in} >{\centering\arraybackslash}m{1in}}
\hline 
Id. Consulta & Pregunta & No. de Recursos\\
\hline
\hline 
Q2.1 & �Cu�les son los t�tulos, rutas, extensi�n, idioma de todos los recursos digitales de RyT? & 1330 \\
\hline
Q2.2 & �Cu�les libros tratan sobre algunos temas de Sistemas Distribuidos? & 103\\
\hline 
Q2.3 & �Qu� recursos fueron publicados por la UAM? & 18\\
\hline 
Q2.4 & �Qu� documentos son para dar un curso de Sistemas P2P? & 31\\
\hline 
Q2.5 & �Qu� recursos multimedia son mayores al a�o 2009? & 119\\
\hline 
Q2.6 & �Cu�les documentos tratan sobre Ontolog�as? & 30\\
\hline 
Q2.7 & �Qu� recursos fueron publicados en una Revista cient�fica? & 156\\
\hline 
Q2.8 & �Qu� recursos tienen en su contenido las palabras "linked data"? & 159\\
\hline 
Q2.9 & �Cu�les documentos en ingl�s y mayores al a�o 2000 son de autor�a de Erik Alarc�n Zamora? & 2\\
\hline 
Q2.10 & �Cu�les la tesis de Samuel Hern�ndez Maza? & 4\\
\hline 
\end{tabular}
\caption{Preguntas y cantidad de recursos digitales que responden a �stas.}
\label{tab:qrynoRD}
\end{table}

El \textit{tercer paso} es emplear \textit{Jena} para ejecutar las \textit{consultas SPARQL} a un modelo sin inferencia y otro con inferencia. La finalidad de �sto es ver la calidad de resultados y la cantidad de �stos que al emplear un modelo con inferencia y otro sin inferencia. De esta manera, podemos ver el impacto de la inferencia para la consulta de informaci�n en una ontolog�a.

En esta \textit{ejecuci�n de consultas}, se emplea un script para consultar un \textit{modelo sem�ntico} y proporcionar los resultados qu est�n asociados a una consulta dada. El funcionamiento de este script se presenta en el Algoritmo \ref{alg:srinf}. En este algoritmo, los \textbf{\textit{par�metros de entrada}} son: 1) \textit{la consulta SPARQL}, 2) \textit{la ontolog�a} (\textit{Personas o RecDigi}), 3) \textit{inferencia} (\textit{0 sin inferencia} y \textit{1 con inferencia}). Mientras, los valores de salida son: 1) \textit{la impresi�n en pantalla de los identificadores URI que responden la consulta} y 2) \textit{la impresi�n de la cantidad de recursos respuesta}.

\begin{algorithm}
\SetKwData{Model}{$modelo_{semantico}$}
\KwIn{Consulta SPARQL: $query$, Nombre del modelo: $model$, Uso de inferencia: $inference$}
\KwOut{Resultados: $\Pi_{query}$, N�mero de resultados: $noRes$}
\BlankLine
$noRes \leftarrow 0$\;
$\Pi_{query} \leftarrow \{ \}$\;
\uIf{$model \equiv$ `Personas'}{
	Cargar en memoria el ABox y TBox de la ontolog�a cartograf�a de competencias\;
}
\ElseIf{$model \equiv$ `RecDigi'}{
	Cargar en memoria el ABox y TBox de la ontolog�a b�squeda de recursos digitales\;
}
\uIf{$model \equiv 0$}{
	\Model $\leftarrow$ modelo ABox y TBox\;
}
\ElseIf{$model \equiv 1$}{
	Inferir en memoria sobre el ABox y Tbox\;
	\Model $\leftarrow$ modelo inferido\;
}
Cargar en memoria consulta SPARQL ($query$)\;
Ejecutar consulta ($query$) en el modelo (\Model)\;
$Pi_{query} \leftarrow$ Recuperar resultados de consulta\tcc*{$Pi_{query} = \{ \pi_1, \dots, \pi_n \}$}
\ForAll{elements of $Pi_{query}$}{
	Imprimir $\pi_k$ en pantalla\;
	$noRes \leftarrow noRes + 1$\;
}
Imprimir $noRes$ en pantalla\;
\caption{Algoritmo para la evaluaci�n de la calidad de resultados}
\label{alg:srinf} 
\end{algorithm}

El \textit{cuarto paso} consiste en ver cuantos resultados dados por Jena, responden las consulta de las Tablas \ref{tab:qrynoRP} y \ref{tab:qrynoRD}. La finalidad de �sto, es \textit{comparar} la calidad de los resultados dados por Jena, para un modelo con inferencia y otro sin inferencia. Este paso se hace en dos fases y para cada consulta de las Tablas \ref{tab:qrynoRP} y \ref{tab:qrynoRD}. �stos dos pasos son: 1) encontrar los recursos relevantes y no relevantes de lasa partir de las respuestas del script, y 2) usar (\textit{m�tricas de recuperaci�n de la informaci�n}) para comparar la calidad de los resultados con y sin inferencia.

La \textit{fase uno} consiste en comparar y anotar las cantidades de recursos relevantes e irrelevantes para cada consulta. En esta fase se considera el modelo de uso: con inferencia o sin �sta. Esta comparaci�n y anotaci�n se hizo de forma manual. El primer criterio de comparaci�n es ver la cantidad de recursos de las Tablas \ref{tab:qrynoRP} y \ref{tab:qrynoRD} con el n�mero de respuestas que proporciona el script. Posteriormente, se hace la comparaci�n identificador a identificador de los recursos que son proporcionados por el script con los recursos que realmente responden la consulta.

La segunda fase, consiste en emplear las siguientes m�tricas de \textit{recuperaci�n de la informaci�n}: \textbf{\textit{exhaustividad}} y \textbf{\textit{precisi�n}}.

La \textbf{\textit{exhaustividad (recall)}} \cite{Nasraoui} es ``\textit{la cantidad de elementos relevantes que han sido recuperados, entre la cantidad de elementos relevantes en la base de conocimientos}'' \cite{RecYPres}. �sto significa, la cantidad de \textit{elementos relevantes recuperados} es el \textit{n�mero de recursos relevantes dados por el script para una consulta}. Mientras, el n�mero total de \textit{elementos relevantes en la base de conocimiento} es la \textit{cantidad de resultados de nuestra b�squeda manual que est�n en las Tablas} \ref{tab:qrynoRP} y \ref{tab:qrynoRD}. �sta es la ecuaci�n asociada a la exhaustividad.

$$
	Exhaustividad = \frac{N\acute{u}mero\: de\: recursos\: relevantes\: que\: han\: sido\: recuperados}{N\acute{u}mero\: de\: recursos\: relevantes\: en\: la\: base\: de\: conocimientos} * 0.1
$$

La \textbf{\textit{precisi�n (Precision)}} \cite{Nasraoui} es la ``\textit{cantidad de elementos recuperados que son relevantes, entre el total de elementos recuperados}'' \cite{RecYPres}. En nuestra experimentaci�n. la cantidad de \textit{elementos relevantes recuperados} es el \textit{n�mero de recursos relevantes dados por el script para una consulta}. Mientras, \textit{el total de elementos recuperados en la base de conocimiento} es la cantidad de \textit{recursos dados por el script, sin el an�lisis de aquellos que son relevantes}.

$$
	Precisi�n = \frac{N\acute{u}mero\: de\: elementos\: relevantes\: que\: han\: sido\: recuperados}{Total\: de\:elementos\: recuperados\: en\: la\: base\: de\: conocimientos} * 0.1
$$

----> Aqu� voy

Sigue experimentaci�n de los tiempos de procesamiento
hay que hablar de la manera de llevar a cabo esto
y del uso de un script para calcular estos tiempos, basado en el algoritmo 2.

\section{Resultados}
La Tabla \ref{tab:rrRP} muestra los recursos relevantes y no relevantes que se recuperaron para las preguntas de la Tabla \ref{tab:qrynoRP}. La primer columna presenta el identificador de la consulta. La segunda y tercera columna muestran los recursos no relevantes y relevantes respectivamente cuando se emplea un modelo sin inferencia. La cuarta y quita presentan los recursos no relevantes y relevantes para un modelo con inferencia.

\begin{table}[!htb]
\renewcommand{\arraystretch}{1.2}
\centering
\begin{tabular}{| >{\centering\arraybackslash}m{1in} | >{\centering\arraybackslash}m{1in} | >{\centering\arraybackslash}m{1in} | >{\centering\arraybackslash}m{1in} | >{\centering\arraybackslash}m{1in} | }
\hline 
\multirow{2}{*}{Id. Consulta} & \multicolumn{2}{c|}{Modelo sin inferencia} & \multicolumn{2}{c|}{Modelo con inferencia}\\
\cline{2-5} 
 & Recursos no relevantes & Recursos relevantes & Recursos no relevantes & Recursos relevantes\\
\hline 
\hline
Q1 & 12 & 1330/1330 & 138 & 1330/1330\\
\hline
Q2 & 10 & 0/103 & 194 & 103/103\\
\hline
Q3 & 8 & 18/18 & 406 & 18/18\\
\hline
Q4 & 28 & 15/31 & 129 & 31/31\\
\hline
Q5 & 7 & 66/119 & 157 & 119/119\\
\hline
Q6 & 9 & 15/30 & 4016 & 30/30\\
\hline
Q7 & 12 & 156/156 & 3520 & 156/156\\
\hline
Q8 & 16 & 159/159 & 3472 & 159/159\\
\hline
Q9 & 42 & 0/2 & 3451 & 2/2\\
\hline
Q10 & 13 & 3/4 & 3312 & 4/4\\
\hline
\end{tabular}
\caption{Recursos relevantes y no relevantes asociados a las preguntas de la cartograf�a de competencias.}
\label{tab:rrRP}
\end{table}

De la misma manera, la Tabla \ref{tab:rrRD} presenta los recursos relevantes y no relevantes cuando se consulta el modelo sin inferencia y con inferencia de los recursos digitales. En esta Tabla, las consultas est�n asociadas a las preguntas de la Tabla \ref{tab:qrynoRD}. La estructura de la Tabla \ref{tab:rrRD} es igual a la Tabla \ref{tab:rrRP}.

\begin{table}[!htb]
\renewcommand{\arraystretch}{1.2}
\centering
\begin{tabular}{| >{\centering\arraybackslash}m{1in} | >{\centering\arraybackslash}m{1in} | >{\centering\arraybackslash}m{1in} | >{\centering\arraybackslash}m{1in} | >{\centering\arraybackslash}m{1in} | }
\hline 
\multirow{2}{*}{Id. Consulta} & \multicolumn{2}{c|}{Modelo sin inferencia} & \multicolumn{2}{c|}{Modelo con inferencia}\\
\cline{2-5} 
 & Recursos no relevantes & Recursos relevantes & Recursos no relevantes & Recursos relevantes\\
\hline 
\hline
Q2.1 & 0 & 1330 & 0 & 1330\\
\hline
Q2.2 & 0 & 103 & 0 & 103\\
\hline
Q2.3 & 0 & 18 & 0 & 18\\
\hline
Q2.4 & 0 & 15 & 0 & 31\\
\hline
Q2.5 & 0 & 66 & 0 & 119\\
\hline
Q2.6 & 0 & 15 & 0 & 30\\
\hline
Q2.7 & 0 & 156 & 0 & 156\\
\hline
Q2.8 & 0 & 159 & 0 & 159\\
\hline
Q2.9 & 0 & 0 & 0 & 2\\
\hline
Q2.10 & 0 & 3 & 0 & 4\\
\hline
\end{tabular}
\caption{Recursos relevantes y no relevantes asociados a las preguntas de la b�squeda de recursos digitales.}
\label{tab:rrRD}
\end{table}

----> Aqu� voy **


Para evaluar este costo en tiempo, calculamos el tiempo promedio que tarda cada consulta en responderse. Espec�ficamente, nosotros tomamos el tiempo desde que se consulta la informaci�n del modelo hasta que se presentan los resultados en pantalla. Esta operaci�n la repetimos veinte veces por consulta, de esta manera, sacamos el tiempo promedio por consulta de nuestro modelo. 

Esta medici�n del tiempo promedio se hace a partir de un script en Java. Este script se ejecuta en una computadora que tiene un procesador Intel core I7 a 2.3GHz con 8Gb en ram y 8 n�cleos de procesamiento.  Las siguientes dos tablas presentan los tiempos de respuesta para nuestro conjunto de consultas. En la tabla 7 se muestran los valores de las consultas para el caso de uso Cartograf�a de Competencias, en tanto, la tabla 8 muestra los tiempos para el caso de uso Recursos Digitales. En ambas tablas, se contemplan los tiempos de respuesta y el n�mero de resultados para las consultas que emplean s�lo con ABox, as� como para las consultas que utilizan ABox, TBox y un Razonador (ATR).

--------------
Para los recursos persona en esta experimentaci�n se tienen 1750 triples. Mientras, para los recursos digitales se tienen 20429 triples. De esta manera, el n�mero total de triples en esta experimentaci�n son 22179.

Por otro lado, bas�ndonos en las consultas b�sicas para nuestros modelos, el an�lisis de los triples escritos manualmente y el uso de variables contador en los dos scripts, se identificaron para cada consulta el n�mero de recursos que responden a la misma. En la tabla 3 se listan solamente 10 de nuestras 28 preguntas y el n�mero de resultados de las mismas.
--------------
%%Procedimiento a seguir para �stos

Ahora bien, la finalidad de este listado es tomar los tiempos promedios y el n�mero de respuestas, cu�ndo las consultas SPARQL se hacen con el motor sin razonador y con razonador. Con la finalidad de averiguar la precisi�n [5], as� como el desempe�o del motor de SPARQL y el razonador de Jena.

Las dos variables de experimentaci�n son tiempo y n�mero de resultados.  La variable tiempo nos permite sacar el tiempo promedio que toma una consulta en ser ejecutada K veces. Mientras la variable n�mero de resultados almacena la cantidad de recursos que fueron recuperados para una consulta dada. 
	
Para encontrar los valores de estas variables, nosotros empleamos un programa en Java. Este programa se ha dise�ado para ejecutar �nicamente una consulta e imprimir los valores de las variables en pantalla. El programa permite elegir al usuario el modelo RDF a consultar. Si es modelo RDF con triples expl�citos, entonces solo cargan los triples (ABox) de los recursos. Por el contrario, si el modelo es con triples inferidos, entonces se cargan los triples (ABox), los axiomas (TBox) y se hace inferencia con el razonador de Jena.
Este programa se ejecut� en una computadora con las siguientes capacidades: Procesador Intel Core I7 a 2.3GHz con 8Gb en RAM y 8 n�cleos de procesamiento, y los valores resultantes de las variables se muestran en la Tabla 2.


%%Variables consideradas: dependiente, independiente y auxiliares


%%%% Resultados consiste en describir: cuales son las relaciones observadas entre las variables y la descripci�n gr�fica de estos resultados.

---------------------------
En esta tabla se tienen dos columnas compuestas, la columna (titulada conocimiento expl�cito) muestra los valores de las consultas que emplearon �nicamente el modelo con triples expl�citos. Mientras, la segunda columna (titulada conocimiento inferido) muestra los valores de las consultas que emplearon el modelo con triples expl�citos, axiomas y un razonador. Para las columnas sencillas, la columna ?N�mero de resultados? muestra el n�mero de recursos recuperados del total esperado para la consulta dada, mientras la columna ?Tiempo promedio? muestra el tiempo promedio de consulta en milisegundos.

En algunos casos, la consulta al conocimiento expl�cito recupera todos los recursos esperados y los tiempos de respuesta son peque�os (no pasan del segundo). Sin embargo, en otras consultas se descartaron varios recursos que si responden la consulta. Esto se debe a que algunos recursos carecen un determinado triple. Por otro lado, las consultas al grafo con triples inferidos permitieron recuperar todos los recursos esperados, porque mediante los axiomas y el razonador se deducen triples (materializaron) que ser�n considerados por el motor de b�squeda . Sin embargo, el tiempo de procesamiento es mucho mayor porque se invierte tiempo en procesar e inferir relaciones en el grafo RDF.

Todo tiene un costo, cuando el razonador materializa los triples en el modelo, �ste consume tiempo en procesamiento y al hacer una consulta, el motor debe comparar m�s aserciones. El desarrollador no debe abusar de la axiomatizaci�n, en algunos casos cuando la consulta es sobre  hechos expl�citos, no es necesario el uso del razonador, basta con escribir y hacer la consulta sobre el conocimiento expl�cito.
---------------------------

%%%% Conclusiones, se obtienen a partir del estudio de los resultados. �stas son el resultado de la discusi�n de las premisas.

La primer conclusi�n afirma que el performance mejora cuando se usan axiomas. Esta afirmaci�n resulta cierta, porque un razonador deduce una relaci�n que vincula directamente dos objetos. An�logamente, resulta m�s r�pido ir por el camino directo que por una serie de rutas hasta el mismo objeto.

La segunda conclusi�n tiene que ver con el n�mero de resultados. Si bien, las aserciones establecen un conjunto directo y est�tico de enlaces entre los distintos recursos de nuestro modelo. En muchas ocasiones, al momento de construir una consulta SPARQL no se contemplan algunos de estos enlaces, inclusive en otros casos, estos enlaces no est�n escritos expl�citamente. Por consiguiente, mucho recursos no se contemplan como respuesta para una consulta. En contraste, los axiomas, aserciones y un razonador, establecen estos enlaces entre recursos de forma expl�cita en memoria, de esta manera, las consultas respondan m�s resultados que no se hab�an contemplado.

%http://hal-lirmm.ccsd.cnrs.fr/docs/00/63/97/05/PDF/A_Flexible_System_for_Ontology_Matching.pdf


Funcionamiento de los scripts 
%https://docs.google.com/document/d/1c-JfQYGg7O2xMTNh5gCmdtiDf2vmrsreQfpLU9na4AQ/edit

\section{Escenarios de experimentaci�n}
Alg�n texto...


\begin{table}[!htb]
\renewcommand{\arraystretch}{1.2}
\centering
\begin{tabular}{| >{\centering\arraybackslash}m{1in} | >{\centering\arraybackslash}m{1in} | >{\centering\arraybackslash}m{1in} | >{\centering\arraybackslash}m{1in} | >{\centering\arraybackslash}m{1in} | }
\hline 
\multirow{2}{*}{Id. Consulta} & \multicolumn{2}{c|}{Modelo (ABox)} & \multicolumn{2}{c|}{Modelo (Razonador+Ontolog�a)}\\
\cline{2-5} 
 & Tiempo promedio (ms) & No. Recursos  & Tiempo promedio (ms) & No. Recursos\\
\hline 
\hline
Q1 & 12 & 1330/1330 & 138 & 1330/1330\\
\hline
Q2 & 10 & 0/103 & 194 & 103/103\\
\hline
Q3 & 8 & 18/18 & 406 & 18/18\\
\hline
Q4 & 28 & 15/31 & 129 & 31/31\\
\hline
Q5 & 7 & 66/119 & 157 & 119/119\\
\hline
Q6 & 9 & 15/30 & 4016 & 30/30\\
\hline
Q7 & 12 & 156/156 & 3520 & 156/156\\
\hline
Q8 & 16 & 159/159 & 3472 & 159/159\\
\hline
Q9 & 42 & 0/2 & 3451 & 2/2\\
\hline
Q10 & 13 & 3/4 & 3312 & 4/4\\
\hline
\end{tabular}
\end{table}

\section{Experimentaci�n}
M�s texto...

\section{Resultados}
M�s texto...

%%%%%%%%%%%%%%%%%%%%%%%%%%%%%%%%%%%%%%%
La integraci�n sem�ntica consiste en la b�squeda y recuperaci�n de la informaci�n de los recursos. En este punto, es importante evaluar esa recuperaci�n y medir los tiempos de procesamiento para un modelo con razonamiento en Jena. Para evaluar la recuperaci�n de los recursos consiste en enumerar los recursos que responden a una consulta. Mientras, el tiempo promedio de procesamiento es la media en tiempo (milisegundos) que tarda una consulta en ejecutarse.
	En esta experimentaci�n hay 73 personas y 1330 recursos digitales. Nosotros escribimos manualmente los triples de 11 personas que est�n adscritas al Departamento de Ingenier�a El�ctrica de la Universidad Aut�noma Metropolitana y los triples de las otras 60 fueron generados artificialmente. Por otro lado, nosotros escribimos manualmente los triples de 16 recursos digitales, mientras que los triples de los otros 1314 fueron generados artificialmente.
	Las 73 personas se clasifican en: Profesor, Estudiante, Empleado, Profesionista, Investigador y Persona. Mientras los 1330 recursos digitales se clasifican en: Art�culos, Reportes T�cnicos, P�ginas Web, Tesis, Libros, Presentaciones, Im�genes, Audios, Videos, Documento, Multimedia y Recurso Digital. En particular, para cada clase de Persona y Recursos Digital se tienen las siguientes cantidades:
Para los recursos persona en esta experimentaci�n se tienen 1750 triples. Mientras, para los recursos digitales se tienen 20429 triples. De esta manera, el n�mero total de triples en esta experimentaci�n son 22179.
	Por otro lado, bas�ndonos en las consultas b�sicas para nuestros modelos, el an�lisis de los triples escritos manualmente y el uso de variables contador en los dos scripts, se identificaron para cada consulta el n�mero de recursos que responden a la misma. En la tabla 3 se listan solamente 10 de nuestras 28 preguntas y el n�mero de resultados de las mismas.
	Ahora bien, la finalidad de este listado es tomar los tiempos promedios y el n�mero de respuestas, cu�ndo las consultas SPARQL se hacen con el motor sin razonador y con razonador. Con la finalidad de averiguar la precisi�n [5], as� como el desempe�o del motor de SPARQL y el razonador de Jena.
	Las dos variables de experimentaci�n son tiempo y n�mero de resultados.  La variable tiempo nos permite sacar el tiempo promedio que toma una consulta en ser ejecutada K veces. Mientras la variable n�mero de resultados almacena la cantidad de recursos que fueron recuperados para una consulta dada. 
	Para encontrar los valores de estas variables, nosotros empleamos un programa en Java. Este programa se ha dise�ado para ejecutar �nicamente una consulta e imprimir los valores de las variables en pantalla. El programa permite elegir al usuario el modelo RDF a consultar. Si es modelo RDF con triples expl�citos, entonces solo cargan los triples (ABox) de los recursos. Por el contrario, si el modelo es con triples inferidos, entonces se cargan los triples (ABox), los axiomas (TBox) y se hace inferencia con el razonador de Jena.
Este programa se ejecut� en una computadora con las siguientes capacidades: Procesador Intel Core I7 a 2.3GHz con 8Gb en RAM y 8 n�cleos de procesamiento, y los valores resultantes de las variables se muestran en la Tabla 2.

Interpretaci�n de resultados
	En esta tabla se tienen dos columnas compuestas, la columna (titulada conocimiento expl�cito) muestra los valores de las consultas que emplearon �nicamente el modelo con triples expl�citos. Mientras, la segunda columna (titulada conocimiento inferido) muestra los valores de las consultas que emplearon el modelo con triples expl�citos, axiomas y un razonador. Para las columnas sencillas, la columna ?N�mero de resultados? muestra el n�mero de recursos recuperados del total esperado para la consulta dada, mientras la columna ?Tiempo promedio? muestra el tiempo promedio de consulta en milisegundos.
	En algunos casos, la consulta al conocimiento expl�cito recupera todos los recursos esperados y los tiempos de respuesta son peque�os (no pasan del segundo). Sin embargo, en otras consultas se descartaron varios recursos que si responden la consulta. Esto se debe a que algunos recursos carecen un determinado triple. Por otro lado, las consultas al grafo con triples inferidos permitieron recuperar todos los recursos esperados, porque mediante los axiomas y el razonador se deducen triples (materializaron) que ser�n considerados por el motor de b�squeda . Sin embargo, el tiempo de procesamiento es mucho mayor porque se invierte tiempo en procesar e inferir relaciones en el grafo RDF.
Todo tiene un costo, cuando el razonador materializa los triples en el modelo, �ste consume tiempo en procesamiento y al hacer una consulta, el motor debe comparar m�s aserciones. El desarrollador no debe abusar de la axiomatizaci�n, en algunos casos cuando la consulta es sobre  hechos expl�citos, no es necesario el uso del razonador, basta con escribir y hacer la consulta sobre el conocimiento expl�cito.

%%%%%%%%%%%%%%%%%%%%%%%%%%%%%%%%%%%%%%%
Documento importante
%https://docs.google.com/document/d/1rv-RDqdSB3Sd_zd7WANHDRXqEa_UHgV0WjC0DupYK4M/edit

\chapter{Conclusiones y Trabajo Futuro}
\label{cap:concl}
Aunque los buscadores actuales entregan un conjunto de resultados en poco tiempo. Muchos de �stos no satisfacen la pregunta dada por un usuario. En cambio, al hacer una b�squeda basada en la sem�ntica de los recursos. Los resultados entregados ser�n m�s significativos para un usuario. Nuestra propuesta se basa en �sta idea. As� como el uso de conceptos y est�ndares de la Web Sem�ntica. Un razonador para una interrogaci�n m�s inteligente. En donde el dominio para nuestra propuesta es el de Redes y Telecomunicaciones.  Los recursos que sean devueltos por nuestra propuesta. Los vamos a evaluar con base en la opini�n de los usuarios del dominio. As� como los valores proporcionados por las m�tricas de la Recuperaci�n de la Informaci�n. Aunque nuestra propuesta es para el dominio RyT. El objetivo a largo plazo de nuestra propuesta se implemente en otros dominios. 

% *************** Apendix ***************
\begin{appendices}
\chapter{C�digos interfaz de Usuario}
\label{aped:A}

\chapter{Transformaci�n de preguntas a consultas SPARQL}
\label{aped:TrPC}

\begin{algorithm}
PREFIX sirp: <http://arte.izt.uam.mx/ontologies/personRyT.owl\#> \\
PREFIX redes: <http://mcyti.izt.uam.mx/arios/odaryt.owl\#> \\
PREFIX rdf: <http://www.w3.org/1999/02/22-rdf-syntax-ns\#> \\
\BlankLine
SELECT ?name ?mail ?ws ?gender ?age \\
WHERE \{ \\
\Indp ?x	sirp:has-name ?name\;
			sirp:has-email ?mail\;
			sirp:has-webSite ?ws\;
			sirp:has-gender ?gender. \\
			OPTIONAL \{?x sirp:has-age ?age\} \\
\Indm \}
\label{alg:spa1p1}
\caption{Consulta SPARQL asociada a la pregunta \textit{Q1.1} �Cu�les son los nombres, correos, sitios web, g�neros y edades de las personas del �rea de RyT.}
\end{algorithm}

%\begin{algorithm}
%PREFIX sirp: <http://arte.izt.uam.mx/ontologies/personRyT.owl\#> \\
%PREFIX redes: <http://mcyti.izt.uam.mx/arios/odaryt.owl\#> \\
%PREFIX rdf: <http://www.w3.org/1999/02/22-rdf-syntax-ns\#> \\
%SELECT DISTINCT  ?name ?ws ?work  \\
%WHERE \{ \\
%\Indp ?x 	sirp:has-name ?name\;
%			sirp:has-webSite ?ws. \\
%	\{?x sirp:worksIn ?work.\} UNION \\
%	\{?x sirp:studiesIn ?work.\} \\
%\Indm \}
%\label{alg:spa1p2}
%\caption{Consulta SPARQL asociada a la pregunta \textit{Q1.2} �Cu�les son los nombres, sitios web y los lugares donde laboran las personas del RyT?}
%\end{algorithm}

\begin{algorithm}
PREFIX sirp: <http://arte.izt.uam.mx/ontologies/personRyT.owl\#> \\
PREFIX redes: <http://mcyti.izt.uam.mx/arios/odaryt.owl\#> \\
PREFIX rdf: <http://www.w3.org/1999/02/22-rdf-syntax-ns\#> \\
SELECT DISTINCT  ?name ?ws ?work  \\
WHERE \{ \\
\Indp ?x	sirp:has-name ?name\;
			sirp:has-webSite ?ws\;
			sirp:has-workplace ?work. \\
\Indm \}
\label{alg:spa1p2}
\caption{Consulta SPARQL asociada a la pregunta \textit{Q1.2} �Cu�les son los nombres, sitios web y los lugares donde laboran las personas del RyT?}
\end{algorithm}

\begin{algorithm}
PREFIX sirp: <http://arte.izt.uam.mx/ontologies/personRyT.owl\#> \\
PREFIX redes: <http://mcyti.izt.uam.mx/arios/odaryt.owl\#> \\
PREFIX rdf: <http://www.w3.org/1999/02/22-rdf-syntax-ns\#> \\
SELECT ?name ?ws \\
WHERE \{ \\
\Indp ?x	sirp:has-name ?name\;
			sirp:has-webSite ?ws\;
			sirp:has-age ?edad.\\
		FILTER (45 > ?edad \&\& ?edad > 20) \\
\Indm \}
\label{alg:spa1p3}
\caption{Consulta SPARQL asociada a la pregunta \textit{Q1.3} �Qui�nes son mayores de 20 a�os y menores de 45 a�os?}
\end{algorithm}

%\begin{algorithm}
%PREFIX sirp: <http://arte.izt.uam.mx/ontologies/personRyT.owl\#> \\
%PREFIX redes: <http://mcyti.izt.uam.mx/arios/odaryt.owl\#> \\
%PREFIX rdf: <http://www.w3.org/1999/02/22-rdf-syntax-ns\#> \\
%SELECT DISTINCT ?name ?ws \\
%WHERE \{ \\
%\Indp \{?x rdf:type sirp:Teacher.\} UNION \\
%	\{?x rdf:type sirp:Employee.\} \\
%		?x	sirp:has-name ?name\;
%			sirp:has-webSite ?ws. \\
%\Indm \}
%\label{alg:spa1p4}
%\caption{Consulta SPARQL asociada a la pregunta \textit{Q1.4} �Cu�les son los nombres y sitios web de los profesionistas del �rea de RyT?}
%\end{algorithm}

\begin{algorithm}
PREFIX sirp: <http://arte.izt.uam.mx/ontologies/personRyT.owl\#> \\
PREFIX redes: <http://mcyti.izt.uam.mx/arios/odaryt.owl\#> \\
PREFIX rdf: <http://www.w3.org/1999/02/22-rdf-syntax-ns\#> \\
SELECT DISTINCT ?name ?ws \\
WHERE \{ \\
\Indp ?x 	rdf:type sirp:Professional\;
			sirp:has-name ?name\;
			sirp:has-webSite ?ws.\\
\Indm \}
\label{alg:spa1p4}
\caption{Consulta SPARQL asociada a la pregunta \textit{Q1.4} �Cu�les son los nombres y sitios web de los profesionistas del �rea de RyT?}
\end{algorithm}

\begin{algorithm}
PREFIX sirp: <http://arte.izt.uam.mx/ontologies/personRyT.owl\#> \\
PREFIX redes: <http://mcyti.izt.uam.mx/arios/odaryt.owl\#> \\
PREFIX rdf: <http://www.w3.org/1999/02/22-rdf-syntax-ns\#> \\
SELECT DISTINCT ?name ?ws \\
WHERE \{ \\
\Indp ?x	sirp:has-gender sirp:Male\;
			sirp:has-name ?name\;
			sirp:has-webSite ?ws\;
			sirp:worksIn sirp:ClarkParsia.\\
\Indm \}
\label{alg:spa1p5}
\caption{Consulta SPARQL asociada a la pregunta \textit{Q1.5} �Qui�nes trabajan en la Clark \& Parsia y son del sexo Masculino?}
\end{algorithm}

\begin{algorithm}
PREFIX sirp: <http://arte.izt.uam.mx/ontologies/personRyT.owl\#> \\
PREFIX redes: <http://mcyti.izt.uam.mx/arios/odaryt.owl\#> \\
PREFIX rdf: <http://www.w3.org/1999/02/22-rdf-syntax-ns\#> \\
SELECT DISTINCT ?name ?ws \\
WHERE \{ \\
\Indp ?x	rdf:type sirp:Student\;
			sirp:reads sirp:English\;
			sirp:has-name ?name\;
			sirp:has-webSite ?ws.\\
\Indm \}
\label{alg:spa1p6}
\caption{Consulta SPARQL asociada a la pregunta \textit{Q1.6} �Qui�nes son estudiantes y leen en ingl�s?}
\end{algorithm}

\begin{algorithm}
PREFIX sirp: <http://arte.izt.uam.mx/ontologies/personRyT.owl\#> \\
PREFIX redes: <http://mcyti.izt.uam.mx/arios/odaryt.owl\#> \\
PREFIX rdf: <http://www.w3.org/1999/02/22-rdf-syntax-ns\#> \\
SELECT DISTINCT ?name ?ws \\
WHERE \{ \\
\Indp ?x	sirp:reads sirp:English\;
			sirp:speaks sirp:English\;
			sirp:writes sirp:English\;
			sirp:has-name ?name\;
			sirp:has-webSite ?ws.\\
\Indm \}
\label{alg:spa1p7}
\caption{Consulta SPARQL asociada a la pregunta \textit{Q1.7} �Quienes hablan, leen y escriben en ingl�s?}
\end{algorithm}

%\begin{algorithm}
%PREFIX sirp: <http://arte.izt.uam.mx/ontologies/personRyT.owl\#> \\
%PREFIX redes: <http://mcyti.izt.uam.mx/arios/odaryt.owl\#> \\
%PREFIX rdf: <http://www.w3.org/1999/02/22-rdf-syntax-ns\#> \\
%SELECT DISTINCT ?name ?ws \\
%WHERE \{ \\
%\Indp \{?x sirp:reads sirp:German.\} UNION\\
%	  \{?x sirp:speaks sirp:German.\} UNION\\
%	  \{?x sirp:writes sirp:German.\} \\
%		?x	rdf:type sirp:Student\;
%			sirp:has-name ?name\;
%			sirp:has-webSite ?ws.\\
%\Indm \}
%\label{alg:spa1p8}
%\caption{Consulta SPARQL asociada a la pregunta \textit{Q1.8} �Qu� estudiantes saben algo de ingl�s?}
%\end{algorithm}

\begin{algorithm}
PREFIX sirp: <http://arte.izt.uam.mx/ontologies/personRyT.owl\#> \\
PREFIX redes: <http://mcyti.izt.uam.mx/arios/odaryt.owl\#> \\
PREFIX rdf: <http://www.w3.org/1999/02/22-rdf-syntax-ns\#> \\
SELECT DISTINCT ?name ?ws \\
WHERE \{ \\
\Indp ?x 	sirp:has-language sirp:German\;
			rdf:type sirp:Student\;
			sirp:has-name ?name\;
			sirp:has-webSite ?ws.\\
\Indm \}
\label{alg:spa1p8}
\caption{Consulta SPARQL asociada a la pregunta \textit{Q1.8} �Qu� estudiantes saben algo de ingl�s?}
\end{algorithm}

\begin{algorithm}
PREFIX sirp: <http://arte.izt.uam.mx/ontologies/personRyT.owl\#> \\
PREFIX redes: <http://mcyti.izt.uam.mx/arios/odaryt.owl\#> \\
PREFIX rdf: <http://www.w3.org/1999/02/22-rdf-syntax-ns\#> \\
SELECT DISTINCT ?name ?ws \\
WHERE \{ \\
\Indp ?x 	sirp:competentIn sirp:Synthesis\_Skills\;
			rdf:type sirp:Teacher\;
			sirp:has-name ?name\;
			sirp:has-webSite ?ws.\\
\Indm \}
\label{alg:spa1p9}
\caption{Consulta SPARQL asociada a la pregunta \textit{Q1.9} �Qu� profesores tienen la capacidad de s�ntesis?}
\end{algorithm}

%\begin{algorithm}
%PREFIX sirp: <http://arte.izt.uam.mx/ontologies/personRyT.owl\#> \\
%PREFIX redes: <http://mcyti.izt.uam.mx/arios/odaryt.owl\#> \\
%PREFIX rdf: <http://www.w3.org/1999/02/22-rdf-syntax-ns\#> \\
%SELECT DISTINCT ?name ?ws \\
%WHERE \{ \\
%\Indp \{?x rdf:type sirp:Teacher.\} UNION \\
%	  \{?x rdf:type sirp:Employee.\} \\
%		?x	sirp:expertiseIn redes:Semantic\_Web\;
%			sirp:has-name ?name\;
%			sirp:has-webSite ?ws.\\
%\Indm \}
%\label{alg:spa1p10}
%\caption{Consulta SPARQL asociada a la pregunta \textit{Q1.10} �Qu� profesionistas tienen conocimiento en los temas de Web Sem�ntica?}
%\end{algorithm}

\begin{algorithm}
PREFIX sirp: <http://arte.izt.uam.mx/ontologies/personRyT.owl\#> \\
PREFIX redes: <http://mcyti.izt.uam.mx/arios/odaryt.owl\#> \\
PREFIX rdf: <http://www.w3.org/1999/02/22-rdf-syntax-ns\#> \\
PREFIX rdfs: <http://www.w3.org/2000/01/rdf-schema\#>\\
SELECT DISTINCT ?name ?ws \\
WHERE \{ \\
\Indp ?k 	rdfs:subClassOf redes:Semantic\_Web.\\
	  ?x 	rdf:type sirp:Professional\;
			sirp:expertiseIn ?k\;
			sirp:has-name ?name\;
			sirp:has-webSite ?ws.\\
\Indm \}
\label{alg:spa1p10}
\caption{Consulta SPARQL asociada a la pregunta \textit{Q1.10} �Qu� profesionistas tienen conocimiento en los temas de Web Sem�ntica?}
\end{algorithm}

\begin{algorithm}
PREFIX sirp: <http://arte.izt.uam.mx/ontologies/personRyT.owl\#> \\
PREFIX redes: <http://mcyti.izt.uam.mx/arios/odaryt.owl\#> \\
PREFIX rdf: <http://www.w3.org/1999/02/22-rdf-syntax-ns\#> \\
SELECT DISTINCT ?name ?ws \\
WHERE \{ \\
\Indp ?x	rdf:type sirp:Teacher\;
			sirp:expertiseIn redes:Distributed\_Systems\;
			sirp:has-name ?name\;
			sirp:has-webSite ?ws.\\
\Indm \}
\label{alg:spa1p11}
\caption{Consulta SPARQL asociada a la pregunta \textit{Q1.11} �Qu� profesores tienen conocimientos en Sistemas Distribuidos?}
\end{algorithm}

\begin{algorithm}
PREFIX sirp: <http://arte.izt.uam.mx/ontologies/personRyT.owl\#> \\
PREFIX redes: <http://mcyti.izt.uam.mx/arios/odaryt.owl\#> \\
PREFIX rdf: <http://www.w3.org/1999/02/22-rdf-syntax-ns\#> \\
SELECT DISTINCT ?name ?ws \\
WHERE \{ \\
\Indp ?x 	sirp:expertiseIn redes:Java, redes:OWL, redes:RDF, \\ 
\Indp				redes:Threads, redes:C, redes:OpenMP\;
\Indm		sirp:has-name ?name\;
			sirp:has-webSite ?ws.\\
\Indm \}
\label{alg:spa1p12}
\caption{Consulta SPARQL asociada a la pregunta \textit{Q1.12} �Qui�nes tienen conocimiento en Java, OWL, RDF, Threads, C, OpenMP?}
\end{algorithm}

%\begin{algorithm}
%PREFIX sirp: <http://arte.izt.uam.mx/ontologies/personRyT.owl\#> \\
%PREFIX redes: <http://mcyti.izt.uam.mx/arios/odaryt.owl\#> \\
%PREFIX rdf: <http://www.w3.org/1999/02/22-rdf-syntax-ns\#> \\
%SELECT DISTINCT ?name ?ws \\
%WHERE \{ \\
%\Indp \{?x	sirp:expertiseIn redes:Invocation.\} UNION \\
%	  \{?x sirp:expertiseIn redes:Process\_Management.\} UNION\\
%	  \{?x sirp:expertiseIn redes:Processes.\} UNION\\
%	  \{?x sirp:expertiseIn redes:Threads.\} UNION\\
%	  \{?x sirp:expertiseIn redes:Operating\_System.\} \\
%		?x	sirp:has-name ?name\;
%			sirp:has-webSite ?ws.\\
%\Indm \}
%\label{alg:spa1p13}
%\caption{Consulta SPARQL asociada a la pregunta \textit{Q1.13} �Qu� estudiantes tienen alg�n conocimiento en los subtemas de Sistemas Operativos?}
%\end{algorithm}

\begin{algorithm}
PREFIX sirp: <http://arte.izt.uam.mx/ontologies/personRyT.owl\#> \\
PREFIX redes: <http://mcyti.izt.uam.mx/arios/odaryt.owl\#> \\
PREFIX rdf: <http://www.w3.org/1999/02/22-rdf-syntax-ns\#> \\
PREFIX rdfs: <http://www.w3.org/2000/01/rdf-schema\#>\\
SELECT DISTINCT ?name ?ws \\
WHERE \{ \\
\Indp ?k	rdfs:subClassOf redes:Operating\_system.\\
	  ?x	sirp:expertiseIn ?k\;
			sirp:has-name ?name\;
			sirp:has-webSite ?ws.\\
\Indm \}
\label{alg:spa1p13}
\caption{Consulta SPARQL asociada a la pregunta \textit{Q1.13} �Qu� estudiantes tienen alg�n conocimiento en los subtemas de Sistemas Operativos?}
\end{algorithm}

%\begin{algorithm}
%PREFIX sirp: <http://arte.izt.uam.mx/ontologies/personRyT.owl\#> \\
%PREFIX redes: <http://mcyti.izt.uam.mx/arios/odaryt.owl\#> \\
%PREFIX rdf: <http://www.w3.org/1999/02/22-rdf-syntax-ns\#> \\
%SELECT DISTINCT ?name ?ws \\
%WHERE \{ \\
%\Indp \{?x	sirp:worksIn sirp:UAM.\} UNION \\
%	  \{?x	sirp:worksIn sirp:UNAM.\} UNION \\
%	  \{?x	sirp:worksIn sirp:IPN.\}\\
%			?x	sirp:has-name ?name\;
%				sirp:has-webSite ?ws.\\
%\Indm \}
%\label{alg:spa1p14a}
%\caption{Consulta SPARQL sin inferencia asociada a la pregunta \textit{Q1.14} �Qui�nes trabajan en una Universidad?}
%\end{algorithm}

\clearpage

\begin{algorithm}
PREFIX sirp: <http://arte.izt.uam.mx/ontologies/personRyT.owl\#> \\
PREFIX redes: <http://mcyti.izt.uam.mx/arios/odaryt.owl\#> \\
PREFIX rdf: <http://www.w3.org/1999/02/22-rdf-syntax-ns\#> \\
SELECT DISTINCT ?name ?ws \\
WHERE \{ \\
\Indp ?k rdf:type sirp:University. \\
	  ?x	sirp:worksIn ?k\;
			sirp:has-name ?name\;
			sirp:has-webSite ?ws.\\
\Indm \}
\label{alg:spa1p14}
\caption{Consulta SPARQL asociada a la pregunta \textit{Q1.14} �Qui�nes trabajan en una Universidad?}
\end{algorithm}

\begin{algorithm}
PREFIX sirp: <http://arte.izt.uam.mx/ontologies/personRyT.owl\#> \\
PREFIX redes: <http://mcyti.izt.uam.mx/arios/odaryt.owl\#> \\
PREFIX rdf: <http://www.w3.org/1999/02/22-rdf-syntax-ns\#> \\
SELECT DISTINCT ?name ?ws \\
WHERE \{ \\
\Indp ?x 	sirp:has-workplace sirp:UAM\;
			sirp:expertiseIn redes:Semantic\_Web\;
			sirp:has-name ?name\;
			sirp:has-webSite ?ws.\\
\Indm \}
\label{alg:spa1p15}
\caption{Consulta SPARQL asociada a la pregunta \textit{Q1.15} �Quienes laboran en la UAM y tienen alg�n conocimiento en  Web Sem�ntica?}
\end{algorithm}

\begin{algorithm}
PREFIX sirp: <http://arte.izt.uam.mx/ontologies/personRyT.owl\#> \\
PREFIX rdf: <http://www.w3.org/1999/02/22-rdf-syntax-ns\#> \\
SELECT DISTINCT ?name ?ws \\
WHERE \{ \\
\Indp ?x sirp:has-advisor ?y.\\
	  ?y sirp:has-name ?z.\\
	  FILTER regex(?z, ``Carolina Medina'',``i'')\\
	  ?x sirp:has-name ?name\;
	  sirp:has-webSite ?ws. \}
\label{alg:spa1p16}
\caption{Consulta SPARQL asociada a la pregunta \textit{Q1.16} �Qu� personas tienen como asesor a Carolina Medina?}
\end{algorithm}

\begin{algorithm}
PREFIX sirp: <http://arte.izt.uam.mx/ontologies/personRyT.owl\#> \\
PREFIX redes: <http://mcyti.izt.uam.mx/arios/odaryt.owl\#> \\
PREFIX rdf: <http://www.w3.org/1999/02/22-rdf-syntax-ns\#> \\
SELECT DISTINCT ?name ?ws \\
WHERE \{ \\
\Indp ?y sirp:has-name ?z.\\
	  FILTER regex(?z, ``Ricardo Marcelin'', ``i'')\\
	  ?x	sirp:has-colleague ?y\;
			sirp:has-name ?name\;
			sirp:has-webSite ?ws.\\
\Indm \}
\label{alg:spa1p17}
\caption{Consulta SPARQL asociada a la pregunta \textit{Q1.17} �Qui�nes son los colegas de Ricardo Marcelin?}
\end{algorithm}

\begin{algorithm}
PREFIX sirp: <http://arte.izt.uam.mx/ontologies/personRyT.owl\#> \\
PREFIX rdf: <http://www.w3.org/1999/02/22-rdf-syntax-ns\#> \\
SELECT DISTINCT ?name ?ws \\
WHERE \{ \\
\Indp ?y sirp:has-name ?z.\\
	  FILTER regex(?z, ``Carolina Medina Ramirez'',``i'')\\
	  ?x 	sirp:knows ?y\;
			sirp:has-name ?name\;
			sirp:has-webSite ?ws. \}
\label{alg:spa1p18}
\caption{Consulta SPARQL asociada a la pregunta \textit{Q1.18} �Cu�les son los nombres y correos de las personas que conocen a Carolina Medina Ram�rez?}
\end{algorithm}

\begin{algorithm}
PREFIX sirp: <http://arte.izt.uam.mx/ontologies/personRyT.owl\#> \\
PREFIX rdf: <http://www.w3.org/1999/02/22-rdf-syntax-ns\#> \\
SELECT DISTINCT ?name ?ws \\
WHERE \{ \\
\Indp ?x	rdf:type sirp:Researcher\;
			sirp:has-name ?name\;
			sirp:has-webSite ?ws.\}
\label{alg:spa1p19}
\caption{Consulta SPARQL asociada a la pregunta \textit{Q1.19} �Qu� personas son profesores-investigadores?}
\end{algorithm}

\begin{algorithm}
PREFIX sird: <http://arte.izt.uam.mx/ontologies/digiResourceRyT.owl\#>\\
PREFIX redes: <http://mcyti.izt.uam.mx/arios/odaryt.owl\#>\\
PREFIX rdf: <http://www.w3.org/1999/02/22-rdf-syntax-ns\#>\\
SELECT ?title ?path ?ext ?lang \\
WHERE \{ \\
\Indp ?x	sird:has-title ?title\;
			sird:has-filePath ?path\;
			sird:has-fileExtension ?ext\;
			sird:has-languageSource ?lang.\\
\Indm \}
\label{alg:spa2p1}
\caption{Consulta SPARQL asociada a la pregunta \textit{Q2.1} �Cu�les son los t�tulos, rutas, extensi�n, idioma de todos los recursos digitales de RyT?}
\end{algorithm}

\begin{algorithm}
PREFIX sird: <http://arte.izt.uam.mx/ontologies/digiResourceRyT.owl\#>\\
PREFIX redes: <http://mcyti.izt.uam.mx/arios/odaryt.owl\#>\\
PREFIX rdf: <http://www.w3.org/1999/02/22-rdf-syntax-ns\#>\\
SELECT DISTINCT ?title ?path \\
WHERE \{ \\
\Indp ?x	rdf:type sird:Book\;
			sird:has-title ?title\;
			sird:has-filePath ?path\;
			sird:has-topic ?k.\\
	  ?k 	rdfs:subClassOf redes:Distributed\_Systems.\\
\Indm \}
\label{alg:spa2p2}
\caption{Consulta SPARQL asociada a la pregunta \textit{Q2.2} �Cu�les libros tratan sobre algunos temas de Sistemas Distribuidos?}
\end{algorithm}

\begin{algorithm}
PREFIX sird: <http://arte.izt.uam.mx/ontologies/digiResourceRyT.owl\#>\\
PREFIX redes: <http://mcyti.izt.uam.mx/arios/odaryt.owl\#>\\
PREFIX rdf: <http://www.w3.org/1999/02/22-rdf-syntax-ns\#>\\
SELECT DISTINCT ?title ?path \\
WHERE \{ \\
\Indp ?x 	sird:publishedIn sirp:UAM\;
			sird:has-filePath ?path\;
			sird:has-title ?title.\\
\Indm \}
\label{alg:spa2p3}
\caption{Consulta SPARQL asociada a la pregunta \textit{Q2.3} �Qu� recursos fueron publicados por la UAM?}
\end{algorithm}

\begin{algorithm}
PREFIX sird: <http://arte.izt.uam.mx/ontologies/digiResourceRyT.owl\#>\\
PREFIX redes: <http://mcyti.izt.uam.mx/arios/odaryt.owl\#>\\
PREFIX rdf: <http://www.w3.org/1999/02/22-rdf-syntax-ns\#>\\
SELECT DISTINCT ?title ?path \\
WHERE \{ \\
\Indp ?x 	rdf:type sird:Document\;
			sird:has-topic redes:Peer\_to\_Peer\_Systems\;
			sird:has-title ?title\;
			sird:has-filePath ?path.\\
\Indm \}
\label{alg:spa2p4}
\caption{Consulta SPARQL asociada a la pregunta \textit{Q2.4} �Qu� documentos sirven para dar un curso de Sistemas P2P?}
\end{algorithm}

\begin{algorithm}
PREFIX sird: <http://arte.izt.uam.mx/ontologies/digiResourceRyT.owl\#>\\
PREFIX redes: <http://mcyti.izt.uam.mx/arios/odaryt.owl\#>\\
PREFIX rdf: <http://www.w3.org/1999/02/22-rdf-syntax-ns\#>\\
SELECT DISTINCT ?title ?path \\
WHERE \{ \\
\Indp ?x	rdf:type sird:Multimedia\;
			sird:has-year ?year\;
		  	sird:has-title ?title\;
		  	sird:has-filePath ?path.\\
		FILTER (?year > 2009)\\
\Indm \}
\label{alg:spa2p5}
\caption{Consulta SPARQL asociada a la pregunta \textit{Q2.5} �Qu� recursos multimedia son mayores al a�o 2009?}
\end{algorithm}

\begin{algorithm}
PREFIX sird: <http://arte.izt.uam.mx/ontologies/digiResourceRyT.owl\#>\\
PREFIX redes: <http://mcyti.izt.uam.mx/arios/odaryt.owl\#>\\
PREFIX rdf: <http://www.w3.org/1999/02/22-rdf-syntax-ns\#>\\
SELECT DISTINCT ?title ?path \\
WHERE \{ \\
\Indp ?x	rdf:type sird:Document\;
			sird:has-topic redes:Ontology\;
			sird:has-title ?title\;
			sird:has-filePath ?path.\\
\Indm \}
\label{alg:spa2p6}
\caption{Consulta SPARQL asociada a la pregunta \textit{Q2.6} �Cu�les documentos tratan sobre Ontolog�as?}
\end{algorithm}

\begin{algorithm}
PREFIX sird: <http://arte.izt.uam.mx/ontologies/digiResourceRyT.owl\#>\\
PREFIX redes: <http://mcyti.izt.uam.mx/arios/odaryt.owl\#>\\
PREFIX rdf: <http://www.w3.org/1999/02/22-rdf-syntax-ns\#>\\
SELECT DISTINCT ?title ?path \\
WHERE \{ \\
\Indp ?x	sird:publishedInJournal ?k\;
			sird:has-title ?title\;
			sird:has-filePath ?path.\\
\Indm \}
\label{alg:spa2p7}
\caption{Consulta SPARQL asociada a la pregunta \textit{Q2.7} �Qu� recursos fueron publicados en una Revista cient�fica?}
\end{algorithm}

\begin{algorithm}
PREFIX sird: <http://arte.izt.uam.mx/ontologies/digiResourceRyT.owl\#>\\
PREFIX redes: <http://mcyti.izt.uam.mx/arios/odaryt.owl\#>\\
PREFIX rdf: <http://www.w3.org/1999/02/22-rdf-syntax-ns\#>\\
SELECT DISTINCT ?title ?path \\
WHERE \{ \\
\Indp ?x sird:has-compendium ?txt \\
	  FILTER regex(?txt,``state of the art'',``i'')\\
	  ?x	sird:has-title ?title\;
			sird:has-filePath ?path.\\
\Indm \}
\label{alg:spa2p8}
\caption{Consulta SPARQL asociada a la pregunta \textit{Q2.8} �Qu� recursos tienen en su descripci�n las palabras ``linked data''?}
\end{algorithm}

\begin{algorithm}
PREFIX sird: <http://arte.izt.uam.mx/ontologies/digiResourceRyT.owl\#>\\
PREFIX redes: <http://mcyti.izt.uam.mx/arios/odaryt.owl\#>\\
PREFIX rdf: <http://www.w3.org/1999/02/22-rdf-syntax-ns\#>\\
SELECT DISTINCT ?title ?path \\
WHERE \{ \\
\Indp ?x	rdf:type sird:Document\;
			sird:has-year ?year.\\
	  FILTER (?year > 2000)\\
	  ?x	sird:has-languageSource sirp:English\;
			sird:has-author ?z\;
			sird:has-title ?title\;
			sird:has-filePath ?path.\\
	  FILTER regex(?z, ``Erik Alarcon Zamora'', ``i'')\\
\Indm \}
\label{alg:spa2p9}
\caption{Consulta SPARQL asociada a la pregunta \textit{Q2.9} �Cu�les documentos en Ingl�s y mayores al a�o 2000 son de autor�a de Erik Alarcon Zamora?}
\end{algorithm}

\begin{algorithm}
PREFIX sird: <http://arte.izt.uam.mx/ontologies/digiResourceRyT.owl\#>\\
PREFIX redes: <http://mcyti.izt.uam.mx/arios/odaryt.owl\#>\\
PREFIX rdf: <http://www.w3.org/1999/02/22-rdf-syntax-ns\#>\\
SELECT DISTINCT ?title ?path \\
WHERE \{ \\
\Indp ?x	rdf:type sird:Thesis\;
			sird:has-author ?y\;
			sird:has-title ?title\;
			sird:has-filePath ?path.\\
	  FILTER regex(?y, ``Samuel Hernandez Maza'',``i'')\\
\Indm \}
\label{alg:spa2p10}
\caption{Consulta SPARQL asociada a la pregunta \textit{Q2.10} �Cu�l son las tesis de Samuel Hernandez Maza?}
\end{algorithm}
\end{appendices}

% *************** Bibliography ***************
%\begin{thebibliography}{99}
\bibitem{yianilos}

P. {N}. Yianilos and S. Sobti, \emph{The Envolving Field of Distributed Storage}, IEEE Computing, September-October 2001, pp. 35-39.

\bibitem{rhea}
%S. Rhea, et al, \emph{Maintenance-Free Global Storage}, IEEE Internet Computing, September-October 2001, pp. 40-40.

\bibitem{kubiatowicz}
%J. Kubiatowicz, \emph{OceanStore: An Architecture for Global-Scale Persistent Storage}, ASPLOS 2000, ACM.


\end{thebibliography}
\bibliographystyle{unsrt}
{\small\bibliography{bibliografia}}

\end{document}
