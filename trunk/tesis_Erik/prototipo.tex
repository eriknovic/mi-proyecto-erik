\chapter{Prototipo}
\label{cap:piu}
Cualquier usuario no puede realizar el proceso de consulta (interacci�n) de informaci�n (integraci�n ISR) en las ontolog�as de la \textit{cartograf�a de competencias} y \textit{b�squeda de recursos digitales}. Porque los usuarios deben tener un conocimiento b�sico en: las tecnolog�as sem�nticas (tripletas y consultas SPARQL), el vocabulario utilizado en las tripletas (recursos y propiedades), manejo del triplestore (carga, inferencia ).

La integraci�n sem�ntica de recursos (ISR) a partir del uso de un
triplestore, no es una tarea que cualquier usuario (profesor o estudiante) puede hacer,
ya que �ste debe estar familiarizado con el triplestore, el
lenguaje de consulta SPARQL y las ternas RDF.


La manera de probar el enfoque sem�ntico y la integraci�n sem�ntica es desarrollar un prototipo de sistema.

Este prototipo basa su funcionamiento en los elementos, tecnolog�as y est�ndares actuales de la Web Sem�ntica.

Nosotros para construir el sistema seguimos una serie de actividades. En donde estas actividades se apegan a la metodolog�a que propusimos.

%%%%%%%%%%%%%%%%%%%%%%%%%%%
La integraci�n sem�ntica es un proceso de b�squeda y recuperaci�n de informaci�n sobre los recursos de una Memoria Corporativa, bajo el enfoque de las Tecnolog�as Sem�nticas. Esta integraci�n se basa en la representaci�n del conocimiento de los recursos en forma de un grafo RDF y en la consulta del mismo, para satisfacer la pregunta de un usuario.
En esta integraci�n sem�ntica, el grafo se constituye de triples y las consultas est�n est�n compuestas de patrones parecidos a triples. Ahora bien, cualquier usuario del �rea de Redes y Telecomunicaciones (RyT) que quiera consultar el modelo RDF, debe aprender a construir consultas SPARQL y tener un conocimiento general sobre los triples del modelo RDF. Sin embargo, esta situaci�n puede ser molesta para los usuarios. As� que, nosotros proponemos un prototipo para la interacci�n amigable de los usuarios con el modelo RDF, de esta forma, un usuario podr� consultar al modelo RDF y visualizar los resultados asociados a la misma, sin que �ste tenga conocimientos en las Tecnolog�as Sem�nticas.
Los objetivos particulares del prototipo son: 1) permitir a los usuarios estructurar su pregunta, para que se mapee a una consulta SPARQL. 2) cargar el modelo RDF y los axiomas, e invocar el razonador para inferir nuevas relaciones al grafo RDF. 3) invocar el motor de consulta SPARQL, para que ejecute la consulta SPARQL al grafo RDF inferido. 4) mostrar para cada resultado un conjunto de datos significativos (nombre, ruta, lenguaje, etc.). A partir de estos objetivos se plantea la siguiente arquitectura.
%%%%%%%%%%%%%%%%%%%%%%%%%%%

%%No solo basta con un razonador también se requiere un modulo integrador de la información que transforme las consultas de los usuarios expresadas en lenguaje natural, a lenguaje que sea interpretado por el razonador. Además este integrador es el encargado de regresar los enlaces y los datos del recurso. De esta manera el motor de búsqueda queda de la siguiente manera:

%%Para llevar acabo está integración de la información es necesario un prototipo que satisfaga más eficientemente las consultas que escribe el usuario. De esta manera es necesario un análisis documental y técnico de los módulos de Anotaciones, Ontología de Dominio y Razonadores. Con la finalidad de tener una propuesta de sistema con las últimas novedades hechas en búsqueda y recuperación de la información basada en la semántica de los recursos.

%%se requiere proporcionar interfaces fáciles de usar, para simplificar a los miembros el proceso de generación de anotaciones y colocar en contexto su trabajo.  Un buen enfoque para un sistema de anotaciones es aquel donde se maneja una única interfaz, y es en esta donde los usuarios crean, modifican y comparten sus anotaciones.

%%La interfaz debe tener ..., para que los usuarios estructuren sus consulta, capturando los valores que desean buscar. En la interfaz debe proporcionar un navegador entre personas, documentos, multimedia, para que los usuarios que no tienen algún conocimiento previo de las personas y recursos digitales, puedan tener una vision general de la información de los recursos.

%%La interfaz debe permitir hacer las siguientes actividades a los usuarios:
%%•	Login.
%%•	Navegar entre personas.
%%o	Filtrar por ocupación.
%%o	Mostrar la información más detallada de una persona.
%%o	Búsqueda Avanzada de las personas.
%%•	Navegar entre documentos.
%%o	Filtrar por clase de documento.
%%o	Mostrar la información más detallada de un documento.
%%o	Búsqueda Avanzada de documentos.
%%•	Navegar entre multimedia.
%%o	Filtrar por clase multimedia.
%%o	Mostrar la información más detallada de un recurso multimedia.
%%o	Búsqueda Avanzada de recursos multimedia.
%%•	Búsqueda en todos los recursos de información por información semejante.

%%Las distintas aplicaciones que hay pueden ser elaboradas para Windows, Linux, Macs, u otros sistemas operativos, y también se pueden tener distintas versiones del mismo sistema para poder trabajar entre los distintos sistemas operativos. Lo ideal es que sin importar cual sea el sistema operativo el usuario pueda realizar sus anotaciones semánticas, para logara esto se puede emplear una aplicación web o aplicaciones elaboradas en java.
%%Si se emplea una aplicación web no es necesario instalar algún software extra, solo basta que el usuario acceda utilizando su navegador web de preferencia y comience el proceso de creación de anotaciones semánticas.

%%Por otro lado al usar una aplicación basada en Java, es necesario tener Java Development Kit (JDK) que es independiente de la plataforma, para tener un entorno amigable al usuario.

%%Finalmente, los usuarios necesitan una interfaz de usuario, para la consulta de información de las tripletas. Nosotros proponemos una interfaz amigable que sea accesible vía Web. De esta manera, los usuarios no instalan ningún componente y simplemente acceden a la página Web del sistema.
%%La interfaz al ser accesible vía Web, requiere ser instalada en un servidor Web. Para tomar la decisión sobre qué servidor es el apropiado para la interfaz. Nosotros debemos tomar en cuenta, el lenguaje de implementación del triplestore. Si el lenguaje es PHP9, entonces, podemos emplear un servidor HTTP Apache10. En otro caso, si el lenguaje es Java11 y permite implementar Servlet, entonces el servidor es Apache Tomcat12. 
