\chapter*{Abstract}
En una instituci�n educativa que se encarga de la formaci�n de profesionales, un elemento importante es la generaci�n y gesti�n de conocimiento. En particular, esta tesis va dirigida a gestionar el conocimiento del �rea de Redes y Telecomunicaciones (RyT) del departamento de ingenier�a el�ctrica de la Universidad Aut�noma Metropolitana. Esta �rea tiene una amplia y rica variedad de recursos de informaci�n que representan el conocimiento sobre investigaciones, colaboraciones, proyectos, cursos, tambi�n temas de inter�s de los profesores y alumnos en el dominio RyT. Esta tesis propone y describe una metodolog�a basada en el uso de tecnolog�as sem�nticas para integrar el conocimiento de los recursos de informaci�n en una organizaci�n. La metodolog�a est� dividida en tres etapas. La primera etapa es la representaci�n del conocimiento sobre los recursos de informaci�n en un modelo. La segunda  etapa es la introducci�n de reglas de inferencia en el modelo, para describir el conocimiento impl�cito. La tercera etapa es la b�squeda y recuperaci�n inteligente de informaci�n en el modelo. Tambi�n, se presenta un prototipo que muestra la viabilidad del enfoque sem�ntico en los recursos. Este prototipo es una interfaz gr�fica de usuario, cuya finalidad es que cualquier usuario pueda recuperar y visualizar la informaci�n desde el modelo. Finalmente, este trabajo expone una evaluaci�n experimental sobre la calidad de resultados y el tiempo de desempe�o durante el proceso de consulta de informaci�n. \\ \\ \\ \\

\textbf{Keywords:} \textit{Semantic technologies}, \textit{information resources}, \textit{knowledge base}, \textit{knowledge representation}, \textit{axioms}, \textit{search and information retrieval}, \textit{corporate memory}, \textit{ontology}, \textit{networks and telecommunications}, \textit{RDF}, \textit{RDF(S)}, \textit{OWL}, \textit{SPARQL}.