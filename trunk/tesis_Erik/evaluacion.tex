\chapter{Evaluaci�n experimental}
\label{cap:exp}

%http://hal-lirmm.ccsd.cnrs.fr/docs/00/63/97/05/PDF/A_Flexible_System_for_Ontology_Matching.pdf

%%Precisi�n
%%?Es la proporci�n del material recuperado que realmente es relevante, del total de documentos recuperados? [2, 3]. En donde la precisi�n est� dada por la relaci�n entre Documentos Relevantes Recuperados y Documentos recuperados. Y cuyo intervalo est� entre el cero y uno.
%%Presisi�n =  (Documentos Relevantes Recuperados)/(Total de Documentos recuperados)
%%Dibujo de ejemplo
%%Exhaustividad
%%?Proporci�n de material relevante recuperado del total de documentos que son relevantes, Donde la exhaustividad es inversamente proporcional a la precisi�n? [3]. Igual que en la precisi�n el intervalo est� entre el cero y uno.
%%Exhasutividad =  (Documentos Relevantes Recuperados)/(Total de Documentos Relevantes en la MC)


\section{Escenarios de experimentaci�n}
Alg�n texto...

\begin{table}[!htb]
%\renewcommand{\arraystretch}{1.2}
\centering
\begin{tabular}{>{\centering\arraybackslash}m{1in} >{\arraybackslash}m{3.5in} >{\centering\arraybackslash}m{1in}}
\hline 
Id. Consulta & Pregunta & No. de Recursos\\
\hline
\hline 
Q1 & �Cu�les son los t�tulos, rutas, extensi�n, idioma de todos los recursos digitales de RyT? & 1330 \\
\hline
Q2 & �Cu�les libros tratan sobre algunos temas de Sistemas Distribuidos? & 103\\
\hline 
Q3 & �Qu� recursos fueron publicados por la UAM? & 18\\
\hline 
Q4 & �Qu� documentos son para dar un curso de Sistemas P2P? & 31\\
\hline 
Q5 & �Qu� recursos multimedia son mayores al a�o 2009? & 119\\
\hline 
Q6 & �Cu�les documentos tratan sobre Ontolog�as? & 30\\
\hline 
Q7 & �Qu� recursos fueron publicados en una Revista cient�fica? & 156\\
\hline 
Q8 & �Qu� recursos tienen en su contenido las palabras "linked data"? & 159\\
\hline 
Q9 & �Cu�les documentos en ingl�s y mayores al a�o 2000 son de autor�a de Erik Alarc�n Zamora? & 2\\
\hline 
Q10 & �Cu�les la tesis de Samuel Hern�ndez Maza? & 4\\
\hline 
\end{tabular}
\end{table}


\begin{table}[!htb]
\renewcommand{\arraystretch}{1.2}
\centering
\begin{tabular}{| >{\centering\arraybackslash}m{1in} | >{\centering\arraybackslash}m{1in} | >{\centering\arraybackslash}m{1in} | >{\centering\arraybackslash}m{1in} | >{\centering\arraybackslash}m{1in} | }
\hline 
\multirow{2}{*}{Id. Consulta} & \multicolumn{2}{c|}{Modelo (ABox)} & \multicolumn{2}{c|}{Modelo (Razonador+Ontolog�a)}\\
\cline{2-5} 
 & Tiempo promedio (ms) & No. Recursos  & Tiempo promedio (ms) & No. Recursos\\
\hline 
\hline
Q1 & 12 & 1330/1330 & 138 & 1330/1330\\
\hline
Q2 & 10 & 0/103 & 194 & 103/103\\
\hline
Q3 & 8 & 18/18 & 406 & 18/18\\
\hline
Q4 & 28 & 15/31 & 129 & 31/31\\
\hline
Q5 & 7 & 66/119 & 157 & 119/119\\
\hline
Q6 & 9 & 15/30 & 4016 & 30/30\\
\hline
Q7 & 12 & 156/156 & 3520 & 156/156\\
\hline
Q8 & 16 & 159/159 & 3472 & 159/159\\
\hline
Q9 & 42 & 0/2 & 3451 & 2/2\\
\hline
Q10 & 13 & 3/4 & 3312 & 4/4\\
\hline
\end{tabular}
\end{table}

\section{Experimentaci�n}
M�s texto...

\section{Resultados}
M�s texto...
