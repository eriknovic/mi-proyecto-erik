\section{Conclusiones}
\begin{frame}
	\frametitle{Conclusiones}
	\begin{alertblock}{}
	\begin{itemize}%[<+-| alert@+>]
	\item \justifying El uso de las tecnolog�as sem�nticas contribuye a la integraci�n sem�ntica de los recursos de informaci�n en una memoria corporativa.
	\item \justifying El conocimiento impl�cito es determinante para la obtenci�n de \textit{recursos de informaci�n pertinentes}.
	\item \justifying La inferencia no es gratis, tiene costo en tiempo.
	\item \justifying Un marco de referencia, modelo sem�ntico, prototipo y scripts son las contribuciones para la integraci�n sem�ntica de recursos de informaci�n en una memoria corporativa.
	\end{itemize}
	\end{alertblock}
\end{frame}

\begin{frame}
	\frametitle{Recomendaciones}
	\begin{block}{}
	\begin{itemize}%[<+-| alert@+>]
	%\item \justifying Utilizar inferencia para materializar conocimiento impl�cito (impl�cito >> expl�cito).
	\item \justifying Introducir nuevos \textit{casos de uso} para modelar mayor conocimiento.
	\item \justifying Mejorar la seguridad del prototipo y agregar un recuadro para b�squedas por \textit{palabras clave}.
	\item \justifying Construir un m�dulo (aplicaci�n) para generar \textit{tripletas RDF} a partir de las descripciones de los \textit{recursos de informaci�n}.
		\begin{itemize}
		\item \justifying \small Generaci�n guiada por los usuarios.
		\item \justifying \small Generaci�n automatizada.
		\end{itemize}
	\item \justifying Comparar los tiempos de procesamiento y calidad de los recursos con otros triplestores: Stardog y Sesame.
	\end{itemize}	
	\end{block}
\end{frame}